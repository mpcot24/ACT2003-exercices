\chapter{Révision de certains concepts de statistique et tables}
\label{chap:tables}

\section{Quelques distributions bien connues}

\begin{description}
\item[Loi Normale:] Si $Y\sim \mathcal{N}(\mu, \sigma^2)$ avec $\mu \in\mathbb{R}$ et $\sigma^2>0$, alors sa densité est donnée, pour tout $y\in\mathbb{R}$, par
$$
f_{Y}(y)=\frac{1}{\sqrt{2\pi}\sigma}\exp\left\{-\frac{(y-\mu)^2}{2\sigma^2}\right\}.
$$ 
Dans ce cas, $Z=(Y-\mu)/\sigma$ suit une loi normale centrée réduite, notée $\mathcal{N}(0,1)$ ou $N(0,1)$. 

\bigskip

\item[Loi Khi-Carrée:] Si $X\sim \chi^{2}_{(\nu)}$ avec $\nu>0$, sa densité est donnée, pour tout $x \in (0,\infty)$, par
$$
f_{X}(x)=\frac{x^{\nu/2-1}e^{-x/2}}{2^{\nu/2}\Gamma(\nu/2)}.
$$
On note que $\esp{X}=\nu$. Les distributions normale et Khi-carrée sont reliées:

\begin{itemize}
\item [-]
Si $Z\sim \N(0,1)$, alors $Z^2\sim \chi^{2}_{(1)}$. 

\item[-]
Si $Z_1,\ldots,Z_n$ sont mutuellement indépendantes et distribuées selon une loi $\mathcal{N}(0,1)$, alors 
$$
\sum_{i=1}^{n}Z_i^2\sim \chi^2_{(n)}.
$$

\end{itemize}

\bigskip
\item[Loi Student $t$:] Si $T\sim t_{(\nu)}$ avec $\nu>0$, sa densité est donnée, pour tout $t\in\mathbb{R}$, par
$$
f_T(t)=\frac{\Gamma\left(\frac{\nu+1}{2}\right)}{\sqrt{\nu\pi}\Gamma\left(\frac{\nu}{2}\right)}\left(1+\frac{x^2}{\nu}\right)^{-\frac{\nu+1}{2}}.
$$ 
Quand $\nu\rightarrow\infty$, la loi $t$ tend vers la loi normale centrée réduite. Aussi, si $Z\sim \N(0,1)$ et $X\sim \chi^{2}_{(\nu)}$ sont indépendantes, alors on a la représentation stochastique suivante~:
$$
\frac{Z}{\sqrt{X/\nu}}\sim t_{(\nu)}.
$$

\bigskip
\item[Loi de Fisher:] Si $X_1\sim \chi^{2}_{(\nu_1)}$ et $X_2\sim \chi^{2}_{(\nu_2)}$ sont indépendantes et $\nu_1,\nu_2>0$, alors 
$$
\frac{X_1/\nu_1}{X_2/\nu_2}\sim F(\nu_1,\nu_2).
$$ 
Aussi, on peut facilement montrer avec les relations précédentes que si $T\sim t_{(\nu)}$, alors $T^{2}\sim F(1, \nu).$ La densité de la loi de Fisher est complexe et rarement utilisée. Cette distribution a un support positif.
\end{description}

\section{Maximum de vraisemblance}

Soient les observations $y_1,\ldots,y_n$, provenant de variables aléatoires indépendantes avec densités $f_{Y_i}(y_i;\theta),$ pour $i=1,\ldots,n$. La fonction de vraisemblance est 
$$
L(\theta;y_1,\ldots,y_n)=\prod_{i=1}^{n}f_{Y_i}(y_i;\theta).
$$ 
La fonction de log-vraisemblance est 
$$
\ell(\theta;y_1,\ldots,y_n)=\ln L(\theta;y_1,\ldots,y_n)=\sum_{i=1}^{n}\ln f_{Y_i}(y_i;\theta).
$$ 
La méthode du maximum de vraisemblance permet de trouver l'estimateur $\hat{\theta}_n$ qui maximise $L(\theta;y_1,\ldots,y_n)$, et par conséquent $\ell(\theta;y_1,\ldots,y_n)$. On travaille habituellement avec le logarithme naturel pour simplifier les calculs. La fonction de score est 
$$
\dot{\ell}_{\theta}(\theta;y_1,\ldots,y_n)=\frac{\partial}{\partial \theta}\ell(\theta;y_1,\ldots,y_n).
$$ 
L'estimateur du maximum de vraisemblance (EMV, ou MLE pour \emph{maximum likelihood estimator}) est $\hat{\theta}_n$ tel que 
$$
\left.\dot{\ell}_{\theta}(\theta;y_1,...,y_n)\right|_{\hat{\theta}_n}=0.
$$ 


\section{Estimateur sans biais}

Soit un échantillon aléatoire $Y_1,\ldots,Y_n$, avec densité $f(y;\theta)$. Soit l'estimateur de $\theta$ suivant: $\hat{\theta}_n=\hat{\theta}(Y_1,...,Y_n)$. On dit de $\hat{\theta}_n$ qu'il est sans biais si $\esp{\hat{\theta}_n}=\theta.$ Dans ce cas, le biais est zéro, 
$$
\mbox{biais}(\hat{\theta}_n)=\esp{\hat{\theta}_n}-\theta=0,
$$ 
ce qui réduit l'erreur quadratique moyenne (EQM ou \emph{MSE} pour \emph{mean squared error}) de l'estimateur: 
$$
\mbox{EQM}(\hat{\theta}_n)=\esp{(\hat{\theta}_n-\theta)^2}=\var{\hat{\theta}_n}+\mbox{biais}(\hat{\theta}_n).
$$


\newpage
\section{Table de quantiles de la loi khi carré}
\label{chap:khi2}

Le tableau donne $\chi_\alpha^2(\nu)$, le quantile sup\'erieur de niveau $\alpha$ de la loi khi carr\'e avec $\nu$ degr\'es de libert\'e, $\alpha$ est donn\'e dans les colonnes, et $\nu$ est donn\'e dans les lignes. Pr\'ecision : Si $X\sim \chi^2(\nu)$, alors $\Pr\{X> \chi_\alpha^2(\nu)\}=\alpha$.

\addtolength{\textheight}{5mm}
\addtolength{\textwidth}{15mm}
%\begin{landscape}

\centerline{\setlength{\tabcolsep}{2.75pt} \resizebox{!}{\textwidth}{
\begin{sideways}
\centering
\begin{minipage}[t]{\textwidth}
\hspace{-1in}
\begin{small}{
\begin{tabular}{|r|rrrrr||rrrrr|}
\hline
    &  \multicolumn{5}{c||}{Queue de gauche} &  \multicolumn{5}{|c|}{Queue de droite}\\[6pt]
$\nu$  &   0.99500&  0.99000&  0.97500&  0.95000&  0.90000&  0.10000&  0.05000&  0.02500&  0.01000&  0.00500\\
  \hline
  1&  0.00004&  0.00016&  0.00098&  0.00393&  0.01579&  2.70554&  3.84146&  5.02389&  6.63490&  7.87944\\
  2&  0.01003&  0.02010&  0.05064&  0.10259&  0.21072&  4.60517&  5.99146&  7.37776&  9.21034& 10.59663\\
  3&  0.07172&  0.11483&  0.21580&  0.35185&  0.58437&  6.25139&  7.81473&  9.34840& 11.34487& 12.83816\\
  4&  0.20699&  0.29711&  0.48442&  0.71072&  1.06362&  7.77944&  9.48773& 11.14329& 13.27670& 14.86026\\
  5&  0.41174&  0.55430&  0.83121&  1.14548&  1.61031&  9.23636& 11.07050& 12.83250& 15.08627& 16.74960\\
  6&  0.67573&  0.87209&  1.23734&  1.63538&  2.20413& 10.64464& 12.59159& 14.44938& 16.81189& 18.54758\\
  7&  0.98926&  1.23904&  1.68987&  2.16735&  2.83311& 12.01704& 14.06714& 16.01276& 18.47531& 20.27774\\
  8&  1.34441&  1.64650&  2.17973&  2.73264&  3.48954& 13.36157& 15.50731& 17.53455& 20.09024& 21.95495\\
  9&  1.73493&  2.08790&  2.70039&  3.32511&  4.16816& 14.68366& 16.91898& 19.02277& 21.66599& 23.58935\\
 10&  2.15586&  2.55821&  3.24697&  3.94030&  4.86518& 15.98718& 18.30704& 20.48318& 23.20925& 25.18818\\
 11&  2.60322&  3.05348&  3.81575&  4.57481&  5.57778& 17.27501& 19.67514& 21.92005& 24.72497& 26.75685\\
 12&  3.07382&  3.57057&  4.40379&  5.22603&  6.30380& 18.54935& 21.02607& 23.33666& 26.21697& 28.29952\\
 13&  3.56503&  4.10692&  5.00875&  5.89186&  7.04150& 19.81193& 22.36203& 24.73560& 27.68825& 29.81947\\
 14&  4.07467&  4.66043&  5.62873&  6.57063&  7.78953& 21.06414& 23.68479& 26.11895& 29.14124& 31.31935\\
 15&  4.60092&  5.22935&  6.26214&  7.26094&  8.54676& 22.30713& 24.99579& 27.48839& 30.57791& 32.80132\\
 16&  5.14221&  5.81221&  6.90766&  7.96165&  9.31224& 23.54183& 26.29623& 28.84535& 31.99993& 34.26719\\
 17&  5.69722&  6.40776&  7.56419&  8.67176& 10.08519& 24.76904& 27.58711& 30.19101& 33.40866& 35.71847\\
 18&  6.26480&  7.01491&  8.23075&  9.39046& 10.86494& 25.98942& 28.86930& 31.52638& 34.80531& 37.15645\\
 19&  6.84397&  7.63273&  8.90652& 10.11701& 11.65091& 27.20357& 30.14353& 32.85233& 36.19087& 38.58226\\
 20&  7.43384&  8.26040&  9.59078& 10.85081& 12.44261& 28.41198& 31.41043& 34.16961& 37.56623& 39.99685\\
 21&  8.03365&  8.89720& 10.28290& 11.59131& 13.23960& 29.61509& 32.67057& 35.47888& 38.93217& 41.40106\\
 22&  8.64272&  9.54249& 10.98232& 12.33801& 14.04149& 30.81328& 33.92444& 36.78071& 40.28936& 42.79565\\
 23&  9.26042& 10.19572& 11.68855& 13.09051& 14.84796& 32.00690& 35.17246& 38.07563& 41.63840& 44.18128\\
 24&  9.88623& 10.85636& 12.40115& 13.84843& 15.65868& 33.19624& 36.41503& 39.36408& 42.97982& 45.55851\\
 25& 10.51965& 11.52398& 13.11972& 14.61141& 16.47341& 34.38159& 37.65248& 40.64647& 44.31410& 46.92789\\
 26& 11.16024& 12.19815& 13.84390& 15.37916& 17.29188& 35.56317& 38.88514& 41.92317& 45.64168& 48.28988\\
 27& 11.80759& 12.87850& 14.57338& 16.15140& 18.11390& 36.74122& 40.11327& 43.19451& 46.96294& 49.64492\\
 28& 12.46134& 13.56471& 15.30786& 16.92788& 18.93924& 37.91592& 41.33714& 44.46079& 48.27824& 50.99338\\
 29& 13.12115& 14.25645& 16.04707& 17.70837& 19.76774& 39.08747& 42.55697& 45.72229& 49.58788& 52.33562\\
 30& 13.78672& 14.95346& 16.79077& 18.49266& 20.59923& 40.25602& 43.77297& 46.97924& 50.89218& 53.67196\\
 40& 20.70654& 22.16426& 24.43304& 26.50930& 29.05052& 51.80506& 55.75848& 59.34171& 63.69074& 66.76596\\
 50& 27.99075& 29.70668& 32.35736& 34.76425& 37.68865& 63.16712& 67.50481& 71.42020& 76.15389& 79.48998\\
 60& 35.53449& 37.48485& 40.48175& 43.18796& 46.45889& 74.39701& 79.08194& 83.29767& 88.37942& 91.95170\\
 70& 43.27518& 45.44172& 48.75756& 51.73928& 55.32894& 85.52704& 90.53123& 95.02318&100.42518&104.21490\\
 80& 51.17193& 53.54008& 57.15317& 60.39148& 64.27784& 96.57820&101.87947&106.62857&112.32879&116.32106\\
 90& 59.19630& 61.75408& 65.64662& 69.12603& 73.29109&107.56501&113.14527&118.13589&124.11632&128.29894\\
100& 67.32756& 70.06489& 74.22193& 77.92947& 82.35814&118.49800&124.34211&129.56120&135.80672&140.16949\\
\hline
\end{tabular}}
\end{small}
\end{minipage}
\end{sideways}}}

%\end{landscape}
\addtolength{\textheight}{-5mm}
\addtolength{\textwidth}{-15mm}

\newpage
\section{Table de quantiles de la loi $t$}
\label{chap:t}

Le tableau donne $t_{\nu,\alpha}$, le quantile sup\'erieur de niveau $\alpha$ de la loi de Student avec $\nu$ degr\'es de libert\'e, $\alpha$ est donn\'e dans les colonnes, $\nu$ est donn\'e dans les lignes. Pr\'ecision : Si $T\sim t_{(\nu)}$, alors $\Pr \{T > t_{\nu,\alpha}\} =\alpha$.

\begin{center}
\setlength{\tabcolsep}{10pt}
%\begin{small}{
\begin{tabular}{|c|ccccc|}
\hline
&\multicolumn{5}{|c|}{$\alpha$}\\
\hline
$\quad \nu \quad $&0.100&0.050&0.025&0.01&0.005\\
\hline
 1&  3.078&  6.314& 12.706& 31.821& 63.657\\
 2&  1.886&  2.920&  4.303&  6.965&  9.925\\
 3&  1.638&  2.353&  3.182&  4.541&  5.841\\
 4&  1.533&  2.132&  2.776&  3.747&  4.604\\
 5&  1.476&  2.015&  2.571&  3.365&  4.032\\
 6&  1.440&  1.943&  2.447&  3.143&  3.707\\
 7&  1.415&  1.895&  2.365&  2.998&  3.499\\
 8&  1.397&  1.860&  2.306&  2.896&  3.355\\
 9&  1.383&  1.833&  2.262&  2.821&  3.250\\
10&  1.372&  1.812&  2.228&  2.764&  3.169\\
11&  1.363&  1.796&  2.201&  2.718&  3.106\\
12&  1.356&  1.782&  2.179&  2.681&  3.055\\
13&  1.350&  1.771&  2.160&  2.650&  3.012\\
14&  1.345&  1.761&  2.145&  2.624&  2.977\\
15&  1.341&  1.753&  2.131&  2.602&  2.947\\
16&  1.337&  1.746&  2.120&  2.583&  2.921\\
17&  1.333&  1.740&  2.110&  2.567&  2.898\\
18&  1.330&  1.734&  2.101&  2.552&  2.878\\
19&  1.328&  1.729&  2.093&  2.539&  2.861\\
20&  1.325&  1.725&  2.086&  2.528&  2.845\\
21&  1.323&  1.721&  2.080&  2.518&  2.831\\
22&  1.321&  1.717&  2.074&  2.508&  2.819\\
23&  1.319&  1.714&  2.069&  2.500&  2.807\\
24&  1.318&  1.711&  2.064&  2.492&  2.797\\
25&  1.316&  1.708&  2.060&  2.485&  2.787\\
26&  1.315&  1.706&  2.056&  2.479&  2.779\\
27&  1.314&  1.703&  2.052&  2.473&  2.771\\
28&  1.313&  1.701&  2.048&  2.467&  2.763\\
29&  1.311&  1.699&  2.045&  2.462&  2.756\\
30&  1.310&  1.697&  2.042&  2.457&  2.750\\
40&  1.303&  1.684&  2.021&  2.423&  2.704\\
60&  1.296&  1.671&  2.000&  2.390&  2.660\\
120&  1.289&  1.658&  1.980&  2.358&  2.617\\
\hline
$\infty$ &  1.282&  1.645&  1.960&  2.326&  2.576\\
\hline
\end{tabular}%}
%\end{small}
\end{center}



%%% Local Variables:
%%% mode: latex
%%% TeX-master: "exercices_analyse_statistique"
%%% End:
