\documentclass[letterpaper,10pt]{memoir}\usepackage[]{graphicx}\usepackage[]{color}
% maxwidth is the original width if it is less than linewidth
% otherwise use linewidth (to make sure the graphics do not exceed the margin)
\makeatletter
\def\maxwidth{ %
  \ifdim\Gin@nat@width>\linewidth
    \linewidth
  \else
    \Gin@nat@width
  \fi
}
\makeatother

\definecolor{fgcolor}{rgb}{0.345, 0.345, 0.345}
\newcommand{\hlnum}[1]{\textcolor[rgb]{0.686,0.059,0.569}{#1}}%
\newcommand{\hlstr}[1]{\textcolor[rgb]{0.192,0.494,0.8}{#1}}%
\newcommand{\hlcom}[1]{\textcolor[rgb]{0.678,0.584,0.686}{\textit{#1}}}%
\newcommand{\hlopt}[1]{\textcolor[rgb]{0,0,0}{#1}}%
\newcommand{\hlstd}[1]{\textcolor[rgb]{0.345,0.345,0.345}{#1}}%
\newcommand{\hlkwa}[1]{\textcolor[rgb]{0.161,0.373,0.58}{\textbf{#1}}}%
\newcommand{\hlkwb}[1]{\textcolor[rgb]{0.69,0.353,0.396}{#1}}%
\newcommand{\hlkwc}[1]{\textcolor[rgb]{0.333,0.667,0.333}{#1}}%
\newcommand{\hlkwd}[1]{\textcolor[rgb]{0.737,0.353,0.396}{\textbf{#1}}}%
\let\hlipl\hlkwb

\usepackage{framed}
\makeatletter
\newenvironment{kframe}{%
 \def\at@end@of@kframe{}%
 \ifinner\ifhmode%
  \def\at@end@of@kframe{\end{minipage}}%
  \begin{minipage}{\columnwidth}%
 \fi\fi%
 \def\FrameCommand##1{\hskip\@totalleftmargin \hskip-\fboxsep
 \colorbox{shadecolor}{##1}\hskip-\fboxsep
     % There is no \\@totalrightmargin, so:
     \hskip-\linewidth \hskip-\@totalleftmargin \hskip\columnwidth}%
 \MakeFramed {\advance\hsize-\width
   \@totalleftmargin\z@ \linewidth\hsize
   \@setminipage}}%
 {\par\unskip\endMakeFramed%
 \at@end@of@kframe}
\makeatother

\definecolor{shadecolor}{rgb}{.97, .97, .97}
\definecolor{messagecolor}{rgb}{0, 0, 0}
\definecolor{warningcolor}{rgb}{1, 0, 1}
\definecolor{errorcolor}{rgb}{1, 0, 0}
\newenvironment{knitrout}{}{} % an empty environment to be redefined in TeX

\usepackage{alltt}

  \usepackage[T1]{fontenc}
  \usepackage[utf8]{inputenc}
  \usepackage{natbib,url}
  \usepackage[english,french]{babel}
  \usepackage[autolanguage]{numprint}
  
  \usepackage{lucidabr,pslatex}
  \usepackage[sc]{mathpazo}
  \usepackage{vgmath,vgsets,icomma,amsmath,amsthm,upgreek} 
  \usepackage{graphicx,color}
  \usepackage[absolute]{textpos}
  \usepackage{answers,listings}
  \usepackage[alwaysadjust,defblank]{paralist}
  %\usepackage{verbatim}
  %\usepackage[noae]{Sweave}
  %\usepackage{threeparttable}
  

  %%% Hyperliens
  \usepackage{hyperref}
  \definecolor{link}{rgb}{0,0,0.3}
  \hypersetup{
    pdftex,
    colorlinks,%
    citecolor=link,%
    filecolor=link,%
    linkcolor=link,%
    urlcolor=link}

  %%% Page titre
  \title{\HUGE
    \fontseries{ub}\selectfont Modèles \\
    \fontseries{ub}\selectfont linéaires\\
    \fontseries{ub}\selectfont en actuariat \\[0.5\baselineskip]
    \huge\fontseries{m}\selectfont Exercices et solutions}
  \author{\LARGE Marie-Pier Côté \\[3mm]
          \LARGE Vincent Mercier \\[3mm]
    \large École d'actuariat, Université Laval}
  \date{\large Seconde édition}
  %\newcommand{\ISBN}{978-2-9811416-0-6}
  
  %%% Marge plus large 
	\setlrmarginsandblock{3.5cm}{3cm}{*}
	\setulmarginsandblock{3.5cm}{3cm}{*}
	\checkandfixthelayout
  %%%%%%%%%%%%%%%%%%%%%%%%%%%%%
  
  %% Options de babel
  \frenchbsetup{ThinSpaceInFrenchNumbers=true}
  \addto\captionsfrench{\def\tablename{{\scshape Tab.}}}
  \addto\captionsfrench{\def\figurename{{\scshape Fig.}}}

  %%% Style des entêtes de chapitres
  \chapterstyle{hangnum}

  %%% Styles des entêtes et pieds de page
  \setlength{\marginparsep}{1mm}
  \setlength{\marginparwidth}{1mm}
  \setlength{\headwidth}{\textwidth}
  \addtolength{\headwidth}{\marginparsep}
  \addtolength{\headwidth}{\marginparwidth}
  
  %%% Style de la bibliographie
  %\bibliographystyle{francais}
  \bibliographystyle{plain}
  
    %%% Numéroter les sous-sections
  \maxsecnumdepth{subsection}

%  %%% Nouveaux environnements
%  \theoremstyle{plain}
%  \newtheorem{thm}{Théorème}[chapter]
%  \theoremstyle{definition}
%  \newtheorem{exemple}{Exemple}[chapter]
%  \theoremstyle{remark}
%  \newtheorem*{rem}{Remarque}
%  %%%%%%%%%%%%%%%%%%%%%%%%%%%%%%%%%%%%%%%%%%
  
  %%% Associations entre les environnements et les fichiers
  \Newassociation{sol}{solution}{solutions}
  \Newassociation{rep}{reponse}{reponses}

  %%% Environnement pour les exercices
  \newcounter{exercice}[chapter]
  \newenvironment{exercice}{%
     \begin{list}{\bfseries \arabic{chapter}.\arabic{exercice}}{%
         \refstepcounter{exercice}
         \settowidth{\labelwidth}{\bfseries \arabic{chapter}.\arabic{exercice}}
         \setlength{\leftmargin}{\labelwidth}
         \addtolength{\leftmargin}{\labelsep}
         \setdefaultenum{a)}{i)}{}{}}\item}
     {\end{list}}

  %%% Environnement pour les réponses
  \renewenvironment{reponse}[1]{%
    \begin{list}{\bfseries #1}{%
        \settowidth{\labelwidth}{#1}
        \setlength{\leftmargin}{\labelwidth}
        \addtolength{\leftmargin}{\labelsep}
        \setdefaultenum{a)}{i)}{}{}}\item}
    {\end{list}}
  \renewcommand{\reponseparams}{{\thechapter.\theexercice}}

%  %%% Environnement pour les listes de commandes
%  \newenvironment{ttscript}[1]{%
%    \begin{list}{}{%
%        \setlength{\labelsep}{1.5ex}
%        \settowidth{\labelwidth}{\code{#1}}
%        \setlength{\leftmargin}{\labelwidth}
%        \addtolength{\leftmargin}{\labelsep}
%        \setlength{\parsep}{0.5ex plus0.2ex minus0.2ex}
%        \setlength{\itemsep}{0.3ex}
%        \renewcommand{\makelabel}[1]{##1\hfill}}}
%    {\end{list}}
%  %%%%%%%%%%%%%%%%%%%%%%%%%%%%%%%%%%%%%%%%%%%%%%%%%%%%%%
%  
  %%% Environnement pour les solutions
  \renewenvironment{solution}[1]{%
    \begin{list}{\bfseries #1}{%
        \settowidth{\labelwidth}{#1}
        \setlength{\leftmargin}{\labelwidth}
        \addtolength{\leftmargin}{\labelsep}
        \setdefaultenum{a)}{i)}{}{}}\item}
    {\end{list}}
  \renewcommand{\solutionparams}{{\thechapter.\theexercice}}

  %%% Nouvelles commandes
  \renewcommand{\mat}[1]{\mathbf{#1}}
  \newcommand{\betab}{\boldsymbol{\upbeta}}
  \newcommand{\betabh}{\boldsymbol{\hat{\upbeta}}}
  \newcommand{\vepsb}{\boldsymbol{\upvarepsilon}}
  \newcommand{\SST}{\mathrm{SST}}
  \newcommand{\SSR}{\mathrm{SSR}}
  \newcommand{\SSE}{\mathrm{SSE}}
  \newcommand{\MSE}{\mathrm{MSE}}
  \newcommand{\MSR}{\mathrm{MSR}}
  \newcommand{\tr}{\mathrm{tr}}
  \newcommand{\cov}{\mathbb{C}\text{ov}}
  
  %%% Styles pour les noms de fonctions, code, etc.
  \newcommand{\code}[1]{\texttt{#1}}

  %%% Sous-figures
  \newsubfloat{figure}

%  %%% Paramètres pour les sections de code source
%  \lstloadlanguages{R}
%  \lstdefinelanguage{Renhanced}[]{R}{%
%    morekeywords={acf,ar,arima,arima.sim,colMeans,colSums,is.na,is.null,%
%      mapply,ms,na.rm,nlmin,replicate,row.names,rowMeans,rowSums,seasonal,%
%      sys.time,system.time,ts.plot,which.max,which.min},
%    deletekeywords={c},
%    alsoletter={.\%},%
%    alsoother={:_\$}}
%  \lstset{language=Renhanced,
%    extendedchars=true,
%    inputencoding=latin1,
%    basicstyle=\small\ttfamily,
%    commentstyle=\textsl,
%    keywordstyle=\mdseries,
%    showstringspaces=false,
%    index=[1][keywords],
%    indexstyle=\indexfonction}
%  %%%%%%%%%%%%%%%%%%%%%%%%%%%%%%%%%%%%%%%%%%%%%%%%
  
%  %%% Environnements pour le code S: police plus petite
%  \RecustomVerbatimEnvironment{Sinput}{Verbatim}{fontsize=\small}
%  \RecustomVerbatimEnvironment{Soutput}{Verbatim}{fontsize=\small}
%  \RecustomVerbatimEnvironment{Scode}{Verbatim}{fontsize=\small}

%  %%% Index
%  \renewcommand{\preindexhook}{%
%    Cet index contient des entrées pour les annexes
%    \ref{chap:regression} seulement. Les numéros de
%    page en caractères gras indiquent les pages où les concepts sont
%    introduits, définis ou expliqués.\vskip\onelineskip}
%  \newcommand{\Index}[1]{\index{#1|textbf}}
%  \newcommand{\indexargument}[1]{\index{#1@\code{#1}}}
%  \newcommand{\indexclasse}[1]{\index{#1@\code{#1} (classe)}}
%  \newcommand{\indexfonction}[1]{\index{#1@\code{#1}}}
%  \newcommand{\Indexfonction}[1]{\Index{#1@\code{#1}}}
%
%  \newcommand{\argument}[1]{\code{#1}\indexargument{#1}}
%  \newcommand{\classe}[1]{\code{#1}\indexclasse{#1}}
%  \newcommand{\fonction}[1]{\code{#1}\indexfonction{#1}}
%  \newcommand{\Fonction}[1]{\code{#1}\Indexfonction{#1}}
%  \makeindex
%  %%%%%%%%%%%%%%%%%%%%%%%%%%%%%%%%%%%%%%%%%%%%%%%%%%%%%%%%%
\IfFileExists{upquote.sty}{\usepackage{upquote}}{}
\begin{document}

% \shortcites{R:intro} aucune idée c'est pour quoi

\frontmatter

\pagestyle{empty}
%%% Police de caractère pour la page titre
%\renewcommand{\sfdefault}{hls}

%%% Marge de gauche (1/3 de la page)
\newlength{\gauche}
\addtolength{\gauche}{72mm}
\addtolength{\gauche}{-\spinemargin}

%%% Épaisseur de la bande sur la page titre
\newlength{\ruleheight}
\setlength{\ruleheight}{7.75mm}

%%% Définition de la bande
\definecolor{gray}{gray}{0.4}
\textblockorigin{0mm}{279mm}
\newcommand{\banderecto}{%
  \begin{textblock*}{71.5mm}[0,1](0mm,-46.5mm)
    \textblockcolor{black} \rule{0mm}{\ruleheight}
  \end{textblock*}
  \begin{textblock*}{144mm}[0,1](72mm,-46.5mm)
    \textblockcolor{gray} \rule{0mm}{\ruleheight}
  \end{textblock*}}
\newcommand{\bandeverso}{%
  \begin{textblock*}{144mm}[0,1](0mm,-46.5mm)
    \textblockcolor{gray} \rule{0mm}{\ruleheight}
  \end{textblock*}
  \begin{textblock*}{71.5mm}[0,1](144.5mm,-46.5mm)
    \textblockcolor{black} \rule{0mm}{\ruleheight}
  \end{textblock*}}

%%% Titre
\begin{adjustwidth*}{\gauche}{-15mm}
  \sffamily\fontseries{ub}\selectfont
  \raggedright
  \vspace*{2cm}
  \thetitle
\end{adjustwidth*}

%%% Affichage de la bande
\banderecto
\cleardoublepage

%%% Page de garde
\begin{adjustwidth*}{\gauche}{-15mm}
  \sffamily\fontseries{ub}\selectfont
  \raggedright
  \vspace*{2cm}
  \thetitle \\
  \bfseries
  \vspace*{3cm}
  \theauthor \\
  \vspace*{\fill}
  \thedate
\end{adjustwidth*}

\clearpage

%%% Page de notices
\begingroup
\calccentering{\unitlength}
\begin{adjustwidth*}{\unitlength}{-\unitlength}
  \small
  \setlength{\parindent}{0pt}
  \setlength{\parskip}{\baselineskip}

  {\textcopyright} 2019 Marie-Pier Côté. La partie I du recueil \og~Modèles linéaires en actuariat~: Exercices et solutions~\fg~ est dérivée partiellement de la première partie de la deuxième édition de \og~Modèles de régression et de séries chronologiques~: Exercices et solutions~\fg~ de Vincent Goulet, sous contrat CC BY-SA 2.5. \\
  

  %\raisebox{-1pt} 
  \includegraphics[height=7mm,keepaspectratio=true]{cc-by-sa.jpg} \\  
  Cette création est mise à disposition selon le contrat
  Paternité-Partage des conditions initiales à l'identique 2.5 Canada
  disponible en ligne
  \url{http://creativecommons.org/licenses/by-sa/2.5/ca/} ou par
  courrier postal à Creative Commons, 171 Second Street, Suite 300,
  San Francisco, California 94105, USA.


  \textbf{Historique de publication}
  \vspace{-\baselineskip}
  \begin{tabbing}
    Septembre 2019:\quad\= Première édition 
  \end{tabbing}

  \textbf{Code source} \\
  Le code source {\LaTeX} de la première édition de ce document est disponible en communiquant directement avec les auteurs.

  \vspace{1cm}

\end{adjustwidth*}
\endgroup

%%% Retour à la police normale
\renewcommand{\sfdefault}{phv}

\clearpage

%%% Local Variables:
%%% mode: latex
%%% TeX-master: "exercices_methodes_statistiques"
%%% End:


\pagestyle{companion}

\chapter*{Introduction}
\addcontentsline{toc}{chapter}{Introduction}
\markboth{Introduction}{Introduction}


Ce document contient les exercices proposés par Marie-Pier Côté pour le cours ACT-2003 Modèles linéaires en actuariat, donné à l'École d'actuariat de l'Université Laval. Le recueil a été mis en forme par Vincent Mercier, auxiliaire d'enseignement, à l'aide du soutien financier de la Chaire de leadership en enseignement en analyse de données massives pour l'actuariat --- Intact. 

Plusieurs des exercices de la première partie proviennent d'une ancienne version du recueil, rédigée par Vincent Goulet, professeur à l'École d'actuariat de l'Université Laval. Certains exercices sont le fruit de l'imagination des auteurs ou de ceux des versions précédentes, alors que plusieurs autres sont des adaptations d'exercices tirés des ouvrages cités dans la bibliographie. C'est d'ailleurs afin de ne pas usurper de droits d'auteur que ce document est publié selon les termes du contrat Paternité-Partage des conditions initiales à l’identique 2.5 Canada de Creative Commons. Il s'agit donc d'un document «libre» que quiconque peut réutiliser et modifier à sa guise, à condition que le nouveau document soit publié avec le même contrat.

Le document est séparé en deux parties correspondant aux deux sujets faisant l'objet du cours: d'abord la régression linéaire (simple, multiple et régularisée), puis les modèles linéaires généralisés. L'estimation des paramètres, le calcul de prévisions et l'analyse des résultats sont toutes des procédures à forte composante numérique. Il serait tout à fait artificiel de se restreindre, dans les exercices, à de petits ensembles de données se prêtant au calcul manuel. Dans cette optique, plusieurs des exercices requièrent l'utilisation du logiciel statistique \textsf{R}.

L'annexe \ref{chap:tables} rappelle quelques concepts de statistique de base et contient les tables de la loi khi-carrée et Student, nécessaires pour quelques exercices. L'annexe \ref{chap:multnorm} détaille la notation et les propriétés de la loi normale multivariée. L'annexe \ref{chap:elements} contient quelques résultats d'algèbre matricielle utiles pour résoudre certains exercices.

Les réponses des exercices se trouvent à la fin de chacun des
chapitres, alors que les solutions complètes sont regroupées à
l'annexe~\ref{chap:solutions}.

Tous les jeux de données mentionnés dans ce document sont disponibles en format électronique sur le site de cours ainsi qu'à l'adresse
\url{https://github.com/mpcot24/ACT2003-exercices/tree/master/exercices_modeles_lineaires/data}
  
Ces jeux de données sont importés dans \textsf{R} avec une commande \texttt{scan}, \texttt{read.table} ou \texttt{read.csv}. Certains jeux de données sont également fournis avec \textsf{R}; la commande \\
\texttt{> data()}\\
en fournit une liste complète.

Nous remercions d'avance les lecteurs qui voudront bien nous faire part de toute erreur ou omission dans les exercices ou leurs réponses.



\begin{flushright}
  Marie-Pier Côté \url{<marie-pier.cote@act.ulaval.ca>} \\
  Vincent Mercier \url{<vincent.mercier.7@ulaval.ca>} \\
  Québec, septembre 2019
\end{flushright}

%%% Local Variables:
%%% mode: latex
%%% TeX-master: "exercices_methodes_statistiques"
%%% End:


\cleardoublepage
\tableofcontents*

\mainmatter

\part{Régression linéaire}

\stepcounter{chapter}

%<<child='simple.Rnw'>>=
%@

%<<child='multiple.Rnw'>>=
%@

%<<child='selection.Rnw'>>=
%@


\chapter{Modèles linéaires généralisés (GLM)}
\label{chap:glm}

\Opensolutionfile{reponses}[reponses-glm]
\Opensolutionfile{solutions}[solutions-glm]

\begin{Filesave}{reponses}
\bigskip
\section*{Réponses}

\end{Filesave}

\begin{Filesave}{solutions}
\section*{Chapitre \ref{chap:glm}}
\addcontentsline{toc}{section}{Chapitre \protect\ref{chap:glm}}

\end{Filesave}




\newcommand{\bm}{\boldmath}
\newcommand{\bobeta}{\mbox{\bm$\beta$}}

\begin{exercice}
Est-ce que les distributions suivantes font partie de la famille exponentielle linéaire? Si oui, écrire la densité sous la forme exponentielle linéaire, donner le paramètre canonique, le paramètre de dispersion, l'espérance et la variance de $Y$ en termes de la fonction $b()$ et la relation $V()$ entre la moyenne et la variance.

\begin{enumerate}
\item Normale$(\mu,\sigma^2)$
\item Uniforme$(0,\beta)$
\item Poisson$(\lambda)$
\item Bernoulli$(\pi)$
\item Binomiale$(m, \pi)$, $m>0$ est un entier et est connu (On considère $Y^*=Y/m$).
\item Pareto$(\alpha,\lambda)$
\item Gamma$(\alpha,\beta)$
\item Binomiale négative$(r,\pi)$ avec $r$ connu (On considère $Y^*=Y/r$).
\end{enumerate}

\begin{sol}

\end{sol}
\end{exercice}

\begin{exercice}
Quelles fonctions de lien peut-on utiliser pour un GLM avec une loi de Poisson?
\begin{sol}

\end{sol}
\end{exercice}

\begin{exercice}
Quel est le lien canonique pour la loi gamma? Est-ce que ce lien est toujours approprié?
\begin{sol}

\end{sol}
\end{exercice}

\begin{exercice}
On suppose que $Y_1,...,Y_n$ sont des v.a.s indépendantes et $Y_i\sim Poisson(\mu_i)$. Pour chaque observation, on a une seule variable explicative $x_i$.
\begin{enumerate}
\item Quel est le lien canonique?
\item Trouver les fonctions de score (à résoudre pour l'estimation des paramètres par maximum de vraisemblance) 
\end{enumerate}

\begin{sol}

\end{sol}
\end{exercice}

\begin{exercice}
Montrer que la déviance pour le modèle binomial est $$D(y,\hat{\mu})=2\sum_{i=1}^n m_i \left[y_i\ln\left(\frac{y_i}{\hat{\mu}_i}\right)+(1-y_i)\ln\left(\frac{1-y_i}{1-\hat{\mu}_i}\right)\right].$$

\begin{sol}

\end{sol}
\end{exercice}

\begin{exercice}
Trouver les expressions des résidus de Pearson, d'Anscombe et de déviance pour la loi Gamma.
\begin{sol}

\end{sol}
\end{exercice}

\begin{exercice}
Les données suivantes représentent des données de comptage, du nombre d'échec pour trois appareils médicaux (M1, M2 et M3) lors de tests de résistance sur 1000 appareils de chaque type et pour quatre niveaux de résistance mécanique différents (I, II, III, IV).

\begin{center}
\begin{tabular}{c|cccc}
\texttt{Device}$\backslash$ \texttt{Stress Level} & \texttt{I} & \texttt{II} & \texttt{III} & \texttt{IV}\\ \hline
\texttt{M1} &6 &8& 18 &10\\
\texttt{M2}& 13& 18& 29& 20\\
\texttt{M3} &9& 8 &21& 19\\ 
\end{tabular}
\end{center}

À l'aide de la modélisation Poisson (lien canonique), évaluer s'il y a une différence significative entre les taux d'échec des appareils.
\begin{sol}

\end{sol}
\end{exercice}

\begin{exercice}
Les données pour cet exercice sot contenues dans le fichier \texttt{BritishCar.csv (sep='';'')} disponible sur le site du cours. On y trouve les montants de réclamations moyens pour les dommages causés au véhicule du détenteur de la police pour les véhicules assurés au Royaume-Uni en 1975. Les moyennes sont en livres sterling ajustées pour l'inflation.

\begin{center}
\begin{tabular}{l|l}
\hline
Variable & Description\\ \hline
\texttt{OwnerAge} & Âge du détenteur de la police (8 catégories) \\
\texttt{Model} & Type de voiture (4 groupes) \\
\texttt{CarAge} & Âge du véhicule, en années (4 catégories) \\
\texttt{NClaims} & Nombre de réclamations\\
\texttt{AvCost} & Coût moyen par réclamation, en livres sterling\\ \hline
\end{tabular}
\end{center}

On s'intéresse à la modélisation du coût moyen par réclamation.

\begin{enumerate}
\item Ajuster un modèle de régression Gamma avec lien inverse pour la variable endogène \texttt{AvCost}. Inclure les effets principaux \texttt{OwnerAge}, \texttt{Model} et \texttt{CarAge}.

\item Quelle est l'espérance du coût moyen de la réclamation pour un détenteur de police âgé entre 17 et 20 ans, avec une auto de type A âgée de moins de 3 ans ?

\item Interpréter brièvement les coefficients pour la variable exogène \texttt{OwnerAge}.

\item Interpréter brièvement les coefficients pour la variable exogène \texttt{Model}.

\item Interpréter brièvement les coefficients pour la variable exogène \texttt{CarAge}.

\item Pour quelle combinaison de variables exogènes l'espérance du coût de réclamation est-elle la plus élevée? Calculer sa valeur.

\item Pour quelle combinaison de variables exogènes l'espérance du coût de réclamation est-elle la plus faible? Calculer sa valeur.

\item Quelle est la déviance pour ce modèle? Est-ce que le modèle semble adéquat?

\item Tracer le graphique des résidus de Pearson en fonction des valeurs prédites, des résidus d'Ascombe en fonction des valeurs prédites et des résidus de déviance en fonction des valeurs prédites.

\item Obtient-on les mêmes conclusions aux sous-questions a) à h) si on utilise un lien logarithmique plutôt que le lien inverse?
\end{enumerate}

\begin{sol}

\end{sol}
\end{exercice}

\begin{exercice}
On considère les données suivantes, qui contiennent le nombre $Y_i$ de turbines sur $m_i$ qui ont été fissurées après $x_i$ heures d'opération.

\begin{center}
\begin{tabular}{rrr}
\hline
$x_i$ &$m_i$ & $Y_i$\\\hline
400& 39 &2\\
1000& 53 &4\\
1400 &33& 3\\
1800& 73 &7\\
2200 &30& 5\\
2600& 39 &9\\
3000 &42& 9\\
3400& 13 &6\\
3800 &34& 22\\
4200& 40 &21\\
4600 &36& 21\\\hline
\end{tabular}
\end{center}

\begin{enumerate}
\item En utilisant un GLM binomial avec lien canonique, dériver les estimateurs des paramètres lorsque $x_i$ est traité comme une variable exogène dichotomique avec 11 niveaux, et lorsque le prédicteur linéaire pour la donnée $i$ est $$\eta_i=\beta_0+\beta_i, \mbox{ pour } i=1,...,11,$$ avec la contrainte d'identifiabilité que $\beta_1=0$.

\item En utilisant \textsf{R} et un GLM binomial avec lien canonique, ajuster le modèle où le prédicteur linéaire est $$\eta_i=\beta_0+\beta_1 x_i, \mbox{ pour } i=1,...,11.$$ Donner les estimations des paramètres et leur écart-type.

\item Refaire (b) en utilisant un lien probit. Donner les estimations des paramètres et leur écart-type.

\item Refaire (b) en utilisant un lien log-log complémentaire. Donner les estimations des paramètres et leur écart-type.

\item Comparer les prévisions (et leurs mesures d'incertitude) sous les trois modèles ajustés en (b), (c) et (d) pour une turbine qui était en opération pour 2000 heures.

\item Tracer un graphique pour montrer si les modèles en (b), (c) et (d) ajustent bien (ou non) les données. Commenter.

\end{enumerate}
\begin{sol}

\end{sol}
\end{exercice}

\Closesolutionfile{solutions}
\Closesolutionfile{reponses}

%%%
%%% Insérer les réponses
%%%
\input{reponses-glm}


%%% Local Variables:
%%% mode: latex
%%% TeX-master: "exercices_methodes_statistiques"
%%% End:

%<<child='comptage.Rnw'>>=
%@

\appendix
%\include{regression}
%\include{ts}
%\chapter{Éléments d'algèbre matricielle}
\label{chap:elements}

Cette annexe présente quelques résultats d'algèbre matricielle utiles
en régression linéaire.

\section{Trace}

La \emph{trace} d'une matrice est la somme des éléments de la diagonale.

\begin{thm}
  \label{thm:elements:trace}
  Soient $\mat{A} = [a_{ij}]$ et $\mat{B} = [b_{ij}]$ des matrices
  carrées $k \times k$. Alors
  \begin{enumerate}[a)]
  \item $\tr(\mat{A}) = \sum_{i=1}^k a_{ii}$
  \item $\tr(\mat{A} + \mat{B}) = \tr(\mat{A}) + \tr(\mat{B})$.
  \end{enumerate}
\end{thm}

\begin{thm}
  \label{thm:elements:symetrie_trace}
  Soient les matrices $\mat{A}_{p \times q}$ et $\mat{B}_{q \times
    p}$. Alors $\tr(\mat{AB}) = \tr(\mat{BA})$.
\end{thm}

\enlargethispage{\baselineskip}
\begin{proof}
  Posons $\mat{C} = \mat{AB}$ et $\mat{D} = \mat{BA}$. Par définition
  du produit matriciel, l'élément $c_{ij}$ de la matrice $\mat{C}$ est
  égal au produit scalaire entre la ligne $i$ de $\mat{A}$ et de la
  colonne $j$ de $\mat{B}$, soit
  \begin{displaymath}
    c_{ij} = \sum_{k=1}^q a_{ik} b_{kj}.
  \end{displaymath}
  Les éléments de la diagonale de $\mat{C}$ sont donc $c_{ii} =
  \sum_{j=1}^q a_{ij} b_{ji}$ et, par symétrie, ceux de la diagonale
  de $\mat{D}$ sont $d_{jj} = \sum_{i=1}^p b_{ji} a_{ij}$.  Or,
  \begin{align*}
    \tr(\mat{C})
    &= \sum_{i=1}^p c_{ii} \\
    &= \sum_{i=1}^p \sum_{j=1}^q a_{ij} b_{ji} \\
    &= \sum_{j=1}^q \sum_{i=1}^p b_{ji} a_{ij} \\
    &= \sum_{j=1}^p d_{jj} \\
    &= \tr(\mat{D}).
  \end{align*}
\end{proof}


\section{Formes quadratiques et dérivées}

Soit $\mat{A} = [a_{ij}]$ une matrice $k \times k$ symétrique et
$\mat{x} = (x_1, \dots, x_k)^\prime$ un vecteur. Alors
\begin{displaymath}
  \mat{x^\prime A x} = \sum_{i=1}^k \sum_{j=1}^k a_{ij} x_i x_j
\end{displaymath}
est une forme quadratique.

Par exemple, si
\begin{displaymath}
  \mat{x} =
  \begin{bmatrix}
    x_1 \\
    x_2
  \end{bmatrix}
  \quad
  \text{et}
  \quad
  \mat{A} =
  \begin{bmatrix}
    a_{11} & a_{12} \\
    a_{12} & a_{22}
  \end{bmatrix},
\end{displaymath}
alors
\begin{align*}
  \mat{x^\prime Ax}
  &= \sum_{i=1}^2 \sum_{j=1}^2 a_{ij} x_i x_j \\
  &= a_{11} x_1^2 + 2 a_{12} x_1 x_2 + a_{22} x_2^2.
\end{align*}

\begin{rem}
  Si $\mat{A}$ est diagonale, $\mat{x^\prime Ax} = \sum_{i=1}^k a_{ii} x_i^2$.
\end{rem}

\begin{thm}
  \label{thm:elements:derivee_produit_scalaire}
  Soient $\mat{x} = (x_1, \dots, x_k)^\prime$ et $\mat{a} = (a_1,
  \dots, a_k)^\prime$, d'où $\mat{x^\prime a} = a_1 x_1 + \dots + a_k x_k =
  \sum_{i=1}^k a_i x_i$. Alors
  \begin{align*}
    \frac{d}{d \mat{x}}\, \mat{x^\prime a}
    &= \frac{d}{d \mat{x}} \sum_{i=1}^k a_i x_i \\
    &=
    \begin{bmatrix}
      \frac{d}{d x_1}\, \sum_{i=1}^k a_i x_i \\
      \vdots \\
      \frac{d}{d x_k}\, \sum_{i=1}^k a_i x_i
    \end{bmatrix} \\
    &=
    \begin{bmatrix}
      a_1 \\
      \vdots \\
      a_k
    \end{bmatrix} \\
    &= \mat{a}.
  \end{align*}
\end{thm}

\begin{thm}
  \label{thm:elements:derivee_forme_quadratique}
  Soit $\mat{A}_{k \times k}$ une matrice symétrique. Alors
  \begin{displaymath}
    \frac{d}{d \mat{x}}\, \mat{x^\prime A x} = 2 \mat{A x}.
  \end{displaymath}
\end{thm}

\begin{proof}
  On a
  \begin{align*}
    \mat{x^\prime A x}
    &= \sum_{i=1}^k \sum_{j=1}^k a_{ij} x_i x_j \\
    &= \sum_{i=1}^k a_{ii} x_i^2 +
       \sum_{i=1}^k \sum_{\overset{j=1}{j \neq i}}^k a_{ij} x_i x_j.
  \end{align*}
  Par conséquent, pour $t = 1, \dots, k$ et puisque $a_{ij} = a_{ji}$,
  \begin{align*}
    \frac{\partial}{\partial x_t}\, \mat{x^\prime A x}
    &= 2 a_{tt} x_t + \sum_{\overset{i=1}{i \neq t}}^k a_{it} x_i +
       \sum_{\overset{j=1}{j \neq t}}^k a_{tj} x_j \\
    &= 2 \sum_{i=1}^k a_{it} x_t, \\
    \intertext{d'où}
    \frac{d}{d \mat{x}}\, \mat{x^\prime A x}
    &= 2 \mat{A x}.
  \end{align*}
\end{proof}

\begin{thm}
  \label{thm:elements:derivee_fonction}
  Si $f(\mat{x})$ est une fonction quelconque du vecteur $\mat{x}$, alors
  \begin{displaymath}
    \frac{d}{d \mat{x}}\, f(\mat{x})^\prime \mat{A} f(\mat{x}) =
    2 \left( \frac{d}{d \mat{x}} f(\mat{x}) \right)^\prime \mat{A} f(\mat{x}).
  \end{displaymath}
\end{thm}

Vérifier en exercice les résultats ci-dessus pour une matrice
$\mat{A}$ $3 \times 3$.


\section{Vecteurs et matrices aléatoires}

Soit $X_1, \dots, X_n$ des variables aléatoires. Alors
\begin{displaymath}
  \mat{x} =
  \begin{bmatrix}
    X_1 \\
    \vdots \\
    X_n
  \end{bmatrix}
\end{displaymath}
est un \emph{vecteur aléatoire}. On définit le vecteur espérance
\begin{align*}
  \esp{\mat{x}}
  &=
  \begin{bmatrix}
    \esp{X_1} \\
    \vdots \\
    \esp{X_n}
  \end{bmatrix}
\end{align*}
et la matrice de variance-covariance
\begin{align*}
  \varmat{\mat{x}}
  &= \esp{(\mat{x} - \esp{\mat{x}})(\mat{x} - \esp{\mat{x}})^\prime } \\
  &=
  \begin{bmatrix}
    \var{X_1} & \dots & \Cov(X_1, X_n) \\
    \vdots    & \ddots & \vdots \\
    \Cov(X_n, X_1) & \dots & \var{X_n}
  \end{bmatrix}
\end{align*}

\begin{thm}
  \label{thm:elements:esp_var}
  Soit $\mat{x}$ un vecteur aléatoire et $\mat{A}$ une matrice de
  constantes. Alors
  \begin{enumerate}[a)]
  \item $\esp{\mat{Ax}} = \mat{A} \esp{\mat{x}}$
    \label{thm:elements:esp_var:esp}
  \item $\varmat{\mat{Ax}} = \mat{A} \varmat{\mat{x}} \mat{A}^\prime $.
    \label{thm:elements:esp_var:var}
  \end{enumerate}
\end{thm}

\begin{proof}[Démonstration de b)]
  \begin{align*}
    \varmat{\mat{Ax}}
    &= \esp{(\mat{Ax} - \esp{\mat{Ax}})(\mat{Ax} - \esp{\mat{Ax}})^\prime } \\
    &= \esp{\mat{A}(\mat{x} - \esp{\mat{x}})(\mat{x} - \esp{\mat{x}})^\prime
       \mat{A}^\prime } \\
    &= \mat{A} \varmat{\mat{x}} \mat{A}^\prime .
  \end{align*}
\end{proof}

\begin{exemple}
  Soit $\mat{A} = [1\; 1]$, $\mat{x}^\prime = [X_1\; X_2]$ et $Y =
  \mat{Ax}$, donc $Y = X_1 + X_2$. Alors
  \begin{align*}
    \esp{Y}
    &= \mat{A} \esp{\mat{x}} \\
    &= [1\:\: 1]
    \begin{bmatrix}
    \esp{X_1} \\
    \esp{X_2}
    \end{bmatrix} \\
    &= \esp{X_1} + \esp{X_2}
    \intertext{et}
    \varmat{Y}
    &= \mat{A} \varmat{\mat{x}} \mat{A}^\prime \\
    &= [1\:\: 1]
    \begin{bmatrix}
      \var{X_1} & \Cov(X_1, X_2) \\
      \Cov(X_2, X_1) & \var{X_2}
    \end{bmatrix}
    \begin{bmatrix}
      1 \\
      1
    \end{bmatrix} \\
    &= \var{X_1} + \var{X_2} + 2\, \Cov(X_1, X_2).
  \end{align*}
\end{exemple}

%%% Local Variables:
%%% mode: latex
%%% TeX-master: "exercices_methodes_statistiques"
%%% End:

\chapter{Solutions}
\label{chap:solutions}

\section*{Chapitre \ref{chap:simple}}
\addcontentsline{toc}{section}{Chapitre \protect\ref{chap:simple}}

\begin{solution}{2.1}
    \begin{enumerate}
    \item Voir la figure \ref{fig:simple:base}. Remarquer que l'on
      peut, dans la fonction \texttt{plot}, utiliser une formule pour
      exprimer la relation entre les variables.
      \begin{figure}
        \centering
\begin{knitrout}
\definecolor{shadecolor}{rgb}{0.969, 0.969, 0.969}\color{fgcolor}\begin{kframe}
\begin{alltt}
\hlstd{x}\hlkwb{<-}\hlkwd{c}\hlstd{(}\hlnum{65}\hlstd{,} \hlnum{43}\hlstd{,} \hlnum{44}\hlstd{,} \hlnum{59}\hlstd{,} \hlnum{60}\hlstd{,} \hlnum{50}\hlstd{,} \hlnum{52}\hlstd{,} \hlnum{38}\hlstd{,} \hlnum{42}\hlstd{,} \hlnum{40}\hlstd{)}
\hlstd{y}\hlkwb{<-}\hlkwd{c}\hlstd{(}\hlnum{12}\hlstd{,} \hlnum{32}\hlstd{,} \hlnum{36}\hlstd{,} \hlnum{18}\hlstd{,} \hlnum{17}\hlstd{,} \hlnum{20}\hlstd{,} \hlnum{21}\hlstd{,} \hlnum{40}\hlstd{,} \hlnum{30}\hlstd{,} \hlnum{24}\hlstd{)}
\hlkwd{plot}\hlstd{(y} \hlopt{~} \hlstd{x,} \hlkwc{pch} \hlstd{=} \hlnum{16}\hlstd{)}
\end{alltt}
\end{kframe}
\includegraphics[width=\maxwidth]{figure/unnamed-chunk-4-1}

\end{knitrout}
        \caption{Relation entre les données de l'exercice
          \ref{chap:simple}.\ref{ex:simple:base}}
        \label{fig:simple:base}
      \end{figure}
    \item Les équations normales sont les équations à résoudre pour
      trouver les estimateurs de $\beta_0$ et $\beta_1$ minimisant la
      somme des carrés
      \begin{align*}
        S(\beta_0, \beta_1)
        &=\sum_{t = 1}^n \varepsilon^2_t \\
        &=\sum_{t = 1}^n \left(Y_t-\beta_0-\beta_1X_t\right)^2.
      \end{align*}
      Or,
      \begin{align*}
        \frac{\partial S}{\partial \beta_0}
        &= -2 \sum_{t=1}^n (Y_t - \beta_0 - \beta_1 X_t) \\
        \frac{\partial S}{\partial \beta_1}
        &= -2 \sum_{t=1}^n (Y_t - \beta_0 - \beta_1 X_t) X_t,
      \end{align*}
      d'où les équations normales sont
      \begin{align*}
        \sum_{t=1}^n (Y_t - \hat{\beta}_0 - \hat{\beta}_1 X_t) &= 0 \\
        \sum_{t=1}^n (Y_t - \hat{\beta}_0 - \hat{\beta}_1 X_t) X_t &= 0.
      \end{align*}
    \item Par la première des deux équations normales, on trouve
      \begin{displaymath}
        \sum_{t=1}^nY_t-n\hat{\beta}_0-\hat{\beta}_1\sum_{t=1}^nX_t = 0,
      \end{displaymath}
      soit, en isolant $\hat{\beta}_0$,
      \begin{displaymath}
        \hat{\beta}_0=\frac{\sum_{t=1}^nY_t-\hat{\beta}_1\sum_{t=1}^nX_t}{n}=\bar{Y}-\hat{\beta}_1\bar{X}.
      \end{displaymath}
      De la seconde équation normale, on obtient
      \begin{displaymath}
        \sum_{t=1}^n X_t Y_t -
        \hat{\beta}_0 \sum_{t=1}^n X_t -
        \hat{\beta}_1 \sum_{t=1}^n X_t^2 = 0
      \end{displaymath}
      puis, en remplaçant $\hat{\beta}_0$ par la valeur obtenue ci-dessus,
      \begin{displaymath}
        \hat{\beta}_1
        \left(
          \sum_{t=1}^n X_t^2 - n \bar{X}^2
        \right) =
        \sum_{t=1}^n X_t Y_t - n \bar{X} \bar{Y}.
      \end{displaymath}
      Par conséquent,
      \begin{align*}
        \hat{\beta}_1
        &= \frac{\sum_{t=1}^n X_t Y_t - n \bar{X}\bar{Y}}{\sum_{t=1}^n
          X_t^2 - n \bar{X}^2} \\
        &= \frac{\nombre{11654} - (10)(49,3)(25)}{\nombre{25103} -
          (10)(49,3)^2} \\
        &= -0,8407 \\
        \intertext{et}
        \hat{\beta}_0
        &=\bar{Y}-\hat{\beta}_1\bar{X}\\
        &=25 - (-0,8407)(49,3)\\
        &=66,4488.
      \end{align*}
    \item On peut calculer les prévisions correspondant à $X_1, \dots,
      X_{10}$ --- ou valeurs ajustées --- à partir de la relation
      $\hat{Y}_t = 66,4488 - 0,8407 X_t$, $t = 1, 2, \dots, 10$. Avec
      \textsf{R}, on crée un objet de type modèle de régression avec
      \texttt{lm} et on en extrait les valeurs ajustées avec
      \texttt{fitted}:
\begin{knitrout}
\definecolor{shadecolor}{rgb}{0.969, 0.969, 0.969}\color{fgcolor}\begin{kframe}
\begin{alltt}
\hlstd{fit} \hlkwb{<-} \hlkwd{lm}\hlstd{(y} \hlopt{~} \hlstd{x)}
\hlkwd{fitted}\hlstd{(fit)}
\end{alltt}
\begin{verbatim}
##        1        2        3        4        5        6
## 11.80028 30.29670 29.45596 16.84476 16.00401 24.41148
##        7        8        9       10
## 22.72998 34.50044 31.13745 32.81894
\end{verbatim}
\end{kframe}
\end{knitrout}
      Pour ajouter la droite de régression au graphique de la figure
      \ref{fig:simple:base}, il suffit d'utiliser la fonction
      \texttt{abline} avec en argument l'objet créé avec
      \texttt{lm}. L'ordonnée à l'origine et la pente de la droite
      seront extraites automatiquement. Voir la figure \ref{fig:simple:base2}.
      \begin{figure}
        \centering
\begin{knitrout}
\definecolor{shadecolor}{rgb}{0.969, 0.969, 0.969}\color{fgcolor}\begin{kframe}
\begin{alltt}
\hlkwd{abline}\hlstd{(fit)}
\end{alltt}
\end{kframe}
\end{knitrout}
\begin{knitrout}
\definecolor{shadecolor}{rgb}{0.969, 0.969, 0.969}\color{fgcolor}
\includegraphics[width=\maxwidth]{figure/unnamed-chunk-7-1}

\end{knitrout}
        \caption{Relation entre les données de l'exercice
          \ref{chap:simple}.\ref{ex:simple:base} et la droite de
          régression}
        \label{fig:simple:base2}
      \end{figure}
    \item Les résidus de la régression sont $e_t = Y_t - \hat{Y}_t$,
      $t = 1, \dots, 10$. Dans \textsf{R}, la fonction
      \texttt{residuals} extrait les résidus du modèle:
\begin{knitrout}
\definecolor{shadecolor}{rgb}{0.969, 0.969, 0.969}\color{fgcolor}\begin{kframe}
\begin{alltt}
\hlkwd{residuals}\hlstd{(fit)}
\end{alltt}
\begin{verbatim}
##          1          2          3          4          5
##  0.1997243  1.7032953  6.5440421  1.1552437  0.9959905
##          6          7          8          9         10
## -4.4114773 -1.7299837  5.4995615 -1.1374514 -8.8189450
\end{verbatim}
\end{kframe}
\end{knitrout}
      On vérifie ensuite que la somme des résidus est
      (essentiellement) nulle:
\begin{knitrout}
\definecolor{shadecolor}{rgb}{0.969, 0.969, 0.969}\color{fgcolor}\begin{kframe}
\begin{alltt}
\hlkwd{sum}\hlstd{(}\hlkwd{residuals}\hlstd{(fit))}
\end{alltt}
\begin{verbatim}
## [1] -4.440892e-16
\end{verbatim}
\end{kframe}
\end{knitrout}
    \end{enumerate}
  
\end{solution}
\begin{solution}{2.2}
    \begin{enumerate}
    \item Nous avons le modèle de régression usuel. Les coefficients
      de la régression sont
      \begin{align*}
        \hat{\beta}_1
        &=\frac{\sum_{t=1}^8 X_tY_t-n\bar{X}\bar{Y}}{\sum_{t=1}^8
          X_t^2-n\bar{X}^2} \\
        &=\frac{146-(8)(32/8)(40/8)}{156-(8)(32/8)^2}  \\
        &=-0,5 \\
        \intertext{et}
        \hat{\beta}_0
        &=\bar{Y}-\hat{\beta}_1\bar{X} \\
        &=(40/8)-(-0,5)(32/8) \\
        &=7.
      \end{align*}
    \item Les sommes de carrés sont
      \begin{align*}
        \SST
        &=\sum_{t=1}^8(Y_t-\bar{Y})^2 \\
        &=\sum_{t=1}^8Y_t^2-n\bar{Y}^2 \\
        &=214-(8)(40/8)^2 \\
        &=14, \\
        \SSR
        &=\sum_{t=1}^8(\hat{Y}_t-\bar{Y})^2 \\
        &=\sum_{t=1}^8\hat{\beta}_1^2(X_t-\bar{X})^2 \\
        &=\hat{\beta}_1^2(\sum_{t=1}^8X_t^2-n\bar{X}^2) \\
        &=(-1/2)^2(156-(8)(32/8)^2) \\
        &=7.
      \end{align*}
      et $\SSE = \SST - \SSR = 14 - 7 = 7$. Par conséquent, $R^2 =
      SSR/SST = 7/14 = 0,5$, donc la régression explique 50~\% de la
      variation des $Y_t$ par rapport à leur moyenne $\bar{Y}$. Le
      tableau ANOVA est le suivant:
      \begin{center}
        \begin{tabular}{lcccc}
          \toprule
          Source & SS & d.l. & MS & Ratio F \\
          \midrule
          Régression & 7 & 1 & 7   & 6 \\
          Erreur     & 7 & 6 & 7/6 &  \\
          \midrule
          Total & 14 & 7 & & \\
          \bottomrule
        \end{tabular}
      \end{center}
    \end{enumerate}
  
\end{solution}
\begin{solution}{2.3}
    \begin{enumerate}
    \item Voir la figure \ref{fig:simple:women}.
      \begin{figure}
        \centering
\begin{knitrout}
\definecolor{shadecolor}{rgb}{0.969, 0.969, 0.969}\color{fgcolor}\begin{kframe}
\begin{alltt}
\hlkwd{data}\hlstd{(women)}
\hlkwd{plot}\hlstd{(weight} \hlopt{~} \hlstd{height,} \hlkwc{data} \hlstd{= women,} \hlkwc{pch} \hlstd{=} \hlnum{16}\hlstd{)}
\end{alltt}
\end{kframe}
\includegraphics[width=\maxwidth]{figure/unnamed-chunk-11-1}

\end{knitrout}
        \caption{Relation entre la taille et le poids moyen de femmes américaines âgées de 30 à 39 ans (données \texttt{women})}
        \label{fig:simple:women}
      \end{figure}
    \item Le graphique montre qu'un modèle linéaire serait
      excellent. On estime les paramètres de ce modèle avec \texttt{lm}:
\begin{knitrout}
\definecolor{shadecolor}{rgb}{0.969, 0.969, 0.969}\color{fgcolor}\begin{kframe}
\begin{alltt}
\hlstd{(fit} \hlkwb{<-} \hlkwd{lm}\hlstd{(weight} \hlopt{~} \hlstd{height,} \hlkwc{data} \hlstd{= women))}
\end{alltt}
\begin{verbatim}
##
## Call:
## lm(formula = weight ~ height, data = women)
##
## Coefficients:
## (Intercept)       height
##      -87.52         3.45
\end{verbatim}
\end{kframe}
\end{knitrout}
    \item Voir la figure \ref{fig:simple:women2}.
      \begin{figure}
        \centering
\begin{knitrout}
\definecolor{shadecolor}{rgb}{0.969, 0.969, 0.969}\color{fgcolor}\begin{kframe}
\begin{alltt}
\hlkwd{abline}\hlstd{(fit)}
\end{alltt}
\end{kframe}
\end{knitrout}
\begin{knitrout}
\definecolor{shadecolor}{rgb}{0.969, 0.969, 0.969}\color{fgcolor}
\includegraphics[width=\maxwidth]{figure/unnamed-chunk-14-1}

\end{knitrout}
        \caption{Relation entre les données \texttt{women} et droite de régression linéaire simple}
        \label{fig:simple:women2}
      \end{figure}
      On constate que l'ajustement est excellent.
    \item Le résultat de la fonction \texttt{summary} appliquée au
      modèle \texttt{fit} est le suivant:
\begin{knitrout}
\definecolor{shadecolor}{rgb}{0.969, 0.969, 0.969}\color{fgcolor}\begin{kframe}
\begin{alltt}
\hlkwd{summary}\hlstd{(fit)}
\end{alltt}
\begin{verbatim}
##
## Call:
## lm(formula = weight ~ height, data = women)
##
## Residuals:
##     Min      1Q  Median      3Q     Max
## -1.7333 -1.1333 -0.3833  0.7417  3.1167
##
## Coefficients:
##              Estimate Std. Error t value Pr(>|t|)
## (Intercept) -87.51667    5.93694  -14.74 1.71e-09 ***
## height        3.45000    0.09114   37.85 1.09e-14 ***
## ---
## Signif. codes:
## 0 '***' 0.001 '**' 0.01 '*' 0.05 '.' 0.1 ' ' 1
##
## Residual standard error: 1.525 on 13 degrees of freedom
## Multiple R-squared:  0.991,	Adjusted R-squared:  0.9903
## F-statistic:  1433 on 1 and 13 DF,  p-value: 1.091e-14
\end{verbatim}
\end{kframe}
\end{knitrout}
      Le coefficient de détermination est donc
      $R^2 = 0,991$, %$
      ce qui est près de 1 et confirme donc l'excellent
      ajustement du modèle évoqué en c).
    \item On a
\begin{knitrout}
\definecolor{shadecolor}{rgb}{0.969, 0.969, 0.969}\color{fgcolor}\begin{kframe}
\begin{alltt}
\hlkwd{attach}\hlstd{(women)}
\hlstd{SST} \hlkwb{<-} \hlkwd{sum}\hlstd{((weight} \hlopt{-} \hlkwd{mean}\hlstd{(weight))}\hlopt{^}\hlnum{2}\hlstd{)}
\hlstd{SSR} \hlkwb{<-} \hlkwd{sum}\hlstd{((}\hlkwd{fitted}\hlstd{(fit)} \hlopt{-} \hlkwd{mean}\hlstd{(weight))}\hlopt{^}\hlnum{2}\hlstd{)}
\hlstd{SSE} \hlkwb{<-} \hlkwd{sum}\hlstd{((weight} \hlopt{-} \hlkwd{fitted}\hlstd{(fit))}\hlopt{^}\hlnum{2}\hlstd{)}
\hlkwd{all.equal}\hlstd{(SST, SSR} \hlopt{+} \hlstd{SSE)}
\end{alltt}
\begin{verbatim}
## [1] TRUE
\end{verbatim}
\begin{alltt}
\hlkwd{all.equal}\hlstd{(}\hlkwd{summary}\hlstd{(fit)}\hlopt{$}\hlstd{r.squared, SSR}\hlopt{/}\hlstd{SST)}
\end{alltt}
\begin{verbatim}
## [1] TRUE
\end{verbatim}
\end{kframe}
\end{knitrout}
    \end{enumerate}
  
\end{solution}
\begin{solution}{2.4}
    Puisque $\hat{Y}_t = (\bar{Y} - \hat{\beta}_1 \bar{X}) +
    \hat{\beta}_1 X_t = \bar{Y} + \hat{\beta}_1 (X_t - \bar{X})$ et
    que $e_t = Y_t - \hat{Y}_t = (Y_t - \bar{Y}) - \hat{\beta}_1 (X_t
    - \bar{X})$, alors
    \begin{align*}
      \sum_{t = 1}^n (\hat{Y}_t - \bar{Y}) e_t
      &= \hat{\beta}_1
      \left(
        \sum_{t=1}^n (X_t - \bar{X})(Y_t - \bar{Y}) -
        \hat{\beta}_1 \sum_{t = 1}^n (X_t - \bar{X})^2
      \right) \\
      & = \hat{\beta}_1
      \left(
        S_{XY} - \frac{S_{XY}}{S_{XX}}\, S_{XX}
      \right) \\
      & = 0.
    \end{align*}
  
\end{solution}
\begin{solution}{2.5}
    On a un modèle de régression linéaire simple usuel avec $X_t =
    t$. Les estimateurs des moindres carrés des paramètres $\beta_0$ et
    $\beta_1$ sont donc
    \begin{align*}
      \hat{\beta}_0
      &= \bar{Y} - \hat{\beta}_1\, \frac{\sum_{t = 1}^n t}{n} \\
      \intertext{et}
      \hat{\beta}_1
      &= \frac{\sum_{t = 1}^n t Y_t - \bar{Y} \sum_{t = 1}^n t}{\sum_{t
          = 1}^n t^2 - n^{-1} (\sum_{t = 1}^n t)^2}.
    \end{align*}
    Or, puisque $\sum_{t = 1}^n t = n(n + 1)/2$ et $\sum_{t = 1}^n t^2
    = n(n + 1)(2n + 1)/6$, les expressions ci-dessus se simplifient en
    \begin{align*}
      \hat{\beta}_0
      & = \bar{Y} - \hat{\beta}_1\, \frac{n + 1}{2} \\
      \intertext{et}
      \hat{\beta}_1
      & = \frac{\sum_{t=1}^n t Y_t - n(n + 1) \bar{Y}/2}{n(n + 1)(2n +
        1)/6 - n(n + 1)^2/4} \\
      & = \frac{12 \sum_{t=1}^n t Y_t - 6 n (n + 1) \bar{Y}}{n (n^2 - 1)}.
    \end{align*}
  
\end{solution}
\begin{solution}{2.6}
    \begin{enumerate}
    \item L'estimateur des moindres carrés du paramètre $\beta$ est la
      valeur $\hat{\beta}$ minimisant la somme de carrés
      \begin{align*}
        S(\beta)
        &=\sum_{t = 1}^n \varepsilon_t^2 \\
        &=\sum_{t = 1}^n (Y_t - \beta X_t)^2.
      \end{align*}
      Or,
      \begin{displaymath}
        \frac{d}{d \beta}\, S(\beta) = -2 \sum_{t = 1}^n (Y_t -
        \hat{\beta} X_t) X_t,
      \end{displaymath}
      d'où l'unique équation normale de ce modèle est
      \begin{displaymath}
        \sum_{t = 1}^n X_t Y_t - \hat{\beta} \sum_{t=1}^n X_t^2 = 0.
      \end{displaymath}
      L'estimateur des moindres carrés de $\beta$ est donc
      \begin{displaymath}
        \hat{\beta} = \frac{\sum_{t=1}^n X_t Y_t}{\sum_{t=1}^n X_t^2}.
      \end{displaymath}
    \item On doit démontrer que $\esp{\hat{\beta}} = \beta$. On a
      \begin{align*}
        \esp{\hat{\beta}}
        &= \Esp{\frac{\sum_{t=1}^n X_t Y_t}{\sum_{t=1}^n X_t^2}} \\
        &= \frac{1}{\sum_{t=1}^n X_t^2} \sum_{t=1}^n X_t \esp{Y_t} \\
        &= \frac{1}{\sum_{t=1}^nX_t^2} \sum_{t=1}^n X_t \beta X_t \\
        &= \beta\, \frac{\sum_{t=1}^n X_t^2}{\sum_{t=1}^n X_t^2} \\
        &= \beta.
      \end{align*}
    \item Des hypothèses du modèle, on a
      \begin{align*}
        \var{\hat{\beta}}
        &= \Var{\frac{\sum_{t=1}^n X_t Y_t}{\sum_{t=1}^n X_t^2}} \\
        &= \frac{1}{(\sum_{t=1}^n X_t^2)^2} \sum_{t=1}^n X_t^2 \var{Y_t} \\
        &= \frac{\sigma^2}{(\sum_{t=1}^n X_t^2)^2} \sum_{t=1}^n X_t^2 \\
        &= \frac{\sigma^2}{\sum_{t=1}^n X_t^2}.
      \end{align*}
    \end{enumerate}
  
\end{solution}
\begin{solution}{2.7}
    On veut trouver les coefficients $c_1, \dots, c_n$ tels que
    $\esp{\beta^*} = \beta$ et $\var{\beta^*}$ est minimale. On
    cherche donc à minimiser la fonction
    \begin{align*}
      f(c_1, \dots, c_n)
      &= \var{\beta^*} \\
      &= \sum_{t=1}^n c_t^2 \var{Y_t} \\
      &= \sigma^2 \sum_{t=1}^n c_t^2
    \end{align*}
    sous la contrainte $\esp{\beta^*} = \sum_{t=1}^nc_t\esp{Y_t} =
    \sum_{t=1}^nc_t\beta X_t = \beta\sum_{t=1}^nc_tX_t = \beta$, soit
    $\sum_{t=1}^n c_t X_t = 1$ ou $g(c_1, \dots, c_n) = 0$ avec
    \begin{displaymath}
      g(c_1, \dots, c_n) = \sum_{t=1}^n c_t X_t - 1.
    \end{displaymath}
    Pour utiliser la méthode des multiplicateurs de Lagrange, on pose
    \begin{align*}
      \mathcal{L}(c_1, \dots, c_n,\lambda)
      &= f(c_1, \dots, c_n) - \lambda g(c_1, \dots, c_n), \\
      &= \sigma^2 \sum_{t=1}^n c_t^2 - \lambda
      \left(
        \sum_{t=1}^n c_t X_t - 1
      \right),
    \end{align*}
    puis on dérive la fonction $\mathcal{L}$ par rapport à chacune des
    variables $c_1, \dots, c_n$ et $\lambda$. On trouve alors
    \begin{align*}
      \frac{\partial \mathcal{L}}{\partial c_u}
      & = 2 \sigma^2 c_u - \lambda X_u, \quad u = 1, \dots, n \\
      \frac{\partial \mathcal{L}}{\partial \lambda}
      & = - \sum_{t=1}^n c_t X_t + 1.
    \end{align*}
    En posant les $n$ premières dérivées égales à zéro, on obtient
    \begin{displaymath}
      c_t = \frac{\lambda X_t}{2 \sigma^2}.
    \end{displaymath}
    Or, de la contrainte,
    \begin{displaymath}
      \sum_{t=1}^n c_t X_t =
      \frac{\lambda}{2\sigma^2} \sum_{t=1}^n X_t^2 = 1,
    \end{displaymath}
    d'où
    \begin{displaymath}
      \frac{\lambda}{2 \sigma^2} = \frac{1}{\sum_{t=1}^n X_t^2}
    \end{displaymath}
    et, donc,
    \begin{displaymath}
      c_t = \frac{X_t}{\sum_{t=1}^n X_t^2}.
    \end{displaymath}
    Finalement,
    \begin{align*}
      \beta^*
      & = \sum_{t=1}^n c_t Y_t \\
      & = \frac{\sum_{t=1}^n X_t Y_t}{\sum_{t=1}^n X_t^2} \\
      & = \hat{\beta}.
    \end{align*}
  
\end{solution}
\begin{solution}{2.8}
    \begin{enumerate}
    \item Tout d'abord, puisque $\MSE = \SSE/(n - 2) = \sum_{t=1}^n
      (Y_t - \hat{Y}_t)^2/(n - 2)$ et que $\esp{Y_t} =
      \esp{\hat{Y}_t}$, alors
      \begin{align*}
        \esp{MSE}
        &= \frac{1}{n - 2} \Esp{\sum_{t=1}^n (Y_t - \hat{Y}_t)^2} \\
        &= \frac{1}{n - 2} \sum_{t=1}^n \esp{(Y_t - \hat{Y}_t)^2} \\
        &= \frac{1}{n - 2} \sum_{t=1}^n \esp{((Y_t - \esp{Y_t}) -
          (\hat{Y}_t - \esp{\hat{Y}_t}))^2} \\
        &= \frac{1}{n - 2} \sum_{t=1}^n
        \left(
          \var{Y_t} + \var{\hat{Y}_t} - 2\, \Cov(Y_t, \hat{Y}_t)
        \right).
      \end{align*}
      Or, on a par hypothèse du modèle que $\Cov(Y_t, Y_s) =
      \Cov(\varepsilon_t, \varepsilon_s) = \delta_{ts} \sigma^2$, d'où
      $\var{Y_t} = \sigma^2$ et $\var{\bar{Y}} = \sigma^2/n$. D'autre
      part,
      \begin{align*}
        \var{\hat{Y}_t}
        &= \var{\bar{Y} + \hat{\beta}_1 (X_t - \bar{X})} \\
        &= \var{\bar{Y}} + (X_t - \bar{X})^2 \var{\hat{\beta}_1} +
        2 (X_t - \bar{X}) \Cov(\bar{Y}, \hat{\beta}_1)
      \end{align*}
      et l'on sait que
      \begin{align*}
        \var{\hat{\beta}_1}
        &= \frac{\sigma^2}{\sum_{t=1}^n(X_t-\bar{X})^2} \\
        \intertext{et que}
        \Cov(\bar{Y}, \hat{\beta}_1)
        & = \Cov
        \left(
          \frac{\sum_{t = 1}^n Y_t}{n},
          \frac{\sum_{s = 1}^n (X_s - \bar{X}) Y_s}{\sum_{t = 1}^n
            (X_t - \bar{X})^2}
        \right) \\
        &= \frac{1}{n \sum_{t = 1}^n (X_t - \bar{X})^2}
        \sum_{t = 1}^n \sum_{s = 1}^n \Cov(Y_t, (X_s - \bar{X}) Y_s) \\
        &= \frac{1}{n \sum_{t = 1}^n (X_t - \bar{X})^2}
        \sum_{t = 1}^n (X_s - \bar{X}) \var{Y_t} \\
        & = \frac{\sigma^2}{n \sum_{t = 1}^n (X_t - \bar{X})^2}
        \sum_{t = 1}^n (X_t - \bar{X}) \\
        & = 0,
      \end{align*}
      puisque $\sum_{i=1}^n(X_i - \bar{X}) = 0$. Ainsi,
      \begin{displaymath}
        \var{\hat{Y}_t} = \frac{\sigma^2}{n} +
        \frac{(X_t - \bar{X})^2 \sigma^2}{\sum_{t=1}^n (X_t - \bar{X})^2}.
      \end{displaymath}
      De manière similaire, on détermine que
      \begin{align*}
        \Cov(Y_t, \hat{Y}_t)
        &= \Cov(Y_t, \bar{Y} + \hat{\beta}_1 (X_t - \bar{X})) \\
        &= \Cov(Y_t, \bar{Y}) +
        (X_t - \bar{X}) \Cov(Y_t, \hat{\beta}_1) \\
        &= \frac{\sigma^2}{n} + \frac{(X_t -
          \bar{X})^2 \sigma^2}{\sum_{t=1}^n (X_t - \bar{X})^2}.
      \end{align*}
      Par conséquent,
      \begin{align*}
        \esp{(Y_t - \hat{Y}_t)^2}
        &= \frac{n - 1}{n}\, \sigma^2 -
        \frac{(X_t - \bar{X})^2 \sigma^2}{\sum_{t = 1}^n (X_t - \bar{X})^2} \\
        \intertext{et}
        \sum_{t=1}^n \esp{(Y_t - \hat{Y}_t)^2}
        & = (n - 2) \sigma^2,
      \end{align*}
      d'où $\esp{\MSE} = \sigma^2$.
    \item On a
      \begin{align*}
        \esp{\MSR}
        &= \esp{\SSR} \\
        &= \Esp{\sum_{t=1}^n (\hat{Y}_t - \bar{Y})^2} \\
        &= \sum_{t=1}^n \esp{\hat{\beta}_1^2 (X_t - \bar{X})^2} \\
        &= \sum_{t=1}^n (X_t - \bar{X})^2 \esp{\hat{\beta}_1^2} \\
        &= \sum_{t=1}^n (X_t - \bar{X})^2 (\var{\hat{\beta}_1} +
        \esp{\hat{\beta}_1}^2) \\
        &= \sum_{t=1}^n (X_t - \bar{X})^2
        \left(
          \frac{\sigma^2}{\sum_{t=1}^n (X_t - \bar{X})^2} + \beta_1^2
        \right) \\
        &= \sigma^2 + \beta_1^2 \sum_{t=1}^n (X_t - \bar{X})^2.
      \end{align*}
    \end{enumerate}
  
\end{solution}
\begin{solution}{2.9}
    \begin{enumerate}
    \item Il faut exprimer $\hat{\beta}_0^\prime$ et
      $\hat{\beta}_1^\prime$ en fonction de $\hat{\beta}_0$ et
      $\hat{\beta}_1$. Pour ce faire, on trouve d'abord une expression
      pour chacun des éléments qui entrent dans la définition de
      $\hat{\beta}_1^\prime$. Tout d'abord,
      \begin{align*}
        \bar{X}^\prime
        &= \frac{1}{n} \sum_{t=1}^n X_t^\prime \\
        &= \frac{1}{n} \sum_{t=1}^n (c + d X_t) \\
        &= c + d \bar{X},
      \end{align*}
      et, de manière similaire, $\bar{Y}^\prime = a + b \bar{Y}$. Ensuite,
      \begin{align*}
        S_{XX}^\prime
        &= \sum_{t=1}^n (X_t^\prime - \bar{X}^\prime)^2 \\
        &= \sum_{t=1}^n (c + d X_t - c - d \bar{X})^2 \\
        &= d^2 S_{XX}
      \end{align*}
      et $S_{YY}^\prime = b^2 S_{YY}$, $S_{XY}^\prime = bd S_{XY}$.
      Par conséquent,
      \begin{align*}
        \hat{\beta}_1^\prime
        &= \frac{S_{XY}^\prime}{S_{XX}^\prime} \\
        &= \frac{bd S_{XY}}{d^2 S_{XX}} \\
        &= \frac{b}{d}\, \hat{\beta}_1 \\
        \intertext{et}
        \hat{\beta}_0^\prime
        &= \bar{Y}^\prime - \hat{\beta}_1^\prime \bar{X}^\prime \\
        &= a + b \bar{Y} - \frac{b}{d}\, \hat{\beta}_1 (c + d \bar{X}) \\
        &= a - \frac{bc}{d}\, \hat{\beta}_1 + b (\bar{Y} -
        \hat{\beta}_1 \bar{X}) \\
        &= a - \frac{bc}{d}\, \hat{\beta}_1 + b \hat{\beta}_0.
      \end{align*}
    \item Tout d'abord, on établit que
      \begin{align*}
        R^2
        &= \frac{\SSR}{\SST} \\
        &= \frac{\sum_{t=1}^n (\hat{Y_t} - \bar{Y})^2}{\sum_{t=1}^n
          (Y_t - \bar{Y})^2} \\
        &= \hat{\beta}_1^2\, \frac{\sum_{t=1}^n (X_t -
          \bar{X})^2}{\sum_{t=1}^n (Y_t - \bar{Y})^2} \\
        &= \hat{\beta}_1^2\, \frac{S_{XX}}{S_{YY}}.
      \end{align*}
      Maintenant, avec les résultats obtenus en a), on démontre
      directement que
      \begin{align*}
        (R^2)^\prime
        &= (\hat{\beta}_1^\prime)^2 \frac{S_{XX}^\prime}{S_{YY}^\prime} \\
        &=
        \left(
          \frac{b}{d}
        \right)^2\,
        \hat{\beta}_1^2\, \frac{d^2 S_{XX}}{b^2 S_{YY}} \\
        &= \hat{\beta}_1^2\, \frac{S_{XX}}{S_{YY}} \\
        &= R^2.
      \end{align*}
    \end{enumerate}
  
\end{solution}
\begin{solution}{2.10}
    Considérons un modèle de régression usuel avec l'ensemble de
    données $(X_1, Y_1), \dots, (X_n, Y_n), (m \bar{X}, m \bar{Y})$,
    où $\bar{X} = n^{-1} \sum_{t = 1}^n X_t$, $\bar{Y} = n^{-1}
    \sum_{t = 1}^n Y_t$, $m = n/a$ et $a = \sqrt{n + 1} - 1$. On
    définit
    \begin{align*}
      \bar{X}^\prime
      &= \frac{1}{n + 1} \sum_{t = 1}^{n + 1} X_t \\
      &= \frac{1}{n + 1} \sum_{t = 1}^n X_t + \frac{m}{n + 1} \bar{X} \\
      &= k \bar{X} \\
      \intertext{et, de manière similaire,}
      \bar{Y}^\prime
      &= k \bar{Y},
      \intertext{où}
      k
      &= \frac{n + m}{n + 1} \\
      &= \frac{n (a + 1)}{a (n + 1)}.
    \end{align*}
    L'expression pour l'estimateur des moindres carrés de la pente de
    la droite de régression pour cet ensemble de données est
    \begin{align*}
      \hat{\beta}_1
      &= \frac{\sum_{t = 1}^{n + 1} X_t Y_t - (n + 1)
        \bar{X}^\prime \bar{Y}^\prime}{%
        \sum_{t = 1}^{n + 1} X_t^2 - (n + 1) (\bar{X}^\prime)^2} \\
      &= \frac{\sum_{t = 1}^n X_t Y_t + m^2 \bar{X} \bar{Y} - (n + 1)
        k^2 \bar{X} \bar{Y}}{%
        \sum_{t = 1}^n X_t^2 + m^2 \bar{X}^2 - (n + 1) k^2 \bar{X}^2}.
    \end{align*}
    Or,
    \begin{align*}
      m^2 - k^2 (n + 1)
      &= \frac{n^2}{a^2} - \frac{n^2 (a + 1)^2}{a^2 (n + 1)} \\
      &= \frac{n^2 (n + 1) - n^2 (n + 1)}{a^2 (n + 1)} \\
      &= 0.
    \end{align*}
    Par conséquent,
    \begin{align*}
      \hat{\beta}_1
      &= \frac{\sum_{t = 1}^n X_t Y_t}{\sum_{t = 1}^n X_t^2} \\
      &= \hat{\beta}.
    \end{align*}
    Interprétation: en ajoutant un point bien spécifique à n'importe
    quel ensemble de données, on peut s'assurer que la pente de la
    droite de régression sera la même que celle d'un modèle passant
    par l'origine. Voir la figure \ref{fig:simple:pointmagique} pour
    une illustration du phénomène.

    \begin{figure}
      \centering
\begin{knitrout}
\definecolor{shadecolor}{rgb}{0.969, 0.969, 0.969}\color{fgcolor}
\includegraphics[width=\maxwidth]{figure/unnamed-chunk-18-1}

\end{knitrout}
      \caption{Illustration de l'effet de l'ajout d'un point spécial à
        un ensemble de données. À gauche, la droite de régression
        usuelle. À droite, le même ensemble de points avec le point
        spécial ajouté (cercle plein), la droite de régression avec ce
        nouveau point (ligne pleine) et la droite de régression
        passant par l'origine (ligne pointillée). Les deux droites
        sont parallèles.}
      \label{fig:simple:pointmagique}
    \end{figure}
  
\end{solution}
\begin{solution}{2.11}
    Puisque, selon le modèle, $\varepsilon_t \sim N(0, \sigma^2)$ et
    que $Y_t = \beta_0 + \beta_1 X_t + \varepsilon_t$, alors $Y_t \sim
    N(\beta_0 + \beta_1 X_t, \sigma^2)$. De plus, on sait que
    \begin{align*}
      \hat{\beta}_1
      &= \frac{\sum_{t=1}^n (X_t - \bar{X})(Y_t - \bar{Y})}{%
        \sum_{t=1}^n (X_t - \bar{X})^2} \\
      &= \frac{\sum_{t=1}^n (X_t - \bar{X}) Y_t}{%
        \sum_{t=1}^n (X_t - \bar{X})^2},
    \end{align*}
    donc l'estimateur $\hat{\beta}_1$ est une combinaison linéaire des
    variables aléatoires $Y_1, \dots, Y_n$. Par conséquent,
    $\hat{\beta}_1 \sim N(\esp{\hat{\beta}_1}, \var{\hat{\beta}_1})$,
    où $\esp{\hat{\beta}_1} = \beta_1$ et $\var{\hat{\beta}_1} =
    \sigma^2/S_{XX}$ et, donc,
    \begin{displaymath}
      \Pr
      \left[
        -z_{\alpha/2} <
        \frac{\hat{\beta}_1 - \beta_1}{\sigma/\sqrt{S_{XX}}} <
        z_{\alpha/2}
      \right] = 1 - \alpha.
    \end{displaymath}
    Un intervalle de confiance de niveau $1 - \alpha$ pour le
    paramètre $\beta_1$ lorsque la variance $\sigma^2$ est connue est donc
    \begin{displaymath}
      \beta_1 \in \hat{\beta}_1 \pm z_{\alpha/2}
      \frac{\sigma}{\sqrt{\sum_{t=1}^n (X_t - \bar{X})^2}}.
    \end{displaymath}
  
\end{solution}
\begin{solution}{2.12}
    L'intervalle de confiance pour $\beta_1$ est
    \begin{align*}
      \beta_1
      &\in \hat{\beta}_1 \pm t_{\alpha/2}(n - 2)
      \sqrt{\frac{\hat{\sigma}^2}{S_{XX}}} \\
      &\in \hat{\beta}_1 \pm t_{0,025}(20 - 2) \sqrt{\frac{MSE}{S_{XX}}}.
     \end{align*}
     On nous donne $\SST = S_{YY} = \nombre{20838}$ et $S_{XX} =
     \nombre{10668}$. Par conséquent,
     \begin{align*}
       \SSR
       &= \hat{\beta}_1^2 \sum_{t=1}^{20} (X_t - \bar{X})^2 \\
       &= (-1,104)^2(\nombre{10668}) \\
       &= \nombre{13002,33} \\
       \SSE
       &= \SST - \SSR \\
       &= \nombre{7835,67} \\
       \intertext{et}
       \MSE
       &= \frac{\SSE}{18} \\
       &= 435,315.
     \end{align*}
     De plus, on trouve dans une table de quantiles de la loi de
     Student (ou à l'aide de la fonction \texttt{qt} dans \textsf{R})
     que $t_{0,025}(18) = 2,101$. L'intervalle de confiance recherché
     est donc
     \begin{align*}
       \beta_1
       &\in -1,104 \pm 2,101 \sqrt{\frac{435,315}{\nombre{10668}}} \\
       &\in (-1,528, -0,680).
     \end{align*}
  
\end{solution}
\begin{solution}{2.13}
    \begin{enumerate}
    \item On trouve aisément les estimateurs de la pente et de
      l'ordonnée à l'origine de la droite de régression:
      \begin{align*}
        \hat{\beta}_1
        &= \frac{\sum_{t=1}^n X_t Y_t - n \bar{X}\bar{Y}}{%
          \sum_{t=1}^n X_t^2 - n \bar{X}^2} \\
        &= 1,436 \\
        \hat{\beta}_0
        &= \bar{Y} - \hat{\beta}_1 \bar{X} \\
        &= 9,273.
      \end{align*}
    \item Les sommes de carrés sont
      \begin{align*}
        \SST
        &= \sum_{t=1}^n Y_t^2 - n \bar{Y}^2 \\
        &= 1194 - 11 (9,273)^2 \\
        &= 248,18 \\
        \SSR
        &= \hat{\beta}_1^2
        \left(
          \sum_{t=1}^n X_t^2 - n \bar{X}^2
        \right) \\
        &= (1,436)^2 (110 - 11 (0)) \\
        &= 226,95
      \end{align*}
      et $\SSE = \SST - \SSR = 21,23$. Le tableau d'analyse de
      variance est donc le suivant:

      \begin{center}
        \begin{tabular}{lrrrc}
          \toprule
          Source
          & \multicolumn{1}{c}{SS}
          & \multicolumn{1}{c}{d.l.}
          & \multicolumn{1}{c}{MS}
          & Ratio F \\
          \midrule
          Régression & 226,95 &   1  & 226,95 & 96,21 \\
          Erreur     &  21,23 &   9  &   2,36 &  \\
          \midrule
          Total      & 248,18 &  10  &        & \\
          \bottomrule
        \end{tabular}
      \end{center}

      Or, puisque $t = \sqrt{F} = 9,809 > t_{\alpha/2}(n-2) =
      t_{0,025}(9) = 2,26$, on rejette l'hypothèse $H_0: \beta_1 =
      0$ soit, autrement dit, la pente est significativement
      différente de zéro.
    \item Puisque la variance $\sigma^2$ est inconnue, on l'estime par
      $s^2 = \MSE = 2,36$. On a alors
      \begin{align*}
        \beta_1
        &\in \hat{\beta}_1 \pm t_{\alpha/2}(n-2)
        \sqrt{\widehat{\mathrm{Var}}[\hat{\beta}_1]} \\
        &\in 1,436 \pm 2,26 \sqrt{\frac{2,36}{110}} \\
        &\in (1,105, 1,768).
      \end{align*}
    \item Le coefficient de détermination de la régression est $R^2 =
      \SSR/\SST = 226,95/248,18 = 0,914$, ce qui indique que
      l'ajustement du modèle aux données est très bon. En outre, suite
      au test effectué à la partie b), on conclut que la régression
      est globalement significative.  Toutes ces informations portent
      à conclure qu'il n'y a pas lieu d'utiliser un autre modèle.
    \end{enumerate}
  
\end{solution}
\begin{solution}{2.14}
    On doit déterminer si la régression est significative, ce qui peut
    se faire à l'aide de la statistique $F$. Or, à partir de
    l'information donnée dans l'énoncé, on peut calculer
    \begin{align*}
      \hat{\beta}_1
      &= \frac{\sum_{t=1}^{50} X_t Y_t - 50 \bar{X} \bar{Y}}{%
        \sum_{t=1}^{50} X_t^2 - 50 \bar{X})^2} \\
      &= -0,0110 \\
      \SST
      &= \sum_{t=1}^{50} Y_t^2 - 50 \bar{Y}^2 \\
      &= 78,4098 \\
      \SSR
      &= \hat{\beta}_1^2 \sum_{t=1}^{50} (X_t - \bar{X})^2 \\
      &= 1,1804 \\
      \SSE
      &= \SST - \SSR \\
      &= 77,2294 \\
      \intertext{d'où}
      \MSR
      &= 1,1804 \\
      \MSE
      &= \frac{\SSE}{50 - 2} \\
      &= 1,6089 \\
      \intertext{et, enfin,}
      F
      &= \frac{\MSR}{\MSE} \\
      &= 0,7337.
    \end{align*}
    Soit $F$ une variable aléatoire ayant une distribution de Fisher
    avec 1 et 48 degrés de liberté, soit la même distribution que la
    statistique $F$ sous l'hypothèse $H_0: \beta_1 = 0$. On a que
    $\Pr[F > 0,7337] = 0,3959$, donc la valeur $p$ du test $H_0:
    \beta_1 = 0$ est $0,3959$. Une telle valeur $p$ est généralement
    considérée trop élevée pour rejeter l'hypothèse $H_0$. On ne peut
    donc considérer la relation entre la ligne de vie et l'espérance
    de vie comme significative. (Ou on ne la considère significative
    qu'avec un niveau de confiance de $1 - p = 60,41$~\%.)
  
\end{solution}
\begin{solution}{2.15}
    Premièrement, selon le modèle de régression passant par l'origine,
    $Y_0 = \beta X_0 + \varepsilon_0$ et $\hat{Y}_0 = \hat{\beta}
    X_0$. Considérons, pour la suite, la variable aléatoire $Y_0 -
    \hat{Y}_0$. On voit facilement que $\esp{\hat{\beta}} = \beta$,
    d'où $\esp{Y_0 - \hat{Y}_0} = \esp{\beta X_0 + \varepsilon_0 -
      \hat{\beta} X_0} = \beta X_0 - \beta X_0 = 0$ et
    \begin{displaymath}
      \var{Y_0 - \hat{Y}_0} = \var{Y_0} + \var{\hat{Y}_0} - 2\,
      \Cov(Y_0, \hat{Y}_0).
    \end{displaymath}
    Or, $\Cov(Y_0, \hat{Y}_0) = 0$ par l'hypothèse ii) de l'énoncé,
    $\var{Y_0} = \sigma^2$ et $\var{\hat{Y}_0} = X_0^2\,
    \var{\hat{\beta}}$. De plus,
    \begin{align*}
      \var{\hat{\beta}}
      &= \frac{1}{(\sum_{t=1}^n X_t^2)^2} \sum_{t=1}^n X_t^2\,
      \var{Y_t} \\
      &= \frac{\sigma^2}{\sum_{t=1}^n X_t^2}
    \end{align*}
    d'où, finalement,
    \begin{displaymath}
      \var{Y_0 - \hat{Y}_0} =
      \sigma^2 \left( 1 + \frac{X_0^2}{\sum_{t=1}^n X_t^2} \right).
    \end{displaymath}
    Par l'hypothèse de normalité et puisque $\hat{\beta}$ est une
    combinaison linéaire de variables aléatoires normales,
    \begin{displaymath}
      Y_0 - \hat{Y}_0 \sim N
      \left(
        0, \sigma^2 \left( 1 + \frac{X_0^2}{\sum_{t=1}^n X_t^2} \right)
      \right)
    \end{displaymath}
    ou, de manière équivalente,
    \begin{displaymath}
      \frac{Y_0 - \hat{Y}_0}{\sigma \sqrt{1 + X_0^2/\sum_{t=1}^n X_t^2}}
      \sim N(0, 1).
    \end{displaymath}
    Lorsque la variance $\sigma^2$ est estimée par $s^2$, alors
    \begin{displaymath}
      \frac{Y_0 - \hat{Y}_0}{s \sqrt{1 + X_0^2/\sum_{t=1}^n X_t^2}}
      \sim t(n - 1).
    \end{displaymath}
    La loi de Student a $n - 1$ degrés de liberté puisque le modèle
    passant par l'origine ne compte qu'un seul paramètre. Les bornes
    de l'intervalle de confiance pour la vraie valeur de $Y_0$ sont
    donc
    \begin{displaymath}
      \hat{Y}_0 \pm t_{\alpha/2}(n - 1)\, s\, \sqrt{1 +
        \frac{X_0^2}{\sum_{t=1}^n X_t^2}}.
    \end{displaymath}
  
\end{solution}
\begin{solution}{2.16}
    \begin{enumerate}
    \item Soit $X_1, \dots, X_{10}$ les valeurs de la masse monétaire
      et $Y_1, \dots, Y_{10}$ celles du PNB. On a $\bar{X} = 3,72$,
      $\bar{Y} = 7,55$, $\sum_{t = 1}^{10} X_t^2 = 147,18$, $\sum_{t =
        1}^{10} Y_t^2 = 597,03$ et $\sum_{t = 1}^{10} X_t Y_t =
      295,95$. Par conséquent,
      \begin{align*}
        \hat{\beta}_1
        &= \frac{\sum_{t=1}^{10} X_t Y_t - 10 \bar{X} \bar{Y}}{%
          \sum_{t=1}^{10} X_t^2 - 10 \bar{X}^2} \\
        &= 1,716 \\
        \intertext{et}
        \hat{\beta}_0
        &= \bar{Y} - \hat{\beta}_1 \bar{X} \\
        &= 1,168.
      \end{align*}
      On a donc la relation linéaire $\text{PNB} = 1,168 + 1,716
      \text{ MM}$.
    \item Tout d'abord, on doit calculer l'estimateur $s^2$ de la
      variance car cette quantité entre dans le calcul des intervalles
      de confiance demandés. Pour les calculs à la main, on peut
      éviter de calculer les valeurs de $\hat{Y}_1, \dots,
      \hat{Y}_{10}$ en procédant ainsi:
      \begin{align*}
        \SST
        &= \sum_{t=1}^{10} Y_t^2 - 10 \bar{Y}^2 \\
        &= 27,005 \\
        \SSR
        &= \hat{\beta}_1^2
        \left(
          \sum_{t=1}^{10} X_t^2 - 10 \bar{X}^2
        \right) \\
        &= 25,901,
      \end{align*}
      puis $\SSE = \SST - \SSR = 1,104$ et $s^2 = \MSE = \SSE/(10 - 2)
      = 0,1380$.  On peut maintenant construire les intervalles de
      confiance:
      \begin{align*}
        \beta_0
        &\in \hat{\beta}_0 \pm t_{\alpha/2}(n - 2)\, s\,
        \sqrt{\frac{1}{n} + \frac{\bar{X}^2}{S_{XX}}} \\
        &\in 1,168 \pm (2,306) (0,3715)
        \sqrt{\frac{1}{10} + \frac{3,72^2}{8,796}} \\
        &\in (0,060, 2,276) \\
        \beta_1
        &\in \hat{\beta}_1 \pm t_{\alpha/2}(n - 2)\, s\,
        \sqrt{\frac{1}{S_{XX}}} \\
        &\in 1,716 \pm (2,306) (0,3715) \sqrt{\frac{1}{8,796}} \\
        &\in (1,427, 2,005).
      \end{align*}
      Puisque l'intervalle de confiance pour la pente $\beta_1$ ne
      contient ni la valeur 0, ni la valeur 1, on peut rejeter, avec
      un niveau de confiance de 95~\%, les hypothèses $H_0: \beta_1 =
      0$ et $H_0: \beta_1 = 1$.
    \item Par l'équation obtenue en a) liant le PNB à la masse
      monétaire (MM), un PNB de 12,0 correspond à une masse monétaire
      de
      \begin{align*}
        \text{MM}
        &= \frac{12,0 - 1,168}{1,716} \\
        &= 6,31.
      \end{align*}
    \item On cherche un intervalle de confiance pour la droite de
      régression en $\text{MM}_{1997} = 6,31$ ainsi qu'un intervalle
      de confiance pour la prévision $\text{PNB} = 12,0$ associée à
      cette même valeur de la masse monétaire.  Avec une probabilité
      de $\alpha = 95~\%$, le PNB moyen se trouve dans l'intervalle
      \begin{displaymath}
        12,0 \pm t_{\alpha/2}(n - 2)\, s\,
        \sqrt{\frac{1}{n} + \frac{(6,31 - \bar{X})^2}{S_{XX}}} =
        (11,20, 12,80),
      \end{displaymath}
      alors que la vraie valeur du PNB se trouve dans l'intervalle
      \begin{displaymath}
        12,0 \pm t_{\alpha/2}(n - 2)\, s\,
        \sqrt{1 + \frac{1}{n} + \frac{(6,31 - \bar{X})^2}{S_{XX}}} =
        (10,83, 13,17).
      \end{displaymath}
    \end{enumerate}
  
\end{solution}
\begin{solution}{2.17}
    \begin{enumerate}
    \item Les données du fichier \texttt{house.dat} sont importées
      dans \textsf{R} avec la commande
\begin{knitrout}
\definecolor{shadecolor}{rgb}{0.969, 0.969, 0.969}\color{fgcolor}\begin{kframe}
\begin{alltt}
\hlstd{house} \hlkwb{<-} \hlkwd{read.table}\hlstd{(}\hlstr{"data/house.dat"}\hlstd{,} \hlkwc{header} \hlstd{=} \hlnum{TRUE}\hlstd{)}
\end{alltt}
\end{kframe}
\end{knitrout}
      La figure \ref{fig:simple:house} contient les graphiques de
      \texttt{medv} en fonction de chacune des variables \texttt{rm},
      \texttt{age}, \texttt{lstat} et \texttt{tax}. Le meilleur choix
      de variable explicative pour le prix médian semble être le
      nombre moyen de pièces par immeuble, \texttt{rm}.
      \begin{figure}
        \centering
\begin{knitrout}
\definecolor{shadecolor}{rgb}{0.969, 0.969, 0.969}\color{fgcolor}\begin{kframe}
\begin{alltt}
\hlkwd{par}\hlstd{(}\hlkwc{mfrow} \hlstd{=} \hlkwd{c}\hlstd{(}\hlnum{2}\hlstd{,} \hlnum{2}\hlstd{))}
\hlkwd{plot}\hlstd{(medv} \hlopt{~} \hlstd{rm} \hlopt{+} \hlstd{age} \hlopt{+} \hlstd{lstat} \hlopt{+} \hlstd{tax,} \hlkwc{data} \hlstd{= house,} \hlkwc{ask} \hlstd{=} \hlnum{FALSE}\hlstd{)}
\end{alltt}
\end{kframe}
\includegraphics[width=\maxwidth]{figure/unnamed-chunk-21-1}

\end{knitrout}
        \caption{Relation entre la variable \texttt{medv} et les
          variables \texttt{rm}, \texttt{age}, \texttt{lstat} et
          \texttt{tax} des données \texttt{house.dat}}
        \label{fig:simple:house}
      \end{figure}
    \item Les résultats ci-dessous ont été obtenus avec \textsf{R}.
\begin{knitrout}
\definecolor{shadecolor}{rgb}{0.969, 0.969, 0.969}\color{fgcolor}\begin{kframe}
\begin{alltt}
\hlstd{fit1} \hlkwb{<-} \hlkwd{lm}\hlstd{(medv} \hlopt{~} \hlstd{rm,} \hlkwc{data} \hlstd{= house)}
\hlkwd{summary}\hlstd{(fit1)}
\end{alltt}
\begin{verbatim}
##
## Call:
## lm(formula = medv ~ rm, data = house)
##
## Residuals:
##     Min      1Q  Median      3Q     Max
## -23.346  -2.547   0.090   2.986  39.433
##
## Coefficients:
##             Estimate Std. Error t value Pr(>|t|)
## (Intercept)  -34.671      2.650  -13.08   <2e-16 ***
## rm             9.102      0.419   21.72   <2e-16 ***
## ---
## Signif. codes:
## 0 '***' 0.001 '**' 0.01 '*' 0.05 '.' 0.1 ' ' 1
##
## Residual standard error: 6.616 on 504 degrees of freedom
## Multiple R-squared:  0.4835,	Adjusted R-squared:  0.4825
## F-statistic: 471.8 on 1 and 504 DF,  p-value: < 2.2e-16
\end{verbatim}
\end{kframe}
\end{knitrout}
      On peut voir que tant l'ordonnée à l'origine que la pente sont
      très significativement différentes de zéro. La régression est
      donc elle-même significative. Cependant, le coefficient de
      détermination n'est que de $R^2 =
      0,4835$, %$
      ce qui indique que d'autres facteurs pourraient expliquer la
      variation dans \texttt{medv}.

      On calcule les bornes de l'intervalle de confiance de la
      régression avec la fonction \texttt{predict}:
\begin{knitrout}
\definecolor{shadecolor}{rgb}{0.969, 0.969, 0.969}\color{fgcolor}\begin{kframe}
\begin{alltt}
\hlstd{pred.ci} \hlkwb{<-} \hlkwd{predict}\hlstd{(fit1,} \hlkwc{interval} \hlstd{=} \hlstr{"confidence"}\hlstd{,} \hlkwc{level} \hlstd{=} \hlnum{0.95}\hlstd{)}
\end{alltt}
\end{kframe}
\end{knitrout}
      La droite de régression et ses bornes d'intervalle de confiance
      inférieure et supérieure sont illustrée à la figure
      \ref{fig:simple:house2}.
      \begin{figure}
        \centering
\begin{knitrout}
\definecolor{shadecolor}{rgb}{0.969, 0.969, 0.969}\color{fgcolor}\begin{kframe}
\begin{alltt}
\hlstd{ord} \hlkwb{<-} \hlkwd{order}\hlstd{(house}\hlopt{$}\hlstd{rm)}
\hlkwd{plot}\hlstd{(medv} \hlopt{~} \hlstd{rm,} \hlkwc{data} \hlstd{= house,} \hlkwc{ylim} \hlstd{=} \hlkwd{range}\hlstd{(pred.ci))}
\hlkwd{matplot}\hlstd{(house}\hlopt{$}\hlstd{rm[ord], pred.ci[ord,],}
        \hlkwc{type} \hlstd{=} \hlstr{"l"}\hlstd{,} \hlkwc{lty} \hlstd{=} \hlkwd{c}\hlstd{(}\hlnum{1}\hlstd{,} \hlnum{2}\hlstd{,} \hlnum{2}\hlstd{),} \hlkwc{lwd}\hlstd{=} \hlnum{2}\hlstd{,}
        \hlkwc{col} \hlstd{=} \hlstr{"black"}\hlstd{,} \hlkwc{add} \hlstd{=} \hlnum{TRUE}\hlstd{)}
\end{alltt}
\end{kframe}
\includegraphics[width=\maxwidth]{figure/unnamed-chunk-24-1}

\end{knitrout}
        \caption{Résultat de la régression de la variable \texttt{rm} sur la variable \texttt{medv} des données \texttt{house.dat}}
        \label{fig:simple:house2}
      \end{figure}
    \item On reprend la même démarche, mais cette fois avec la
      variable \texttt{age}:
\begin{knitrout}
\definecolor{shadecolor}{rgb}{0.969, 0.969, 0.969}\color{fgcolor}\begin{kframe}
\begin{alltt}
\hlstd{fit2} \hlkwb{<-} \hlkwd{lm}\hlstd{(medv} \hlopt{~} \hlstd{age,} \hlkwc{data} \hlstd{= house)}
\hlkwd{summary}\hlstd{(fit2)}
\end{alltt}
\begin{verbatim}
##
## Call:
## lm(formula = medv ~ age, data = house)
##
## Residuals:
##     Min      1Q  Median      3Q     Max
## -15.097  -5.138  -1.958   2.397  31.338
##
## Coefficients:
##             Estimate Std. Error t value Pr(>|t|)
## (Intercept) 30.97868    0.99911  31.006   <2e-16 ***
## age         -0.12316    0.01348  -9.137   <2e-16 ***
## ---
## Signif. codes:
## 0 '***' 0.001 '**' 0.01 '*' 0.05 '.' 0.1 ' ' 1
##
## Residual standard error: 8.527 on 504 degrees of freedom
## Multiple R-squared:  0.1421,	Adjusted R-squared:  0.1404
## F-statistic: 83.48 on 1 and 504 DF,  p-value: < 2.2e-16
\end{verbatim}
\begin{alltt}
\hlstd{pred.ci} \hlkwb{<-} \hlkwd{predict}\hlstd{(fit2,} \hlkwc{interval} \hlstd{=} \hlstr{"confidence"}\hlstd{,} \hlkwc{level} \hlstd{=} \hlnum{0.95}\hlstd{)}
\end{alltt}
\end{kframe}
\end{knitrout}
      La régression est encore une fois très significative. Cependant,
      le $R^2$ est encore plus faible qu'avec la variable
      \texttt{rm}. Les variables \texttt{rm} et \texttt{age}
      contribuent donc chacune à expliquer les variations de la
      variable \texttt{medv} (et \texttt{rm} mieux que \texttt{age}),
      mais aucune ne sait le faire seule de manière satisfaisante. La
      droite de régression et l'intervalle de confiance de celle-ci
      sont reproduits à la figure \ref{fig:simple:house3}. On constate
      que l'intervalle de confiance est plus large qu'en b).
      \begin{figure}
        \centering
\begin{knitrout}
\definecolor{shadecolor}{rgb}{0.969, 0.969, 0.969}\color{fgcolor}\begin{kframe}
\begin{alltt}
\hlstd{ord} \hlkwb{<-} \hlkwd{order}\hlstd{(house}\hlopt{$}\hlstd{age)}
\hlkwd{plot}\hlstd{(medv} \hlopt{~} \hlstd{age,} \hlkwc{data} \hlstd{= house,} \hlkwc{ylim} \hlstd{=} \hlkwd{range}\hlstd{(pred.ci))}
\hlkwd{matplot}\hlstd{(house}\hlopt{$}\hlstd{age[ord], pred.ci[ord,],}
        \hlkwc{type} \hlstd{=} \hlstr{"l"}\hlstd{,} \hlkwc{lty} \hlstd{=} \hlkwd{c}\hlstd{(}\hlnum{1}\hlstd{,} \hlnum{2}\hlstd{,} \hlnum{2}\hlstd{),} \hlkwc{lwd} \hlstd{=} \hlnum{2}\hlstd{,}
        \hlkwc{col} \hlstd{=} \hlstr{"black"}\hlstd{,} \hlkwc{add} \hlstd{=} \hlnum{TRUE}\hlstd{)}
\end{alltt}
\end{kframe}
\includegraphics[width=\maxwidth]{figure/unnamed-chunk-26-1}

\end{knitrout}
        \caption{Résultat de la régression de la variable \texttt{age} sur la variable \texttt{medv} des données \texttt{house.dat}}
        \label{fig:simple:house3}
      \end{figure}
    \end{enumerate}
  
\end{solution}
\begin{solution}{2.18}
    \begin{enumerate}
    \item On importe les données dans \textsf{R}, puis on effectue les
      conversions demandées. La variable \texttt{consommation}
      contient la consommation des voitures en $\ell$/100~km et la
      variable \texttt{poids} le poids en kilogrammes.
\begin{knitrout}
\definecolor{shadecolor}{rgb}{0.969, 0.969, 0.969}\color{fgcolor}\begin{kframe}
\begin{alltt}
\hlstd{carburant} \hlkwb{<-} \hlkwd{read.table}\hlstd{(}\hlstr{"carburant.dat"}\hlstd{,} \hlkwc{header} \hlstd{=} \hlnum{TRUE}\hlstd{)}
\hlstd{consommation} \hlkwb{<-} \hlnum{235.1954}\hlopt{/}\hlstd{carburant}\hlopt{$}\hlstd{mpg}
\hlstd{poids} \hlkwb{<-} \hlstd{carburant}\hlopt{$}\hlstd{poids} \hlopt{*} \hlnum{0.45455} \hlopt{*} \hlnum{1000}
\end{alltt}
\end{kframe}
\end{knitrout}
    \item La fonction \texttt{summary} fournit l'information
      essentielle pour juger de la validité et de la qualité du
      modèle:
\begin{knitrout}
\definecolor{shadecolor}{rgb}{0.969, 0.969, 0.969}\color{fgcolor}\begin{kframe}
\begin{alltt}
\hlstd{fit} \hlkwb{<-} \hlkwd{lm}\hlstd{(consommation} \hlopt{~} \hlstd{poids)}
\hlkwd{summary}\hlstd{(fit)}
\end{alltt}
\begin{verbatim}
##
## Call:
## lm(formula = consommation ~ poids)
##
## Residuals:
##      Min       1Q   Median       3Q      Max
## -2.07123 -0.68380  0.01488  0.44802  2.66234
##
## Coefficients:
##               Estimate Std. Error t value Pr(>|t|)
## (Intercept) -0.0146530  0.7118445  -0.021    0.984
## poids        0.0078382  0.0005315  14.748   <2e-16 ***
## ---
## Signif. codes:
## 0 '***' 0.001 '**' 0.01 '*' 0.05 '.' 0.1 ' ' 1
##
## Residual standard error: 1.039 on 36 degrees of freedom
## Multiple R-squared:  0.858,	Adjusted R-squared:  0.854
## F-statistic: 217.5 on 1 and 36 DF,  p-value: < 2.2e-16
\end{verbatim}
\end{kframe}
\end{knitrout}
      Le modèle est donc le suivant: $Y_t =
      -0,01465 +
      0,007838 X_t +
      \varepsilon_t$, $\varepsilon_t \sim N(0,
      1,039^2)$, où $Y_t$ est la
      consommation en litres aux 100 kilomètres et $X_t$ le poids en
      kilogrammes. La faible valeur $p$ du test $F$ indique une
      régression très significative. De plus, le $R^2$ de
      0,858 %$
      confirme que l'ajustement du modèle est assez bon.
    \item On veut calculer un intervalle de confiance pour la
      consommation en carburant prévue d'une voiture de
      \nombre{1350}~kg. On obtient, avec la fonction \texttt{predict}:
\begin{knitrout}
\definecolor{shadecolor}{rgb}{0.969, 0.969, 0.969}\color{fgcolor}\begin{kframe}
\begin{alltt}
\hlkwd{predict}\hlstd{(fit,} \hlkwc{newdata} \hlstd{=} \hlkwd{data.frame}\hlstd{(}\hlkwc{poids} \hlstd{=} \hlnum{1350}\hlstd{),} \hlkwc{interval} \hlstd{=} \hlstr{"prediction"}\hlstd{)}
\end{alltt}
\begin{verbatim}
##       fit      lwr     upr
## 1 10.5669 8.432089 12.7017
\end{verbatim}
\end{kframe}
\end{knitrout}
    \end{enumerate}
  
\end{solution}
\begin{solution}{2.19}
\begin{enumerate}
\item On a $$\bar{Y}=\frac{\sum_{i=1}^{500}Y_i}{500}=\frac{300\bar{Y}_F+200\bar{Y}_H}{500}.$$ Aussi,
$$\hat{\beta}_1=\frac{S_{xy}}{S_{xx}}=\frac{\sum_{i=1}^{500}x_iY_i-500\bar{x}\bar{Y}}{\sum_{i=1}^{500}x_i^2-500\bar{x}^2}.$$  Or,
\begin{align*}
\bar{x}&=\frac{\sum_{i=1}^{500}x_i}{500}=\frac{300}{500},\\
\sum_{i=1}^{500}x_i^2&=300,\\
\sum_{i=1}^{500}x_i Y_i&=300\bar{Y}_F
\end{align*}
Donc,
\begin{align*}
\hat{\beta}_1&=\frac{300\bar{Y}_F-500\times\frac{300}{500}\times\frac{300\bar{Y}_F+200\bar{Y}_H}{500}}{300-500\left(\frac{300}{500}\right)^2}\\
&=\frac{500\bar{Y}_F-300\bar{Y}_F-200\bar{Y}_H}{500-300}\\
&=\bar{Y}_F-\bar{Y}_H.
\end{align*}

\item Oui, le coefficient relié à la variable indicatrice qui vaut 1 si le sexe est F représente la différence etre la moyenne de l'espérance de vie pour les femmes et la moyenne de l'espérace de vie pour les hommes.

\item
\begin{align*}
\hat{\beta}_0&=\bar{Y}-\hat{\beta}_1\bar{x}=\bar{Y}-(\bar{Y}_F-\bar{Y}_H)\frac{300}{500}=\bar{Y}_H.
\end{align*}
$\Rightarrow \hat{\beta}_0$ est la moyenne de l'espérance de vie pour les hommes.

\end{enumerate}
\end{solution}
\begin{solution}{2.20}
\begin{enumerate}
\item
\begin{align*}
\cov(Y_i,\hat{Y}_j)&=\cov(Y_i,\hat{\beta}_0+\hat{\beta}_1x_j)\\
&=\cov(Y_i,\bar{Y}-\hat{\beta}_1\bar{x}+\hat{\beta}_1x_j)\\
&=\cov(Y_i,\bar{Y})+(x_j-\bar{x})\cov(Y_i,\hat{\beta}_1) \mbox{ par indépendance des observations}\\
&=\frac{\sigma^2}{n}+\frac{(x_j-\bar{x})}{S_{xx}}\sum_{l=1}^n(x_l-\bar{x})\cov(Y_i,Y_l)\\
&=\frac{\sigma^2}{n}+\frac{(x_j-\bar{x})(x_i-\bar{x})}{S_{xx}}\sigma^2\mbox{ par indépendance des observations}.\\
\end{align*}

\item
\begin{align*}
\cov(\hat{Y}_i,\hat{Y}_j)&=\cov(\hat{\beta}_0+\hat{\beta}_1x_i,\hat{\beta}_0+\hat{\beta}_1x_j)\\
&=\var(\hat{\beta}_0)+(x_i+x_j)\cov(\hat{\beta}_0,\hat{\beta}_1)+x_ix_j\var(\hat{\beta}_1)\\
&=\sigma^2\left(\frac{1}{n}+\frac{\bar{x}^2}{S_{xx}}\right)-(x_i+x_j)\frac{\bar{x}\sigma^2}{S_{xx}}+x_ix_j\frac{\sigma^2}{S_{xx}}\\
&=...\\
&=\sigma^2\left(\frac{1}{n}+\frac{(x_j-\bar{x})(x_i-\bar{x})}{S_{xx}}\right).
\end{align*}

\item
\begin{align*}
\cov(\hat{\varepsilon}_i,\hat{\varepsilon}_j)&=\cov(Y_i-\hat{Y}_i,Y_j-\hat{Y}_j)\\
&=\cov(Y_i,Y_j)-\cov(Y_i,\hat{Y}_j)-\cov(\hat{Y}_i,Y_j)+\cov(\hat{Y}_i,\hat{Y}_j)\\
&=0-2\sigma^2\left(\frac{1}{n}+\frac{(x_j-\bar{x})(x_i-\bar{x})}{S_{xx}}\right)+\left(\frac{1}{n}+\frac{(x_j-\bar{x})(x_i-\bar{x})}{S_{xx}}\right)\\
&=-\sigma^2\left(\frac{1}{n}+\frac{(x_i-\bar{x})(x_j-\bar{x})}{S_{xx}}\right).
\end{align*}
\end{enumerate}
\end{solution}
\begin{solution}{2.21}
Utiliser l'approximation de Taylor de premier ordre pour montrer que la variance de $g(Y)=1/Y$ est approximativement constante.
\end{solution}
\begin{solution}{2.22}
\begin{enumerate}
\item Figure~\ref{fig:simple:bact1} shows a scatter plot of the number of bacteria versus the minutes of exposure. The plot shows a straight line would be a reasonable model, but an even better model would capture the curvature. In fact, the plot shows that when the canned food is exposed to $300^{o}$ F for a long time, there is ultimately no bacteria left. This suggests a model that would capture the asymptotic behavior of the number of bacteria when the number of minutes of exposure increases. A linear model would continue to drive down the number of bacteria, eventually leading to negative values, which is nonsensical in this context.

\begin{figure}
\centering
\begin{knitrout}
\definecolor{shadecolor}{rgb}{0.969, 0.969, 0.969}\color{fgcolor}
\includegraphics[width=\maxwidth]{figure/unnamed-chunk-32-1}

\end{knitrout}
\caption{Scatter Plot of the Number of Bacteria versus the Minutes of Exposure to $300^{o}$ F}
\label{fig:simple:bact1}
\end{figure}

\item A simple linear model is fitted to the data using \textsf{R}. Here is a summary of the model:
\begin{knitrout}
\definecolor{shadecolor}{rgb}{0.969, 0.969, 0.969}\color{fgcolor}\begin{kframe}
\begin{alltt}
\hlstd{fit1} \hlkwb{<-} \hlkwd{lm}\hlstd{(bact}\hlopt{~}\hlstd{min)}
\hlkwd{summary}\hlstd{(fit1)}
\end{alltt}
\begin{verbatim}
##
## Call:
## lm(formula = bact ~ min)
##
## Residuals:
##     Min      1Q  Median      3Q     Max
## -17.323  -9.890  -7.323   2.463  45.282
##
## Coefficients:
##             Estimate Std. Error t value Pr(>|t|)
## (Intercept)   142.20      11.26  12.627 1.81e-07 ***
## min           -12.48       1.53  -8.155 9.94e-06 ***
## ---
## Signif. codes:
## 0 '***' 0.001 '**' 0.01 '*' 0.05 '.' 0.1 ' ' 1
##
## Residual standard error: 18.3 on 10 degrees of freedom
## Multiple R-squared:  0.8693,	Adjusted R-squared:  0.8562
## F-statistic: 66.51 on 1 and 10 DF,  p-value: 9.944e-06
\end{verbatim}
\end{kframe}
\end{knitrout}
The fitted model is $$\hat{y}=142.20-12.48x,$$ where the parameters of the model are estimated by the best linear unbiased estimators. The ANOVA table is obtained using \textsf{R}:
\begin{knitrout}
\definecolor{shadecolor}{rgb}{0.969, 0.969, 0.969}\color{fgcolor}\begin{kframe}
\begin{alltt}
\hlkwd{anova}\hlstd{(fit1)}
\end{alltt}
\begin{verbatim}
## Analysis of Variance Table
##
## Response: bact
##           Df  Sum Sq Mean Sq F value    Pr(>F)
## min        1 22268.8 22268.8  66.512 9.944e-06 ***
## Residuals 10  3348.1   334.8
## ---
## Signif. codes:
## 0 '***' 0.001 '**' 0.01 '*' 0.05 '.' 0.1 ' ' 1
\end{verbatim}
\end{kframe}
\end{knitrout}

In order to test for the significance of regression, we use the F-statistic. The F-statistic is 66.512, and it has 1 and 10 degrees of freedom, so the $p$-value is $$P[F_{(1,10)}>66.512]=9.944\times10^{-6} .$$ Since the $p$-value is much smaller than 1\%, there is enough evidence to reject the null hypothesis that $\beta_{1}=0$ at the 1\% level. The simple linear model is significant.

The value of $R^{2}$ is $86.93\%$. This is a high coefficient of correlation, it means that about 87\% of the variation in the number of bacteria in the canned food is explained by the minutes of exposure to $300^{o}$F. The model seems to perform well.

The Q-Q Plot of the studentized residuals is shown in Figure~\ref{qqplot3b}. The line represents when the empirical quantiles are exactly equal to the standard normal quantiles. The normality assumption is seriously violated as the dots are clearly not on a straight line. This means there are serious flaws in the model, including the fact that the hypothesis tests are not reliable.

\begin{figure}
\begin{center}
\begin{knitrout}
\definecolor{shadecolor}{rgb}{0.969, 0.969, 0.969}\color{fgcolor}
\includegraphics[width=\maxwidth]{figure/unnamed-chunk-35-1}

\end{knitrout}
\end{center}
\caption{Q-Q Plot for Simple Linear Model in Problem 5 b)}
\label{fig:simple:bact2}
\end{figure}

\begin{figure}
\begin{center}
\begin{knitrout}
\definecolor{shadecolor}{rgb}{0.969, 0.969, 0.969}\color{fgcolor}
\includegraphics[width=\maxwidth]{figure/unnamed-chunk-36-1}

\end{knitrout}
\end{center}
\caption{Residuals versus the Fitted Values for Simple Linear Model in Problem 5 b)} \label{fig:simple:bact3}
\end{figure}

Figure~\ref{fig:simple:bact3} shows a plot of the studentized residuals versus the fitted values. The plot suggests a clear curve, which is usually an indicator of non-linearity. This is in line with the previous comments.

Finally, this model is inadequate and transformations on the response variables are required.

\item The Box-Cox method is used to determine which transformation is optimal. Figure~\ref{fig:simple:bact4} shows the plot of the log-likelihood function in terms of $\lambda$, for two different ranges of $\lambda$. It was obtained with the \textsf{R} commands:
\begin{knitrout}
\definecolor{shadecolor}{rgb}{0.969, 0.969, 0.969}\color{fgcolor}\begin{kframe}
\begin{alltt}
\hlkwd{boxCox}\hlstd{(bact}\hlopt{~}\hlstd{min,} \hlkwc{lambda} \hlstd{=} \hlkwd{seq}\hlstd{(}\hlopt{-}\hlnum{2}\hlstd{,} \hlnum{2}\hlstd{,} \hlkwc{len} \hlstd{=} \hlnum{20}\hlstd{),} \hlkwc{plotit} \hlstd{=} \hlnum{TRUE}\hlstd{)}
\hlkwd{boxCox}\hlstd{(bact}\hlopt{~}\hlstd{min,} \hlkwc{lambda} \hlstd{=} \hlkwd{seq}\hlstd{(}\hlopt{-}\hlnum{0.2}\hlstd{,} \hlnum{0.5}\hlstd{,} \hlkwc{len} \hlstd{=} \hlnum{20}\hlstd{),} \hlkwc{plotit} \hlstd{=} \hlnum{TRUE}\hlstd{)}
\end{alltt}
\end{kframe}
\end{knitrout}

\begin{figure}
\begin{center}
\begin{knitrout}
\definecolor{shadecolor}{rgb}{0.969, 0.969, 0.969}\color{fgcolor}
\includegraphics[width=\maxwidth]{figure/unnamed-chunk-38-1}

\end{knitrout}
\end{center}
\caption{Log-likelihood versus $\lambda$ in the Box-Cox method for Problem 5 c)} \label{fig:simple:bact4}
\end{figure}

Note that the maximum is around 0.1 and 0 is included in the 95\% confidence interval for $\lambda$. Therefore, it is preferable to use 0 as this is a common transformation, it represents the logarithm transformation. Let $y^{*}=\ln(y)$. A simple linear model is fitted to the transformed data. The output is the following:
\begin{knitrout}
\definecolor{shadecolor}{rgb}{0.969, 0.969, 0.969}\color{fgcolor}\begin{kframe}
\begin{alltt}
\hlstd{logbact} \hlkwb{<-} \hlkwd{log}\hlstd{(bact)}
\hlstd{fit2} \hlkwb{<-} \hlkwd{lm}\hlstd{(logbact}\hlopt{~}\hlstd{min)}
\hlkwd{summary}\hlstd{(fit2)}
\end{alltt}
\begin{verbatim}
##
## Call:
## lm(formula = logbact ~ min)
##
## Residuals:
##       Min        1Q    Median        3Q       Max
## -0.184303 -0.083994  0.001453  0.072825  0.206246
##
## Coefficients:
##             Estimate Std. Error t value Pr(>|t|)
## (Intercept)  5.33878    0.07409   72.05 6.47e-15 ***
## min         -0.23617    0.01007  -23.46 4.49e-10 ***
## ---
## Signif. codes:
## 0 '***' 0.001 '**' 0.01 '*' 0.05 '.' 0.1 ' ' 1
##
## Residual standard error: 0.1204 on 10 degrees of freedom
## Multiple R-squared:  0.9822,	Adjusted R-squared:  0.9804
## F-statistic: 550.3 on 1 and 10 DF,  p-value: 4.489e-10
\end{verbatim}
\end{kframe}
\end{knitrout}

The fitted model is $$\hat{y}^{*}=5.33878-0.23617x,$$ where the parameters of the model are estimated by the best linear unbiased estimators. Figure~\ref{fig:simple:bact5} is a scatter plot of the transformed response variable versus the covariate, along with the fitted line. The scatter plot looks much more linear now than in (a).

\begin{figure}
\begin{center}
\begin{knitrout}
\definecolor{shadecolor}{rgb}{0.969, 0.969, 0.969}\color{fgcolor}
\includegraphics[width=\maxwidth]{figure/unnamed-chunk-40-1}

\end{knitrout}
\end{center}
\caption{Scatter Plot of the Logarithm of the Number of Bacteria versus the Minutes of Exposure to $300^{o}$ F}
\label{fig:simple:bact5}
\end{figure}

The ANOVA table is obtained using \textsf{R}:
\begin{knitrout}
\definecolor{shadecolor}{rgb}{0.969, 0.969, 0.969}\color{fgcolor}\begin{kframe}
\begin{alltt}
\hlkwd{anova}\hlstd{(fit2)}
\end{alltt}
\begin{verbatim}
## Analysis of Variance Table
##
## Response: logbact
##           Df Sum Sq Mean Sq F value    Pr(>F)
## min        1 7.9761  7.9761  550.33 4.489e-10 ***
## Residuals 10 0.1449  0.0145
## ---
## Signif. codes:
## 0 '***' 0.001 '**' 0.01 '*' 0.05 '.' 0.1 ' ' 1
\end{verbatim}
\end{kframe}
\end{knitrout}

The F-statistic for the test of significance of regression is 550.33, and it has 1 and 10 degrees of freedom, so the $p$-value is $$P[F_{(1,10)}>550.33]= 4.489\times10^{-10}.$$ Since the $p$-value is much smaller than 1\%, there is enough evidence to reject the null hypothesis that $\beta_{1}=0$ at the 1\% level. This model is significant.

The value of $R^{2}$ is very high at $98.22\%$. This means that about 98\% of the variation in the log of the number of bacteria in the canned food is explained by the minutes of exposure to $300^{o}$F. The model seems to perform very well, better than the model proposed in (b).

The Q-Q Plot of the studentized residuals is shown in Figure~\ref{fig:simple:bact6}. The dots are beautifully aligned with the standard normal quantiles. The normality assumption is appropriate. Figure~\ref{fig:simple:bact7} shows a plot of the studentized residuals versus the fitted values. The dots can be contained in horizontal bands and looks randomly scattered.

Finally, this model is adequate and the transformation used on the response variables fixed the problems in the model.

\begin{figure}
\begin{center}
\begin{knitrout}
\definecolor{shadecolor}{rgb}{0.969, 0.969, 0.969}\color{fgcolor}
\includegraphics[width=\maxwidth]{figure/unnamed-chunk-42-1}

\end{knitrout}
\end{center}
\caption{Q-Q Plot of Model for the Logarithm of the Number of Bacteria in Problem 5 c)} \label{fig:simple:bact6}
\end{figure}

\begin{figure}
\begin{center}
\begin{knitrout}
\definecolor{shadecolor}{rgb}{0.969, 0.969, 0.969}\color{fgcolor}
\includegraphics[width=\maxwidth]{figure/unnamed-chunk-43-1}

\end{knitrout}
\end{center}
\caption{Residuals versus the Fitted Values for Model for the Logarithm of the Number of Bacteria in Problem 5 c)}
\label{fig:simple:bact7}
\end{figure}

\end{enumerate}
\end{solution}

\newpage
\section*{Chapitre \ref{chap:multiple}}
\addcontentsline{toc}{section}{Chapitre \protect\ref{chap:multiple}}

\begin{solution}{3.1}
    Tout d'abord, selon le théorème
    \ref{thm:elements:derivee_fonction} de l'annexe
    \ref{chap:elements},
    \begin{displaymath}
      \frac{d}{d \mat{x}}\, f(\mat{x})^\prime \mat{A} f(\mat{x}) =
      2 \left( \frac{d}{d \mat{x}} f(\mat{x}) \right)^\prime \mat{A} f(\mat{x}).
    \end{displaymath}
    Il suffit, pour faire la démonstration, d'appliquer directement ce
    résultat à la forme quadratique
    \begin{displaymath}
      S(\betab) = (\mat{y} - \mat{X} \betab)^\prime (\mat{y} - \mat{X} \betab)
    \end{displaymath}
    avec $f(\betab) = \mat{y} - \mat{X} \betab$ et $\mat{A} =
    \mat{I}$, la matrice identité. On a alors
    \begin{align*}
      \frac{d}{d\betab} S(\betab)
      &= 2
      \left(
        \frac{d}{d\betab} (\mat{y} - \mat{X} \betab)
      \right)^\prime
      \mat{y} - \mat{X} \betab \\
      &= 2 (-\mat{X})^\prime (\mat{y} - \mat{X} \betab)\\
      &= -2 \mat{X}^\prime (\mat{y} - \mat{X} \betab).
    \end{align*}
    En posant ces dérivées exprimées sous forme matricielle
    simultanément égales à zéro, on obtient les équations normales à
    résoudre pour calculer l'estimateur des moindres carrés du vecteur
    $\betab$, soit
    \begin{displaymath}
      \mat{X}^\prime \mat{X} \betabh = \mat{X}^\prime \mat{y}.
    \end{displaymath}
    En isolant $\betabh$ dans l'équation ci-dessus, on obtient,
    finalement, l'estimateur des moindres carrés:
    \begin{displaymath}
      \betabh = (\mat{X}^\prime \mat{X})^{-1} \mat{X}^\prime \mat{y}.
    \end{displaymath}
  
\end{solution}
\begin{solution}{3.2}
    \begin{enumerate}
    \item On a un modèle sans variable explicative. Intuitivement, la
      meilleure prévision de $Y_t$ sera alors $\bar{Y}$. En effet,
      pour ce modèle,
      \begin{displaymath}
        \mat{X} =
        \begin{bmatrix}
          1 \\ \vdots \\ 1
        \end{bmatrix}_{n \times 1}
      \end{displaymath}
      et
      \begin{align*}
        \betabh
        &=\left(\mat{X}^\prime \mat{X} \right)^{-1} \mat{X}^\prime
        \mat{y} \\
        &=
        \left(
          \begin{bmatrix}
            1 & \cdots & 1
          \end{bmatrix}
          \begin{bmatrix}
            1 \\ \vdots \\ 1
          \end{bmatrix}
        \right)^{-1}
        \begin{bmatrix}
          1 & \cdots & 1
        \end{bmatrix}
        \begin{bmatrix}
          Y_1 \\ \vdots \\ Y_n
        \end{bmatrix}\\
        &= n^{-1} \sum_{t=1}^n Y_t \\
        &= \bar{Y}.
      \end{align*}
    \item Il s'agit du modèle de régression linéaire simple passant
      par l'origine, pour lequel la matrice de schéma est
      \begin{displaymath}
        \mat{X} =
        \begin{bmatrix}
          X_1 \\ \vdots \\ X_n
        \end{bmatrix}_{n \times 1}.
      \end{displaymath}
      Par conséquent,
      \begin{align*}
        \betabh
        &=
        \left(
          \begin{bmatrix}
            X_1 & \cdots & X_n
          \end{bmatrix}
          \begin{bmatrix}
            X_1 \\ \vdots \\ X_n
          \end{bmatrix}
        \right)^{-1}
        \begin{bmatrix}
          X_1 & \cdots & X_n
        \end{bmatrix}
        \begin{bmatrix}
          Y_1 \\ \vdots \\ Y_n
        \end{bmatrix} \\
        &=
        \left(
          \sum_{t=1}^n X_t^2
        \right)^{-1} \sum_{t=1}^n X_tY_t \\
        &= \frac{\sum_{t=1}^n X_t Y_t}{\sum_{t=1}^n X_t^2},
      \end{align*}
      tel qu'obtenu à l'exercice
      \ref{chap:simple}.\ref{ex:simple:origine}.
    \item On est ici en présence d'un modèle de régression multiple ne
      passant pas par l'origine et ayant deux variables explicatives.
      La matrice de schéma est alors
      \begin{displaymath}
        \mat{X} =
        \begin{bmatrix}
          1      & X_{11}  & X_{12} \\
          \vdots & \vdots & \vdots \\
          1      & X_{n1}  & X_{n2}
        \end{bmatrix}_{n \times 3}.
      \end{displaymath}
      Par conséquent,
      \begin{align*}
        \betabh
        &=
        \left(
          \begin{bmatrix}
            1     & \cdots & 1      \\
            X_{11} & \cdots & X_{n1} \\
            X_{12} & \cdots & X_{n2}
          \end{bmatrix}
          \begin{bmatrix}
            1      & X_{11}  & X_{12} \\
            \vdots & \vdots & \vdots \\
            1      & X_{n1}  & X_{n2}
          \end{bmatrix}
        \right)^{-1}
        \begin{bmatrix}
          1     & \cdots & 1      \\
          X_{11} & \cdots & X_{n1} \\
          X_{12} & \cdots & X_{n2}
        \end{bmatrix}
        \begin{bmatrix}
          Y_1 \\ \vdots \\ Y_n
        \end{bmatrix} \\
        & =
        \begin{bmatrix}
          n           & n \bar{X}_1             & n \bar{X}_2 \\
          n \bar{X}_1 & \sum_{t=1}^n X_{t1}^2    & \sum_{t=1}^n X_{t1}X_{t2} \\
          n \bar{X}_2 & \sum_{t=1}^n X_{t1}X_{t2} & \sum_{t=1}^n X_{t2}^2
        \end{bmatrix}^{-1}
        \begin{bmatrix}
          \sum_{t=1}^n Y_t \\ \sum_{t=1}^n X_{t1}Y_t \\ \sum_{t=1}^n
          X_{t2}Y_t
        \end{bmatrix}.
      \end{align*}
      L'inversion de la première matrice et le produit par la seconde
      sont laissés aux bons soins du lecteur plus patient que les
      rédacteurs de ces solutions.
    \end{enumerate}
  
\end{solution}
\begin{solution}{3.3}
    Dans le modèle de régression linéaire simple, la matrice schéma est
    \begin{displaymath}
      \mat{X} =
      \begin{bmatrix}
        1      & X_1    \\
        \vdots & \vdots \\
        1      & X_n
      \end{bmatrix}.
    \end{displaymath}
    Par conséquent,
    \begin{align*}
      \var{\betabh}
      &= \sigma^2 (\mat{X}^\prime \mat{X})^{-1} \\
      &= \sigma^2
      \left(
        \begin{bmatrix}
          1   & \cdots & 1  \\
          X_1 & \cdots & X_n
        \end{bmatrix}
        \begin{bmatrix}
          1      & X_1    \\
          \vdots & \vdots \\
          1      & X_n
        \end{bmatrix}
      \right)^{-1} \\
      &= \sigma^2
      \begin{bmatrix}
        n         & n \bar{X}        \\
        n \bar{X} & \sum_{t=1}^n X_t^2
      \end{bmatrix}^{-1}\\
      &= \frac{\sigma^2}{n \sum_{t=1}^n X_t^2 - n^2 \bar{X}^2}
      \begin{bmatrix}
        \sum_{t=1}^n X_t^2 & -n \bar{X} \\
        -n \bar{X}        & n
      \end{bmatrix} \\
      &= \frac{\sigma^2}{\sum_{t=1}^n (X_t - \bar{X}^2)}
      \begin{bmatrix}
        n^{-1} \sum_{t=1}^n X_t^2 & -\bar{X} \\
        -\bar{X}                 & 1
      \end{bmatrix},
    \end{align*}
    d'où
    \begin{align*}
      \var{\hat{\beta}_0}
      &= \sigma^2\, \frac{\sum_{t=1}^n X_t^2}{%
        n \sum_{t=1}^n (X_t - \bar{X})^2} \\
      &= \sigma^2\, \frac{\sum_{t=1}^n (X_t - \bar{X}^2) + n \bar{X}^2}{%
        n \sum_{t=1}^n (X_t - \bar{X})^2} \\
      \intertext{et}
      \var{\hat{\beta}_1}
      &= \frac{\sigma^2}{\sum_{t=1}^n (X_t - \bar{X})^2}.
    \end{align*}
    Ceci correspond aux résultats antérieurs.
  
\end{solution}
\begin{solution}{3.4}
    Dans les démonstrations qui suivent, trois relations de base
    seront utilisées:
    $\mat{e} = \mat{y - \hat{y}}$,
    $\mat{\hat{y}} = \mat{X} \betabh$ et
    $\betabh = (\mat{X}^\prime \mat{X})^{-1} \mat{X}^\prime \mat{y}$.
    \begin{enumerate}
    \item On a
      \begin{align*}
        \mat{X}^\prime \mat{e}
        &= \mat{X}^\prime (\mat{y - \hat{y}}) \\
        &= \mat{X}^\prime (\mat{y} - \mat{X} \betabh) \\
        &= \mat{X}^\prime \mat{y} - (\mat{X}^\prime \mat{X}) \betabh \\
        &= \mat{X}^\prime \mat{y} - (\mat{X}^\prime \mat{X})
        (\mat{X}^\prime \mat{X})^{-1} \mat{X}^\prime \mat{y} \\
        &= \mat{X}^\prime \mat{y} - \mat{X}^\prime \mat{y} \\
        &= \mat{0}.
      \end{align*}
      En régression linéaire simple, cela donne
      \begin{align*}
        \mat{X}^\prime \mat{e} &=
        \begin{bmatrix}
          1   & \cdots & 1  \\
          X_1 & \cdots & X_n
        \end{bmatrix}
        \begin{bmatrix}
          e_1 \\ \vdots \\ e_n
        \end{bmatrix} \\
        &=
        \begin{bmatrix}
          \sum_{t=1}^n e_t \\
          \sum_{t=1}^n X_t e_t
        \end{bmatrix}.
      \end{align*}
      Par conséquent, $\mat{X}^\prime \mat{e} = \mat{0}$ se simplifie
      en $\sum_{t=1}^n e_t = 0$ et $\sum_{t=1}^n X_t e_t = 0$ soit,
      respectivement, la condition pour que l'estimateur des moindres
      carrés soit sans biais et la seconde équation normale obtenue à
      la partie \ref{ex:simple:base:eq_normales}) de l'exercice
      \ref{chap:simple}.\ref{ex:simple:base}.
     \item On a
       \begin{align*}
         \hat{\mat{y}}^\prime \mat{e}
         &= (\mat{X}\betabh)^\prime (\mat{y - \hat{y}}) \\
         &= \betabh^\prime \mat{X}^\prime (\mat{y} - \mat{X}\betabh) \\
         &= \betabh^\prime \mat{X}^\prime \mat{y} - \betabh^\prime
         (\mat{X}^\prime \mat{X}) \betabh \\
         &= \betabh^\prime \mat{X}^\prime \mat{y} - \betabh^\prime
         (\mat{X}^\prime \mat{X}) (\mat{X}^\prime \mat{X})^{-1}
         \mat{X}^\prime \mat{y} \\
         &= \betabh^\prime \mat{X}^\prime \mat{y} - \betabh^\prime
         \mat{X}^\prime \mat{y} \\
         &= 0.
       \end{align*}
       Pour tout modèle de régression cette équation peut aussi
       s'écrire sous la forme plus conventionnelle $\sum_{t=1}^n
       \hat{Y}_t e_t = 0$. Cela signifie que le produit scalaire entre
       le vecteur des prévisions et celui des erreurs doit être nul
       ou, autrement dit, que les vecteurs doivent être orthogonaux.
       C'est là une condition essentielle pour que l'erreur
       quadratique moyenne entre les vecteurs $\mat{y}$ et
       $\mat{\hat{y}}$ soit minimale. (Pour de plus amples détails sur
       l'interprétation géométrique du modèle de régression, consulter
       {\shorthandoff{:} \citet[chapitres 20 et
         21]{Draper:regression:1998}}.)
         D'ailleurs, on constate que
       $\hat{\mat{y}}^\prime \mat{e} = \betabh^\prime \mat{X}^\prime
       \mat{e}$ et donc, en supposant sans perte de généralité que
       $\betabh \ne \mat{0}$,
       que $\hat{\mat{y}}^\prime \mat{e} = 0$
       et $\mat{X}^\prime \mat{e} = \mat{0}$ sont des conditions en
       tous points équivalentes.
     \item On a
       \begin{align*}
         \mat{\hat{y}}^\prime \mat{\hat{y}}
         &= (\mat{X} \betabh)^\prime \mat{X} \betabh \\
         &= \betabh^\prime (\mat{X}^\prime \mat{X}) \betabh \\
         &= \betabh^\prime (\mat{X}^\prime \mat{X}) (\mat{X}^\prime
         \mat{X})^{-1} \mat{X}^\prime \mat{y} \\
         &= \betabh^\prime \mat{X}^\prime \mat{y}.
       \end{align*}
       Cette équation est l'équivalent matriciel de l'identité
       \begin{align*}
         \SSR
         &= \hat{\beta}_1^2 \sum_{t = 1}^n (X_t - \bar{X})^2 \\
         &= \frac{S_{XY}^2}{S_{XX}}
       \end{align*}
       utilisée à plusieurs reprises dans les solutions du chapitre
       \ref{chap:simple}.  En effet, en régression linéaire simple,
       $\mat{\hat{y}}^\prime \mat{\hat{y}} = \sum_{t = 1}^n
       \hat{Y}_t^2 = \sum_{t = 1}^n (\hat{Y} - \bar{Y})^2 + n
       \bar{Y}^2 = \SSR + n \bar{Y}^2$ et
       \begin{align*}
         \betabh^\prime \mat{X}^\prime \mat{y}
         &= \hat{\beta}_0 n \bar{Y} + \hat{\beta}_1 \sum_{t = 1}^n X_t Y_t \\
         &= (\bar{Y} - \hat{\beta}_1 \bar{X}) n \bar{Y} +
         \hat{\beta}_1 \sum_{t = 1}^n X_t Y_t \\
         &= \hat{\beta}_1 \sum_{t = 1}^n (X_t - \bar{X})(Y_t -
         \bar{Y}) + n \bar{Y}^2 \\
         &= \frac{S_{XY}^2}{S_{XX}} + n \bar{Y}^2,
       \end{align*}
       d'où $\SSR = S_{XY}^2/S_{XX}$.
     \end{enumerate}
   
\end{solution}
\begin{solution}{3.5}
    \begin{enumerate}
    \item Premièrement, $Y_0 = \mat{x}_0 \betab + \varepsilon_0$ avec
      $\esp{\varepsilon_0} = 0$. Par conséquent, $\esp{Y_0} = \esp{x_0
        \betab + \varepsilon_0} = \mat{x}_0 \betab$. Deuxièmement,
      $\esp{\hat{Y}_0} = \esp{\mat{x}_0 \betabh} = \mat{x}_0
      \esp{\betabh} = x_0 \betab$ puisque l'estimateur des moindres
      carrés de $\betab$ est sans biais. Ceci complète la preuve.
    \item Tout d'abord, $\esp{(\hat{Y}_0 - \esp{Y_0})^2} =
      \varmat{\hat{Y}_0} = \var{\hat{Y}_0}$ puisque la matrice de
      variance-covariance du vecteur aléatoire $\hat{Y}_0$ ne
      contient, ici, qu'une seule valeur. Or, par le théorème
      \ref{thm:elements:esp_var},
       \begin{align*}
         \var{\hat{Y}_0}
         &= \varmat{\mat{x}_0 \betabh} \\
         &= \mat{x}_0 \varmat{\betabh} \mat{x}_0^\prime \\
         &= \sigma^2 \mat{x}_0 (\mat{X}^\prime \mat{X})^{-1}
         \mat{x}_0^\prime.
       \end{align*}
       Afin de construire un intervalle de confiance pour $\esp{Y_0}$,
       on ajoute au modèle l'hypothèse $\vepsb \sim N(\mat{0}, \sigma^2
       \mat{I})$. Par linéarité de l'estimateur des moindres carrés,
       on a alors $\hat{Y}_0 \sim N(\esp{Y_0}, \var{\hat{Y}_0})$. Par
       conséquent,
       \begin{displaymath}
         \Pr
         \left[
           -z_{\alpha/2}
           \leq
           \frac{\hat{Y} - \esp{\hat{Y}_0}}{\sqrt{\var{\hat{Y}_0}}}
           \leq
           z_{\alpha/2}
         \right] = 1 - \alpha
       \end{displaymath}
       d'où un intervalle de confiance de niveau $1 - \alpha$ pour
       $\esp{Y_0}$ est
       \begin{displaymath}
         \esp{Y_0}
         \in \hat{Y}_0 \pm z_{\alpha/2}\, \sigma\,
         \sqrt{\mat{x}_0 (\mat{X}^\prime \mat{X})^{-1} \mat{x}_0^\prime}.
       \end{displaymath}
       Si la variance $\sigma^2$ est inconnue et estimée par $s^2$,
       alors la distribution normale est remplacée par une
       distribution de Student avec $n - p - 1$ degrés de
       liberté. L'intervalle de confiance devient alors
       \begin{displaymath}
         \esp{Y_0}
         \in \hat{Y}_0 \pm t_{\alpha/2}(n - p - 1)\, s\,
         \sqrt{\mat{x}_0 (\mat{X}^\prime \mat{X})^{-1} \mat{x}_0^\prime}.
       \end{displaymath}
     \item Par le résultat obtenu en a) et en supposant que
       $\Cov(\varepsilon_0, \varepsilon_t) = 0$ pour tout $t = 1,
       \dots, n$, on a
       \begin{align*}
         \esp{(Y_0 - \hat{Y}_0)^2}
         &= \var{Y_0 - \hat{Y}_0} \\
         &= \var{Y_0} + \var{\hat{Y}_0} \\
         &= \sigma^2 (1 + \mat{x}_0 (\mat{X}^\prime \mat{X})^{-1}
         \mat{x}_0^\prime).
       \end{align*}
       Ainsi, avec l'hypothèse sur le terme d'erreur énoncée en b),
       $Y_0 - \hat{Y}_0 \sim N(0, \var{Y_0 - \hat{Y}_0})$. En suivant
       le même cheminement qu'en b), on détermine qu'un intervalle de
       confiance de niveau $1 - \alpha$ pour $Y_0$ est
       \begin{displaymath}
         Y_0
         \in \hat{Y}_0 \pm z_{\alpha/2}\, \sigma\,
         \sqrt{1 + \mat{x}_0 (\mat{X}^\prime \mat{X})^{-1} \mat{x}_0^\prime}.
       \end{displaymath}
       ou, si la variance $\sigma^2$ est inconnue et estimée par
       $s^2$,
       \begin{displaymath}
         Y_0
         \in \hat{Y}_0 \pm t_{\alpha/2}(n - p - 1)\, s\,
         \sqrt{1 + \mat{x}_0 (\mat{X}^\prime \mat{X})^{-1} \mat{x}_0^\prime}.
       \end{displaymath}
     \end{enumerate}
   
\end{solution}
\begin{solution}{3.6}
    On a la relation suivante liant la statistique $F$ et le
    coefficient de détermination $R^2$:
    \begin{displaymath}
      F = \frac{R^2}{1 - R^2}\, \frac{n - p - 1}{p}
    \end{displaymath}
    La principale inconnue dans le problème est $n$, le nombre de
    données. Or,
    \begin{align*}
      n
      &= p F \left( \frac{1 - R^2}{R^2} \right) + p + 1 \\
      &= 3 (5,438) \left( \frac{1 - 0,521}{0,521} \right) + 3 + 1 \\
      &= 19.
    \end{align*}
    Soit $F$ une variable aléatoire dont la distribution est une loi
    de Fisher avec $3$ et $19 - 3 - 1 = 15$ degrés de liberté, soit la
    même distribution que la statistique $F$ du modèle. On obtient la
    valeur $p$ du test global de validité du modèle dans un tableau de
    quantiles de la distribution $F$ ou avec la fonction \texttt{pf}
    dans R:
    \begin{align*}
      \Pr[F > 5,438] = 0,0099
    \end{align*}
  
\end{solution}
\begin{solution}{3.7}
    \begin{enumerate}
    \item On a
      \begin{align*}
        \betabh
        &= (\mat{X}^\prime \mat{X})^{-1} \mat{X}^\prime \mat{y} \\
        &= \frac{1}{2}
        \left[
          \begin{array}{rrrr}
            -6 & 34 & -13 & -13 \\
            2 & -4 &   1 &   1 \\
            0 & -2 & 1 & 1
          \end{array}
        \right]
        \begin{bmatrix}
          17 \\ 12 \\ 14 \\ 13
        \end{bmatrix} \\
        &= \frac{1}{2}
        \left[
          \begin{array}{r}
            -45 \\ 13 \\ 3
          \end{array}
        \right] =
        \left[
          \begin{array}{r}
            -22,5 \\ 6,5 \\ 1,5
          \end{array}
        \right]
      \end{align*}
    \item Avec les résultats de la partie a), on a
      \begin{align*}
        \mat{\hat{y}} &= \mat{X} \betabh =
        \begin{bmatrix}
          17 \\ 12 \\ 13,5 \\ 13,5
        \end{bmatrix}, \\
        \mat{e} &= \mat{y} - \mat{\hat{y}} =
        \left[
          \begin{array}{r}
            0 \\ 0 \\ 0,5 \\ -0,5
          \end{array}
        \right]
      \end{align*}
      et $\bar{Y} = 14$. Par conséquent,
      \begin{align*}
        \SST
        &= \mat{y}^\prime \mat{y} - n \bar{Y}^2 = 14 \\
        \SSE
        &= \mat{e}^\prime \mat{e} = 0,5 \\
        \SSR &= \SST - \SSR = 13,5,
      \end{align*}
      d'où le tableau d'analyse de variance est le suivant:
      \begin{center}
        \begin{tabular}{lrrrr}
          \toprule
          Source & SS & d.l. & MS & $F$ \\
          \midrule
          Régression & $13,5$ & 2 & $6,75$ &  $13,5$ \\
          Erreur     &  $0,5$ & 1 &  $0,5$ \\
          \midrule
          Total      &   $14$ & \\
          \bottomrule
        \end{tabular}
      \end{center}
      Le coefficient de détermination est
      \begin{displaymath}
        R^2 = 1 - \frac{\SSE}{\SST} = 0,9643.
      \end{displaymath}
    \item On sait que $\var{\hat{\beta}_i} = \sigma^2 c_{ii}$, où
      $c_{ii}$ est l'élément en position $(i+1, i+1)$ de la matrice
      $(\mat{X}^\prime \mat{X})^{-1}$. Or, $\hat{\sigma}^2 = s^2 =
      \text{MSE} = 0,5$, tel que calculé en b). Par conséquent, la
      statistique $t$ du test $H_0: \beta_1 = 0$ est
      \begin{displaymath}
        t
        = \frac{\hat{\beta}_1}{s \sqrt{c_{11}}}
        = \frac{6,5}{\sqrt{0,5 (\frac{11}{2})}}
        = 3,920,
      \end{displaymath}
      alors que celle du test $H_0: \beta_2 = 0$ est
      \begin{displaymath}
        t
        = \frac{\hat{\beta}_2}{s \sqrt{c_{22}}}
        = \frac{1,5}{\sqrt{0,5 (\frac{3}{2})}}
        = 1,732.
      \end{displaymath}
      À un niveau de signification de 5~\%, la valeur critique de ces
      tests est $t_{0,025}(1) = 12,706$. Dans les deux cas, on ne
      rejette donc pas $H_0$, les variables $X_1$ et $X_2$ ne sont pas
      significatives dans le modèle.
    \item Soit $\mat{x}_0 = \begin{bmatrix} 1 & 3,5 & 9 \end{bmatrix}$
      et $Y_0$ la valeur de la variable dépendante correspondant à
      $\mat{x}_0$. La prévision de $Y_0$ donnée par le modèle trouvé
      en a) est
      \begin{align*}
        \hat{Y}_0
        &= \mat{x}_0 \betabh \\
        &= -22,5 + 6,5(3,5) + 1,5(9) \\
        &= 13,75.
      \end{align*}
      D'autre part,
      \begin{align*}
        \widehat{\text{Var}}[Y_0 - \hat{Y}_0]
        &= s^2 (1 + \mat{x}_0 (\mat{X}^\prime \mat{X})^{-1} \mat{x}_0^\prime) \\
        &= 1,1875.
      \end{align*}
      Par conséquent, un intervalle de confiance à 95~\% pour $Y_0$
      est
      \begin{align*}
        \esp{Y_0} &\in \hat{Y}_0 \pm t_{0,025}(1) s \sqrt{1 + \mat{x}_0
          (\mat{X}^\prime \mat{X})^{-1} \mat{x}_0^\prime} \\
        &\in 13,75 \pm 12,706 \sqrt{1,1875} \\
        &\in (-0,096, 27,596).
      \end{align*}
    \end{enumerate}
  
\end{solution}
\begin{solution}{3.8}
    \begin{enumerate}
    \item On importe les données dans \textsf{R}, puis on effectue les
      conversions nécessaires. Comme précédemment, la variable
      \texttt{consommation} contient la consommation des voitures en
      $\ell$/100~km et la variable \texttt{poids} le poids en
      kilogrammes. On ajoute la variable \texttt{cylindree}, qui
      contient la cylindrée des voitures en litres.
\begin{knitrout}
\definecolor{shadecolor}{rgb}{0.969, 0.969, 0.969}\color{fgcolor}\begin{kframe}
\begin{alltt}
\hlstd{carburant} \hlkwb{<-} \hlkwd{read.table}\hlstd{(}\hlstr{"carburant.dat"}\hlstd{,} \hlkwc{header} \hlstd{=} \hlnum{TRUE}\hlstd{)}
\hlstd{consommation} \hlkwb{<-} \hlnum{235.1954}\hlopt{/}\hlstd{carburant}\hlopt{$}\hlstd{mpg}
\hlstd{poids} \hlkwb{<-} \hlstd{carburant}\hlopt{$}\hlstd{poids} \hlopt{*} \hlnum{0.45455} \hlopt{*} \hlnum{1000}
\hlstd{cylindree} \hlkwb{<-} \hlstd{carburant}\hlopt{$}\hlstd{cylindree} \hlopt{*} \hlnum{2.54}\hlopt{^}\hlnum{3}\hlopt{/}\hlnum{1000}
\end{alltt}
\end{kframe}
\end{knitrout}
    \item La fonction \texttt{summary} fournit l'information
      essentielle pour juger de la validité et de la qualité du
      modèle:
\begin{knitrout}
\definecolor{shadecolor}{rgb}{0.969, 0.969, 0.969}\color{fgcolor}\begin{kframe}
\begin{alltt}
\hlstd{fit} \hlkwb{<-} \hlkwd{lm}\hlstd{(consommation} \hlopt{~} \hlstd{poids} \hlopt{+} \hlstd{cylindree)}
\hlkwd{summary}\hlstd{(fit)}
\end{alltt}
\begin{verbatim}
##
## Call:
## lm(formula = consommation ~ poids + cylindree)
##
## Residuals:
##     Min      1Q  Median      3Q     Max
## -1.8799 -0.5595  0.1577  0.6051  1.7900
##
## Coefficients:
##              Estimate Std. Error t value Pr(>|t|)
## (Intercept) -3.049304   1.098281  -2.776  0.00877 **
## poids        0.012677   0.001512   8.386 6.85e-10 ***
## cylindree   -1.122696   0.333479  -3.367  0.00186 **
## ---
## Signif. codes:
## 0 '***' 0.001 '**' 0.01 '*' 0.05 '.' 0.1 ' ' 1
##
## Residual standard error: 0.9156 on 35 degrees of freedom
## Multiple R-squared:  0.8927,	Adjusted R-squared:  0.8866
## F-statistic: 145.6 on 2 and 35 DF,  p-value: < 2.2e-16
\end{verbatim}
\end{kframe}
\end{knitrout}
      Le modèle est donc le suivant:
      \begin{displaymath}
        Y_t =
        -3,049 +
        0,01268 X_{t1} +
        -1,123 X_{t2} +
        \varepsilon_t,
        \quad
        \vepsb_t \sim N(0,
        0,9156^2 \mat{I})
      \end{displaymath}
      où $Y_t$ est la consommation en litres aux 100 kilomètres,
      $X_{t1}$ le poids en kilogrammes et $X_{t2}$ la cylindrée en
      litres. La faible valeur $p$ du test $F$ indique une régression
      globalement très significative. Les tests $t$ des paramètres
      individuels indiquent également que les deux variables du modèle
      sont significatives. Enfin, le $R^2$ de %
      0,8927 %$
      confirme que l'ajustement du modèle est toujours bon.
    \item On veut calculer un intervalle de confiance pour la
      consommation prévue d'une voiture de \nombre{1350}~kg ayant un
      moteur d'une cylindrée de 1,8 litres. On obtient, avec la
      fonction \texttt{predict}:
\begin{knitrout}
\definecolor{shadecolor}{rgb}{0.969, 0.969, 0.969}\color{fgcolor}\begin{kframe}
\begin{alltt}
\hlkwd{predict}\hlstd{(fit,} \hlkwc{newdata} \hlstd{=} \hlkwd{data.frame}\hlstd{(}\hlkwc{poids} \hlstd{=} \hlnum{1350}\hlstd{,} \hlkwc{cylindree} \hlstd{=} \hlnum{1.8}\hlstd{),}
        \hlkwc{interval} \hlstd{=} \hlstr{"prediction"}\hlstd{)}
\end{alltt}
\begin{verbatim}
##        fit      lwr      upr
## 1 12.04325 9.959855 14.12665
\end{verbatim}
\end{kframe}
\end{knitrout}
    \end{enumerate}
  
\end{solution}
\begin{solution}{3.9}
    Il y a plusieurs réponses possibles pour cet exercice. Si l'on
    cherche, tel que suggéré dans l'énoncé, à distinguer les voitures
    sport des minifourgonnettes (en supposant que ces dernières ont
    moins d'accidents que les premières), alors on pourrait
    s'intéresser, en premier lieu, à la variable \texttt{peak.rpm}. Il
    s'agit du régime moteur maximal, qui est en général beaucoup plus
    élevé sur les voitures sport. Puisque l'on souhaite expliquer le
    montant total des sinistres de différents types de voitures, il
    devient assez naturel de sélectionner également la variable
    \texttt{price}, soit le prix du véhicule. Un véhicule plus luxueux
    coûte en général plus cher à faire réparer à dommages égaux.
    Voyons l'effet de l'ajout, pas à pas, de ces deux variables au
    modèle précédent ne comportant que la variable
    \texttt{horsepower}:
\begin{knitrout}
\definecolor{shadecolor}{rgb}{0.969, 0.969, 0.969}\color{fgcolor}\begin{kframe}
\begin{alltt}
\hlstd{autoprice} \hlkwb{<-} \hlkwd{read.table}\hlstd{(}\hlstr{"data/auto-price.dat"}\hlstd{,} \hlkwc{header} \hlstd{=} \hlnum{TRUE}\hlstd{)}
\hlstd{fit1} \hlkwb{<-} \hlkwd{lm}\hlstd{(losses} \hlopt{~} \hlstd{horsepower} \hlopt{+} \hlstd{peak.rpm,} \hlkwc{data} \hlstd{= autoprice)}
\hlkwd{summary}\hlstd{(fit1)}
\end{alltt}
\begin{verbatim}
##
## Call:
## lm(formula = losses ~ horsepower + peak.rpm, data = autoprice)
##
## Residuals:
##     Min      1Q  Median      3Q     Max
## -67.973 -24.074  -6.373  18.049 130.301
##
## Coefficients:
##              Estimate Std. Error t value Pr(>|t|)
## (Intercept)  5.521414  29.967570   0.184 0.854060
## horsepower   0.318477   0.086840   3.667 0.000336 ***
## peak.rpm     0.016639   0.005727   2.905 0.004205 **
## ---
## Signif. codes:
## 0 '***' 0.001 '**' 0.01 '*' 0.05 '.' 0.1 ' ' 1
##
## Residual standard error: 33.44 on 156 degrees of freedom
## Multiple R-squared:  0.1314,	Adjusted R-squared:  0.1203
## F-statistic:  11.8 on 2 and 156 DF,  p-value: 1.692e-05
\end{verbatim}
\begin{alltt}
\hlkwd{anova}\hlstd{(fit1)}
\end{alltt}
\begin{verbatim}
## Analysis of Variance Table
##
## Response: losses
##             Df Sum Sq Mean Sq F value    Pr(>F)
## horsepower   1  16949 16948.5 15.1573 0.0001463 ***
## peak.rpm     1   9437  9437.0  8.4397 0.0042049 **
## Residuals  156 174435  1118.2
## ---
## Signif. codes:
## 0 '***' 0.001 '**' 0.01 '*' 0.05 '.' 0.1 ' ' 1
\end{verbatim}
\end{kframe}
\end{knitrout}
    La variable \texttt{peak.rpm} est significative, mais le $R^2$
    demeure faible. Ajoutons maintenant la variable \texttt{price} au modèle:
\begin{knitrout}
\definecolor{shadecolor}{rgb}{0.969, 0.969, 0.969}\color{fgcolor}\begin{kframe}
\begin{alltt}
\hlstd{fit2} \hlkwb{<-} \hlkwd{lm}\hlstd{(losses} \hlopt{~} \hlstd{horsepower} \hlopt{+} \hlstd{peak.rpm} \hlopt{+} \hlstd{price,} \hlkwc{data} \hlstd{= autoprice)}
\hlkwd{summary}\hlstd{(fit2)}
\end{alltt}
\begin{verbatim}
##
## Call:
## lm(formula = losses ~ horsepower + peak.rpm + price, data = autoprice)
##
## Residuals:
##     Min      1Q  Median      3Q     Max
## -66.745 -25.214  -5.867  18.407 130.032
##
## Coefficients:
##               Estimate Std. Error t value Pr(>|t|)
## (Intercept) -0.6972172 31.3221462  -0.022  0.98227
## horsepower   0.2414922  0.1408272   1.715  0.08838 .
## peak.rpm     0.0181386  0.0061292   2.959  0.00357 **
## price        0.0005179  0.0007451   0.695  0.48803
## ---
## Signif. codes:
## 0 '***' 0.001 '**' 0.01 '*' 0.05 '.' 0.1 ' ' 1
##
## Residual standard error: 33.49 on 155 degrees of freedom
## Multiple R-squared:  0.1341,	Adjusted R-squared:  0.1173
## F-statistic: 8.001 on 3 and 155 DF,  p-value: 5.42e-05
\end{verbatim}
\begin{alltt}
\hlkwd{anova}\hlstd{(fit2)}
\end{alltt}
\begin{verbatim}
## Analysis of Variance Table
##
## Response: losses
##             Df Sum Sq Mean Sq F value    Pr(>F)
## horsepower   1  16949 16948.5 15.1071 0.0001502 ***
## peak.rpm     1   9437  9437.0  8.4118 0.0042702 **
## price        1    542   542.1  0.4832 0.4880298
## Residuals  155 173893  1121.9
## ---
## Signif. codes:
## 0 '***' 0.001 '**' 0.01 '*' 0.05 '.' 0.1 ' ' 1
\end{verbatim}
\end{kframe}
\end{knitrout}
    Du moins avec les variables \texttt{horsepower} et
    \texttt{peak.rpm}, la variable \texttt{price} n'est pas
    significative. D'ailleurs, l'augmentation du $R^2$ suite à l'ajout
    de cette variable est minime. À ce stade de l'analyse, il vaudrait
    sans doute mieux reprendre tout depuis le début avec d'autres
    variables. Des méthodes de sélection des variables seront étudiées
    plus avant dans le chapitre.
  
\end{solution}
\begin{solution}{3.10}
    \begin{enumerate}
    \item On a $p = 3$ variables explicatives et, du nombre de degrés
      de liberté de la statistique $F$, on apprend que $n - p - 1 =
      16$. Par conséquent, $n = 16 + 3 + 1 = 20$. Les dimensions des
      vecteurs et de la matrice de schéma dans la représentation
      $\mat{y} = \mat{X} \betab + \vepsb$ sont donc: $n \times 1 = 20
      \times 1$ pour les vecteurs $\mat{y}$ et $\vepsb$, $n \times (p
      + 1) = 20 \times 4$ pour la matrice $\mat{X}$, $(p + 1) \times
      1$ pour le vecteur $\betab$.
    \item La valeur $p$ associée à la statistique $F$ est, à toute fin
      pratique, nulle. Cela permet de rejeter facilement l'hypothèse
      nulle selon laquelle la régression n'est pas significative.
    \item On doit se fier ici au résultat du test $t$ associé à la
      variable $X_2$. Dans les résultats obtenus avec \textsf{R}, on
      voit que la valeur $p$ de la statistique $t$ du paramètre
      $\beta_2$ est $0,0916$. Cela signifie que jusqu'à un seuil de
      signification de 9,16~\% (ou un niveau de confiance supérieur à
      90,84~\%), on ne peut rejeter l'hypothèse $H_0: \beta_2 = 0$ en
      faveur de $H_1: \beta_2 \ne 0$. Il s'agit néanmoins d'un cas
      limite et il est alors du ressort de l'analyste de décider
      d'inclure ou non le revenu disponible dans le modèle.
    \item Le coefficient de détermination est de $R^2 = 0,981$. Cela
      signifie que le prix de la bière, le revenu disponible et la
      demande de l'année précédente expliquent plus de 98~\% de la
      variation de la demande en bière. L'ajustement du modèle aux
      données est donc particulièrement bon. Il est tout à fait
      possible d'obtenir un $R^2$ élevé et, simultanément, toutes les
      statistiques $t$ non significatives: comme chaque test $t$
      mesure l'impact d'une variable sur la régression étant donné la
      présence des autres variables, il suffit d'avoir une bonne
      variable dans un modèle pour obtenir un $R^2$ élevé et une ou
      plusieurs autres variables redondantes avec la première pour
      rendre les tests $t$ non significatifs.
    \end{enumerate}
  
\end{solution}
\begin{solution}{3.11}
    \begin{enumerate}
    \item L'information demandée doit évidemment être extraite des
      deux tableaux d'analyse de variance fournis dans l'énoncé. Il
      importe, ici, de savoir que le résultat de la fonction
      \texttt{anova} de \textsf{R} est un tableau d'analyse de
      variance séquentiel, où chaque ligne identifiée par le nom d'une
      variable correspond au test $F$ partiel résultant de l'ajout de
      cette variable au modèle. Ainsi, du premier tableau on obtient
      les sommes de carrés
      \begin{align*}
        \SSR(X_2)          &= 45,59085 \\
        \SSR(X_3|X_2)      &= 8,76355 \\
        \intertext{alors que du second tableau on a}
        \SSR(X_1)          &= 45,59240 \\
        \SSR(X_2|X_1)      &= 0,01842 \\
        \SSR(X_3|X_1, X_2) &= 8,78766,
      \end{align*}
      ainsi que
      \begin{align*}
        \MSE
        &= \frac{SSE(X_1, X_2, X_3)}{n - p - 1} \\
        &= 0,44844.
      \end{align*}
      \begin{enumerate}[i)]
      \item Le test d'hypothèse $H_0: \beta_1 = \beta_2 = \beta_3 = 0$
        est le test global de validité du modèle. La statistique $F$
        pour ce test est
        \begin{align*}
          F
          &= \frac{\SSR(X_1, X_2, X_3)/3}{\MSE} \\
          &= \frac{(\SSR(X_1) + \SSR(X_2|X_1) + \SSR(X_3|X_1,X_2))/3}{\MSE} \\
          &= \frac{(45,5924 + 0,01842 + 8,78766)/3}{0,44844} \\
          &= 40,44.
        \end{align*}
        Puisque la statistique $\MSE$ a 21 degrés de liberté, la
        statistique $F$ en a 3 et 21.
      \item Pour tester cette hypothèse, il faut utiliser un test $F$
        partiel. On teste si la variable $X_1$ est significative dans
        la régression globale. La statistique du test est alors
        \begin{align*}
          F^*
          &= \frac{\SSR(X_1|X_2,X_3)/1}{\MSE} \\
          &= \frac{\SSR(X_1, X_2, X_3) - \SSR(X_2, X_3)}{\MSE} \\
          &= \frac{\SSR(X_1, X_2, X_3) - \SSR(X_2) - \SSR(X_3|X_2)}{\MSE} \\
          &= \frac{54,39848 - 45,59085 - 8,76355}{0,44844} \\
          &= 0,098,
        \end{align*}
        avec 1 et 21 degrés de liberté.
      \item Cette fois, on teste si les variables $X_2$ et $X_3$ (les
        deux ensemble) sont significatives dans la régression globale.
        On effectue donc encore un test $F$ partiel avec la
        statistique
        \begin{align*}
          F^*
          &= \frac{\SSR(X_2, X_3|X_1)/2}{\MSE} \\
          &= \frac{(\SSR(X_1, X_2, X_3) - \SSR(X_1))/2}{\MSE} \\
          &= \frac{(54,39848 - 45,5924)/2}{0,44844} \\
          &= 9,819,
        \end{align*}
        avec 2 et 21 degrés de liberté.
      \end{enumerate}
    \item À la lecture du premier tableau d'analyse de variance que
      tant les variables $X_2$ que $X_3$ sont significatives dans le
      modèle. Par contre, comme on le voit dans le second tableau, la
      variable $X_2$ devient non significative dès lors que la
      variable $X_1$ est ajoutée au modèle. (L'impact de la variable
      $X_3$ demeure, lui, inchangé.) Cela signifie que les variables
      $X_1$ et $X_2$ sont redondantes et qu'il faut choisir l'une ou
      l'autre, mais pas les deux. Par conséquent, les choix de modèle
      possibles sont $X_1$ et $X_3$, ou $X_2$ et $X_3$.
    \end{enumerate}
  
\end{solution}
\begin{solution}{3.12}
    La statistique à utiliser pour faire ce test $F$ partiel est
    \begin{align*}
      F^*
      &= \frac{\SSR(X_2, X_3|X_1, X_4)/2}{\MSE} \\
      &= \frac{\SSR(X_1, X_2, X_3, X_4) - \SSR(X_1, X_4)}{2\, \MSE} \\
      &= \frac{\SSR - \SSR(X_4) - \SSR(X_1|X_4)}{2 s^2}
    \end{align*}
    où $\SSR = \SSR(X_1,X_2,X_3,X_4)$. Or,
    \begin{align*}
      R^2
      &= \frac{\SSR}{\SST} \\
      &= \frac{\SSR}{\SSR + \SSE}, \\
      \intertext{d'où}
      \SSR
      &= \frac{R^2}{1 - R^2}\, \SSE \\
      &= \frac{R^2}{1 - R^2}\, \MSE (n - p - 1) \\
      &= \frac{0,6903}{1 - 0,6903}\, (26,41) (506 - 4 - 1) \\
      &= \nombre{29492}.
    \end{align*}
    Par conséquent,
    \begin{align*}
      F^*
      &= \frac{\nombre{29492} - \nombre{2668} - \nombre{21348}}{%
        (2) (26,41)} \\
      &= 103,67.
    \end{align*}
  
\end{solution}
\begin{solution}{3.13}
    \begin{enumerate}
    \item Tout d'abord, si $Z \sim N(0,1)$ et $V \sim \upchi^2(r)$
      alors, par définition,
      \begin{displaymath}
        \frac{Z}{\sqrt{V/r}} \sim t(r).
      \end{displaymath}
      Tel que mentionné dans l'énoncé, $\hat{\beta}_i \sim N(\beta_i,
      \sigma^2 c_{ii})$ ou, de manière équivalente,
      \begin{displaymath}
        \frac{\hat{\beta}_i - \beta_i}{\sigma \sqrt{c_{ii}}} \sim
        N(0, 1).
      \end{displaymath}
      Par conséquent,
      \begin{displaymath}
        \frac{\frac{\hat{\beta}_i - \beta_i}{\sigma \sqrt{c_{ii}}}}{%
          \sqrt{\frac{\SSE}{\sigma^2(n - p - 1)}}}
        = \frac{\hat{\beta}_i - \beta_i}{s \sqrt{c_{ii}}}
        \sim t(n - p - 1).
      \end{displaymath}
    \item En régression linéaire simple, $c_{11} = 1/\sum_{t = 1}^n
      (X_t - \bar{X})^2 = 1/S_{XX}$ et $\sigma^2 c_{11} =
      \var{\hat{\beta}_1}$. Le résultat général en a) se réduit donc,
      en régression linéaire simple, au résultat bien connu du test $t$
      sur le paramètre $\beta_1$
      \begin{displaymath}
        \frac{\hat{\beta}_1 - \beta_1}{s \sqrt{1/S_{XX}}}
        \sim t(n - 1 - 1).
      \end{displaymath}
    \end{enumerate}
  
\end{solution}
\begin{solution}{3.14}
    En suivant les indications donnée dans l'énoncé, on obtient aisément
    \begin{align*}
      \frac{d}{d\betab} S(\betab)
      &= 2 \left(\frac{d}{d\betab}(\mat{y} - \mat{X}
        \betab)\right)^\prime \mat{W} (\mat{y} - \mat{X}
        \betab) \\
      &= - 2 \mat{X}^\prime \mat{W} (\mat{y} - \mat{X}
        \betab) \\
      &= -2 (\mat{X}^\prime \mat{W} \mat{y} - \mat{X}^\prime \mat{W}
      \mat{X} \betab).
    \end{align*}
    Par conséquent, les équations normales à résoudre pour trouver
    l'estimateur $\betabh^*$ minimisant la somme de carrés pondérés
    $S(\betab)$ sont $(\mat{X}^\prime\, \mat{W}\, \mat{X}) \betabh^* =
    \mat{X}^\prime\, \mat{W}\, \mat{y}$ et l'estimateur des moindres
    carrés pondérés est
    \begin{displaymath}
      \betabh^* = (\mat{X}^\prime \mat{W} \mat{X})^{-1} \mat{X}^\prime
      \mat{W} \mat{y}.
    \end{displaymath}
  
\end{solution}
\begin{solution}{3.15}
    De manière tout à fait générale, l'estimateur linéaire sans biais
    à variance minimale dans le modèle de régression linéaire $\mat{y}
    = \mat{X} \betab + \vepsb$, $\var{\vepsb} = \sigma^2\,
    \mat{W}^{-1}$ est
    \begin{displaymath}
      \betabh^* = (\mat{X}^\prime \mat{W} \mat{X})^{-1} \mat{X}^\prime
      \mat{W} \mat{y}
    \end{displaymath}
    et sa variance est, par le théorème \ref{thm:elements:esp_var},
    \begin{align*}
      \varmat{\betabh^*}
      &= (\mat{X}^\prime \mat{W} \mat{X})^{-1} \mat{X}^\prime \mat{W}
      \varmat{\mat{y}}
      \mat{W}^\prime \mat{X} (\mat{X}^\prime \mat{W} \mat{X})^{-1} \\
      &= \sigma^2
      (\mat{X}^\prime \mat{W} \mat{X})^{-1} \mat{X}^\prime \mat{W}
      \mat{W}^{-1}
      \mat{W} \mat{X} (\mat{X}^\prime \mat{W} \mat{X})^{-1} \\
      &= \sigma^2 (\mat{X}^\prime \mat{W} \mat{X})^{-1}
    \end{align*}
    puisque les matrices $\mat{W}$ et $\mat{X}^\prime \mat{W} \mat{X}$
    sont symétriques. Dans le cas de la régression linéaire simple
    passant par l'origine et en supposant que $\mat{W} = \diag(w_1,
    \dots, w_n)$, ces formules se réduisent en
    \begin{align*}
      \hat{\beta}^*
      &= \frac{\sum_{t=1}^n w_t X_t Y_t}{\sum_{t=1}^n w_t X_t^2} \\
      \intertext{et}
      \var{\hat{\beta}^*}
      &= \frac{\sigma^2}{\sum_{t=1}^n w_t X_t^2}.
    \end{align*}
    \begin{enumerate}
    \item Cas déjà traité à l'exercice
      \ref{chap:simple}.\ref{ex:simple:origine} où $\mat{W} = \mat{I}$
      et, donc,
      \begin{align*}
        \hat{\beta}^*
        &= \frac{\sum_{t=1}^n X_t Y_t}{\sum_{t=1}^n X_t^2} \\
        \intertext{et}
        \var{\hat{\beta}^*}
        &= \frac{\sigma^2}{\sum_{t=1}^n w_t X_t^2}.
      \end{align*}
    \item Cas général traité ci-dessus.
    \item Si $\var{\varepsilon_t} = \sigma^2 X_t$, alors $w_t =
      X_t^{-1}$. Le cas général se simplifie donc en
      \begin{align*}
        \hat{\beta}^*
        &= \frac{\sum_{t=1}^n Y_t}{\sum_{t=1}^n X_t} \\
        &= \frac{\bar{Y}}{\bar{X}}, \\
        \var{\hat{\beta}^*}
        &= \frac{\sigma^2}{\sum_{t=1}^n X_t} \\
        &= \frac{\sigma^2}{n \bar{X}}.
      \end{align*}
    \item Si $\var{\varepsilon_t} = \sigma^2 X_t^2$, alors $w_t =
      X_t^{-2}$. On a donc
      \begin{align*}
        \hat{\beta}^*
        &= \frac{1}{n} \sum_{t=1}^n \frac{Y_t}{X_t} \\
        \var{\hat{\beta}^*}
        &= \frac{\sigma^2}{n}.
      \end{align*}
    \end{enumerate}
  
\end{solution}
\begin{solution}{3.16}
    Le graphique des valeurs de $Y$ en fonction de celles de $X$, à la
    figure \ref{fig:multiple:quadratique}, montre clairement une
    relation quadratique. On postule donc le modèle
    \begin{displaymath}
      Y_t = \beta_0 + \beta_1 X_t + \beta_2 X_t^2 + \varepsilon_t, \quad
      \varepsilon_t \sim N(0, \sigma^2).
    \end{displaymath}
    \begin{figure}
      \centering
\begin{knitrout}
\definecolor{shadecolor}{rgb}{0.969, 0.969, 0.969}\color{fgcolor}\begin{kframe}
\begin{alltt}
\hlkwd{plot}\hlstd{(Y} \hlopt{~} \hlstd{X,} \hlkwc{data} \hlstd{= donnees)}
\end{alltt}
\end{kframe}

{\centering \includegraphics[width=.45\linewidth]{figure/fig-unnamed-chunk-39-1}

}



\end{knitrout}
      \caption{Graphique des données de l'exercice
        \ref{chap:multiple}.\ref{ex:multiple:quadratique}}
      \label{fig:multiple:quadratique}
    \end{figure}
    Par la suite, on peut estimer les paramètres de ce modèle avec la
    fonction \texttt{lm} de \textsf{R}:
\begin{knitrout}
\definecolor{shadecolor}{rgb}{0.969, 0.969, 0.969}\color{fgcolor}\begin{kframe}
\begin{alltt}
\hlstd{fit} \hlkwb{<-} \hlkwd{lm}\hlstd{(Y} \hlopt{~} \hlkwd{poly}\hlstd{(X,} \hlnum{2}\hlstd{),} \hlkwc{data} \hlstd{= donnees)}
\hlkwd{summary}\hlstd{(fit)}
\end{alltt}
\begin{verbatim}
##
## Call:
## lm(formula = Y ~ poly(X, 2), data = donnees)
##
## Residuals:
##     Min      1Q  Median      3Q     Max
## -1.9123 -0.6150 -0.1905  0.6367  1.6921
##
## Coefficients:
##             Estimate Std. Error t value Pr(>|t|)
## (Intercept)  18.1240     0.3025   59.91 3.10e-16 ***
## poly(X, 2)1  29.6754     1.1717   25.33 8.72e-12 ***
## poly(X, 2)2   4.0899     1.1717    3.49  0.00446 **
## ---
## Signif. codes:
## 0 '***' 0.001 '**' 0.01 '*' 0.05 '.' 0.1 ' ' 1
##
## Residual standard error: 1.172 on 12 degrees of freedom
## Multiple R-squared:  0.982,	Adjusted R-squared:  0.979
## F-statistic: 326.8 on 2 and 12 DF,  p-value: 3.434e-11
\end{verbatim}
\begin{alltt}
\hlkwd{anova}\hlstd{(fit)}
\end{alltt}
\begin{verbatim}
## Analysis of Variance Table
##
## Response: Y
##            Df Sum Sq Mean Sq F value    Pr(>F)
## poly(X, 2)  2 897.36  448.68  326.79 3.434e-11 ***
## Residuals  12  16.48    1.37
## ---
## Signif. codes:
## 0 '***' 0.001 '**' 0.01 '*' 0.05 '.' 0.1 ' ' 1
\end{verbatim}
\end{kframe}
\end{knitrout}
    Tant le test $F$ global que les tests $t$ individuels sont
    concluants, le coefficient de détermination est élevé et l'on peut
    constater à la figure \ref{fig:multiple:quadratique2} que
    l'ajustement du modèle est bon. On conclut donc qu'un modèle
    adéquat pour cet ensemble de données est
    \begin{displaymath}
      Y_t = 18,12 +
      29,68 X_t +
      4,09 X_t^2 + \varepsilon_t, \quad
      \varepsilon_t \sim N(0, 1,373).
    \end{displaymath}
    \begin{figure}
      \centering
\begin{knitrout}
\definecolor{shadecolor}{rgb}{0.969, 0.969, 0.969}\color{fgcolor}\begin{kframe}
\begin{alltt}
\hlkwd{plot}\hlstd{(Y} \hlopt{~} \hlstd{X,} \hlkwc{data} \hlstd{= donnees)}
\hlstd{x} \hlkwb{<-} \hlkwd{seq}\hlstd{(}\hlkwd{min}\hlstd{(donnees}\hlopt{$}\hlstd{X),} \hlkwd{max}\hlstd{(donnees}\hlopt{$}\hlstd{X),} \hlkwc{length} \hlstd{=} \hlnum{200}\hlstd{)}
\hlkwd{lines}\hlstd{(x,} \hlkwd{predict}\hlstd{(fit,} \hlkwd{data.frame}\hlstd{(}\hlkwc{X} \hlstd{= x),} \hlkwc{lwd} \hlstd{=} \hlnum{2}\hlstd{))}
\end{alltt}
\end{kframe}

{\centering \includegraphics[width=.45\linewidth]{figure/fig-unnamed-chunk-41-1}

}



\end{knitrout}
      \caption{Graphique des données de l'exercice
        \ref{chap:multiple}.\ref{ex:multiple:quadratique} et courbe
        obtenue par régression}
      \label{fig:multiple:quadratique2}
    \end{figure}
  
\end{solution}
\begin{solution}{3.17}
    Comme on peut le constater à la figure \ref{fig:multiple:pondere},
    le point $(X_{16}, Y_{16})$ est plus éloigné des autres. En b) et
    c), on diminue son poids dans la régression.
    \begin{figure}
      \centering
\begin{knitrout}
\definecolor{shadecolor}{rgb}{0.969, 0.969, 0.969}\color{fgcolor}\begin{kframe}
\begin{alltt}
\hlkwd{plot}\hlstd{(Y} \hlopt{~} \hlstd{X,} \hlkwc{data} \hlstd{= donnees)}
\hlkwd{points}\hlstd{(donnees}\hlopt{$}\hlstd{X[}\hlnum{16}\hlstd{], donnees}\hlopt{$}\hlstd{Y[}\hlnum{16}\hlstd{],} \hlkwc{pch} \hlstd{=} \hlnum{16}\hlstd{)}
\end{alltt}
\end{kframe}

{\centering \includegraphics[width=.45\linewidth]{figure/fig-unnamed-chunk-43-1}

}



\end{knitrout}
      \caption{Graphique des données de l'exercice
        \ref{chap:multiple}.\ref{ex:multiple:pondere}. Le cercle plein
        représente la donnée $(X_{16}, Y_{16})$.}
      \label{fig:multiple:pondere}
    \end{figure}
    \begin{enumerate}
    \item On calcule d'abord l'estimateur des moindres carrés ordinaires:
\begin{knitrout}
\definecolor{shadecolor}{rgb}{0.969, 0.969, 0.969}\color{fgcolor}\begin{kframe}
\begin{alltt}
\hlstd{(fit1} \hlkwb{<-} \hlkwd{lm}\hlstd{(Y} \hlopt{~} \hlstd{X,} \hlkwc{data} \hlstd{= donnees))}
\end{alltt}
\begin{verbatim}
##
## Call:
## lm(formula = Y ~ X, data = donnees)
##
## Coefficients:
## (Intercept)            X
##      1.4256       0.3158
\end{verbatim}
\end{kframe}
\end{knitrout}
    \item Si l'on suppose que la variance de la données $(X_{16},
      Y_{16})$ est quatre fois plus élevée que la variance des autres
      données, alors il convient d'accorder un point quatre fois moins
      grand à cette donnée dans la régression. Cela requiert les
      moindres carrés pondérés. Pour calculer les estimateurs avec
      \texttt{lm} dans \textsf{R}, on utilise l'argument
      \texttt{weights}:
\begin{knitrout}
\definecolor{shadecolor}{rgb}{0.969, 0.969, 0.969}\color{fgcolor}\begin{kframe}
\begin{alltt}
\hlstd{w} \hlkwb{<-} \hlkwd{rep}\hlstd{(}\hlnum{1}\hlstd{,} \hlkwd{nrow}\hlstd{(donnees))}
\hlstd{w[}\hlnum{16}\hlstd{]} \hlkwb{<-} \hlnum{0.25}
\hlstd{(fit2} \hlkwb{<-} \hlkwd{update}\hlstd{(fit1,} \hlkwc{weights} \hlstd{= w))}
\end{alltt}
\begin{verbatim}
##
## Call:
## lm(formula = Y ~ X, data = donnees, weights = w)
##
## Coefficients:
## (Intercept)            X
##      1.7213       0.2243
\end{verbatim}
\end{kframe}
\end{knitrout}
    \item On répète la procédure en b) avec un poids de encore plus
      petit pour la donnée $(X_{16}, Y_{16})$:
\begin{knitrout}
\definecolor{shadecolor}{rgb}{0.969, 0.969, 0.969}\color{fgcolor}\begin{kframe}
\begin{alltt}
\hlstd{w[}\hlnum{16}\hlstd{]} \hlkwb{<-} \hlnum{0.0625}
\hlstd{(fit3} \hlkwb{<-} \hlkwd{update}\hlstd{(fit1,} \hlkwc{weights} \hlstd{= w))}
\end{alltt}
\begin{verbatim}
##
## Call:
## lm(formula = Y ~ X, data = donnees, weights = w)
##
## Coefficients:
## (Intercept)            X
##      1.8080       0.1975
\end{verbatim}
\end{kframe}
\end{knitrout}
      Plus le poids accordé à la donnée $(X_{16}, Y_{16})$ est faible,
      moins la droite de régression est attirée vers ce point (voir la
      figure \ref{fig:multiple:pondere2}).
      \begin{figure}[t]
        \centering
\begin{knitrout}
\definecolor{shadecolor}{rgb}{0.969, 0.969, 0.969}\color{fgcolor}\begin{kframe}
\begin{alltt}
\hlkwd{plot}\hlstd{(Y} \hlopt{~} \hlstd{X,} \hlkwc{data} \hlstd{= donnees)}
\hlkwd{points}\hlstd{(donnees}\hlopt{$}\hlstd{X[}\hlnum{16}\hlstd{], donnees}\hlopt{$}\hlstd{Y[}\hlnum{16}\hlstd{],} \hlkwc{pch} \hlstd{=} \hlnum{16}\hlstd{)}
\hlkwd{abline}\hlstd{(fit1,} \hlkwc{lwd} \hlstd{=} \hlnum{2}\hlstd{,} \hlkwc{lty} \hlstd{=} \hlnum{1}\hlstd{)}
\hlkwd{abline}\hlstd{(fit2,} \hlkwc{lwd} \hlstd{=} \hlnum{2}\hlstd{,} \hlkwc{lty} \hlstd{=} \hlnum{2}\hlstd{)}
\hlkwd{abline}\hlstd{(fit3,} \hlkwc{lwd} \hlstd{=} \hlnum{2}\hlstd{,} \hlkwc{lty} \hlstd{=} \hlnum{3}\hlstd{)}
\hlkwd{legend}\hlstd{(}\hlnum{1.2}\hlstd{,} \hlnum{6}\hlstd{,} \hlkwc{legend} \hlstd{=} \hlkwd{c}\hlstd{(}\hlstr{"Modèle a)"}\hlstd{,} \hlstr{"Modèle b)"}\hlstd{,} \hlstr{"Modèle c)"}\hlstd{),}
       \hlkwc{lwd} \hlstd{=} \hlnum{2}\hlstd{,} \hlkwc{lty} \hlstd{=} \hlnum{1}\hlopt{:}\hlnum{3}\hlstd{)}
\end{alltt}
\end{kframe}

{\centering \includegraphics[width=.45\linewidth]{figure/fig-unnamed-chunk-47-1}

}



\end{knitrout}
        \caption{Graphique des données de l'exercice
          \ref{chap:multiple}.\ref{ex:multiple:pondere} avec les
          droites de régression obtenues à l'aide des moindres carrés
          pondérés.}
        \label{fig:multiple:pondere2}
      \end{figure}
    \end{enumerate}
  
\end{solution}
\begin{solution}{3.18}
    \begin{enumerate}
    \item Voir la figure \ref{fig:multiple:taxi} pour le graphique. Il
      y a effectivement une différence entre la consommation de
      carburant des hommes et des femmes: ces dernières font plus de
      milles avec un gallon d'essence.
      \begin{figure}
        \centering
\begin{knitrout}
\definecolor{shadecolor}{rgb}{0.969, 0.969, 0.969}\color{fgcolor}\begin{kframe}
\begin{alltt}
\hlstd{hommes} \hlkwb{<-} \hlkwd{subset}\hlstd{(donnees, sexe} \hlopt{==} \hlstr{"M"}\hlstd{)}
\hlstd{femmes} \hlkwb{<-} \hlkwd{subset}\hlstd{(donnees, sexe} \hlopt{==} \hlstr{"F"}\hlstd{)}
\hlkwd{plot}\hlstd{(mpg} \hlopt{~} \hlstd{age,} \hlkwc{data} \hlstd{= hommes,}
     \hlkwc{xlim} \hlstd{=} \hlkwd{range}\hlstd{(donnees}\hlopt{$}\hlstd{age),} \hlkwc{ylim} \hlstd{=} \hlkwd{range}\hlstd{(donnees}\hlopt{$}\hlstd{mpg))}
\hlkwd{points}\hlstd{(mpg} \hlopt{~} \hlstd{age,} \hlkwc{data} \hlstd{= femmes,} \hlkwc{pch} \hlstd{=} \hlnum{16}\hlstd{)}
\hlkwd{legend}\hlstd{(}\hlnum{4}\hlstd{,} \hlnum{16}\hlstd{,} \hlkwc{legend} \hlstd{=} \hlkwd{c}\hlstd{(}\hlstr{"Hommes"}\hlstd{,} \hlstr{"Femmes"}\hlstd{),} \hlkwc{pch} \hlstd{=} \hlkwd{c}\hlstd{(}\hlnum{1}\hlstd{,} \hlnum{16}\hlstd{))}
\end{alltt}
\end{kframe}

{\centering \includegraphics[width=.45\linewidth]{figure/fig-unnamed-chunk-49-1}

}



\end{knitrout}
        \caption{Graphique des données de l'exercice
          \ref{chap:multiple}.\ref{ex:multiple:taxi}}
        \label{fig:multiple:taxi}
      \end{figure}
    \item Remarquer que la variable \texttt{sexe} est un facteur et peut
      être utilisée telle quelle dans \texttt{lm}:
\begin{knitrout}
\definecolor{shadecolor}{rgb}{0.969, 0.969, 0.969}\color{fgcolor}\begin{kframe}
\begin{alltt}
\hlstd{(fit} \hlkwb{<-} \hlkwd{lm}\hlstd{(mpg} \hlopt{~} \hlstd{age} \hlopt{+} \hlstd{sexe,} \hlkwc{data} \hlstd{= donnees))}
\end{alltt}
\begin{verbatim}
##
## Call:
## lm(formula = mpg ~ age + sexe, data = donnees)
##
## Coefficients:
## (Intercept)          age        sexeM
##      16.687       -1.040       -1.206
\end{verbatim}
\end{kframe}
\end{knitrout}
    \item Calcul d'une prévision pour la valeur moyenne de la variable
      \texttt{mpg}:
\begin{knitrout}
\definecolor{shadecolor}{rgb}{0.969, 0.969, 0.969}\color{fgcolor}\begin{kframe}
\begin{alltt}
\hlkwd{predict}\hlstd{(fit,} \hlkwc{newdata} \hlstd{=} \hlkwd{data.frame}\hlstd{(}\hlkwc{age} \hlstd{=} \hlnum{4}\hlstd{,} \hlkwc{sexe} \hlstd{=} \hlstr{"F"}\hlstd{),}
        \hlkwc{interval} \hlstd{=} \hlstr{"confidence"}\hlstd{,} \hlkwc{level} \hlstd{=} \hlnum{0.90}\hlstd{)}
\end{alltt}
\begin{verbatim}
##        fit      lwr      upr
## 1 12.52876 11.94584 13.11168
\end{verbatim}
\end{kframe}
\end{knitrout}
    \end{enumerate}
  
\end{solution}
\begin{solution}{3.19}
\begin{enumerate}
\item Le postulat de normalité semble violé.

La distribution des résidus a une queue inférieure plus épaisse que la loi normale, ce que l'on voit à gauche du Q-Q plot, puisque les poits ne sont pas alignés.

Le postulat de normalité n'est pas critique, parce que les estimateurs des moindres carrés ont un sens quand même. Toutefois, les tests d'hypothèses et les intervalles de confiance ne sont pas valides.

\item Le graphique des résidus en fonction de $x_2$ montre que le postulat de linéarité semble violé. Cela implique que le modèle n'est pas valide.

On observe de l'hétéroscédasticité (par exemple, dans les graphiques 1, 3 ou 4) puisque les résidus ne semblent pas avoir une variance constante.

Cela signifie que les variances des paramètres ne sont pas calculées de façon appropriée OU il faudrait effectuer une transformation sur les variables pour régler ces problèmes.


\end{enumerate}
\end{solution}
\begin{solution}{3.20}
On pourrait croire qu'un point sur 20, ça ne change rien, mais ce n'est pas le cas! Le point 1 a un impact sur la pente et la qualité de l'ajustement. Le point 2 a un grand levier mais n'affecte pas beaucoup les estimations, le point 3 a un grand levier et un gros impact.

\begin{verbatim}
dat <- read.csv("OutlierExample.csv")

dim(dat)

summary(dat)

library(ggplot2)

ggplot(dat, aes(x= X, y= Y, label=CODES))+
  geom_point() +
  geom_text(aes(label=ifelse(CODES>0,CODES,'')),hjust=0,vjust=0)

fit0 <- lm(Y~X,dat,subset=(CODES==0))
summary(fit0)
plot(dat[,1:2],pch=16)
points(dat[match(1:3,dat$CODES),1:2],col=2:4,pch=16:18,cex=1.2)
abline(fit0)

fit1 <- lm(Y~X,dat,subset=(CODES<=1))
summary(fit1)
abline(fit1,col=2,lty=2)

fit2 <- lm(Y~X,dat,subset=(CODES%in%c(0,2)))
summary(fit2)
abline(fit2,col=3,lty=3)

fit3 <- lm(Y~X,dat,subset=(CODES%in%c(0,3)))
summary(fit3)
abline(fit3,col=4,lty=4)

influence.measures(fit0)
influence.measures(fit1)
influence.measures(fit2)
influence.measures(fit3)

\end{verbatim}
\end{solution}

\newpage
\section*{Chapitre \ref{chap:selection}}
\addcontentsline{toc}{section}{Chapitre \protect\ref{chap:selection}}

\begin{solution}{2.1}
\begin{enumerate}
\item
\begin{enumerate}
\item modèle D
\item modèle D
\item modèle G
\item modèle G
\item modèle C
\item modèle H
\end{enumerate}

\item Il y a un très gros problème de multicolinéarité pour les modèle F, G et H, car certains VIFs sont beaucoup plus grands que 10. Ce problème augmente inutilement la variance des paramètres estimés.

\item On évite les modèles F G et H pour ne pas avoir de problème de multicolinéarité. Le modèle D est préférable selon les critères PRESS et $R^2_p$. De plus, ses critères AIC et BIC sont les deuxièmes plus petits. Le $C_p$ est 8, donc 8-5=3. Ce n'est pas parfait, mais ce n'est pas si mal, etc.

\end{enumerate}
\end{solution}
\begin{solution}{2.2}
\begin{enumerate}
\item Puisque $n=p$, $\beta_0=0$ et que la matrice d'incidence est diagonale, on a $\hat{y}_i = \hat\beta_i$ pour $i=1,\ldots,n$. On minimise $S(\bobeta)=\sum_{i=1}^n (y_i-\beta_i)^2$ et on trouve pour $i\in \{1,\ldots,n\}$,
$$
\left.\frac{\partial}{\partial\beta_i}S(\bobeta)\right|_{\hat\beta_i}= -2 (y_i-\hat\beta_i) =0 \quad \Rightarrow \quad \hat\beta_i =  y_i.
$$

\item On minimise, pour une valeur $\lambda>0$,
$$
S^{\mathrm{ridge}}(\bobeta)=\sum_{i=1}^n (y_i-\beta_i)^2 +\lambda \sum_{i=1}^n \beta_i^2.
$$

\item On a
$$
\frac{\partial}{\partial\beta_i}S^{\mathrm{ridge}}(\bobeta)=-2(y_i-\beta_i) +2 \lambda  \beta_i.
$$
On pose égal à 0 et on trouve
$$
y_i-\hat \beta_i^{\mathrm{ridge}} = \lambda  \hat\beta_i^{\mathrm{ridge}} \quad \Rightarrow \quad \hat\beta_i^{\mathrm{ridge}} =  \frac{y_i}{1+\lambda}.
$$

\item On minimise, pour une valeur $\lambda>0$,
$$
S^{\mathrm{lasso}}(\bobeta)=\sum_{i=1}^n (y_i-\beta_i)^2 +\lambda \sum_{i=1}^n |\beta_i|.
$$

\item On a
$$
\frac{\partial}{\partial\beta_i}S^{\mathrm{lasso}}(\bobeta)=-2(y_i-\beta_i) + \lambda\, \mathrm{signe}(\beta_i) .
$$
On utilise les EMV trouvés en a) pour définir le signe. Supposons d'abord que $\hat\beta_i = y_i>0$. Alors, on a aussi $\hat\beta_i^{\mathrm{lasso}}>0$ (sinon, changer le signe donnera une valeur plus petite de l'équation à minimiser). On pose la dérivée égale à 0 et on trouve
$$
2(y_i-\hat \beta_i^{\mathrm{lasso}}) = \lambda  \quad \Rightarrow \quad \hat\beta_i^{\mathrm{ridge}} =  y_i- \lambda/2,
$$
ce qui tient seulement si $\hat\beta_i^{\mathrm{lasso}}>0$, alors on a $\hat\beta_i^{\mathrm{ridge}} =  \max(0,y_i- \lambda/2)$.
Supposons ensuite que $\hat \beta_i = y_i <0$. Alors, on a aussi $\hat\beta_i^{\mathrm{lasso}}<0$. On pose la dérivée égale à 0 et on trouve
$$
2(y_i-\hat \beta_i^{\mathrm{lasso}}) = -\lambda  \quad \Rightarrow \quad \hat\beta_i^{\mathrm{ridge}} =  y_i+ \lambda/2,
$$
sous la contrainte que ce soit négatif, donc dans ce cas, $\hat\beta_i^{\mathrm{ridge}} =  \min(0,y_i+\lambda/2)$. On combine les deux cas et on obtient l'équation donnée.

\item On peut voir que la façon de rapetisser les paramètres est bien différente pour les deux méthodes. Avec ridge, chaque coefficient des moindres carrés est réduit par la même proportion. Avec lasso, chaque coefficient des moindres carrés est réduit vers 0 d'un montant constant $\lambda/2$; ceux qui sont plus petits que $\lambda/2$ en valeur absolue sont mis exactement égaux à 0. C'est de cette façon que le lasso permet de faire la sélection des variables explicatives.

\end{enumerate}
\end{solution}

\newpage
\section*{Chapitre \ref{chap:glm}}
\addcontentsline{toc}{section}{Chapitre \protect\ref{chap:glm}}

\begin{solution}{2.1}

\end{solution}
\begin{solution}{2.2}

\end{solution}
\begin{solution}{2.3}

\end{solution}
\begin{solution}{2.4}

\end{solution}
\begin{solution}{2.5}

\end{solution}
\begin{solution}{2.6}

\end{solution}
\begin{solution}{2.7}

\end{solution}
\begin{solution}{2.8}

\end{solution}
\begin{solution}{2.9}

\end{solution}

\newpage
\section*{Chapitre \ref{chap:comptage}}
\addcontentsline{toc}{section}{Chapitre \protect\ref{chap:comptage}}

\begin{solution}{6.1}
On ajuste d'abord le modèle avec les effets principaux et les interactions.
\begin{knitrout}
\definecolor{shadecolor}{rgb}{0.969, 0.969, 0.969}\color{fgcolor}\begin{kframe}
\begin{alltt}
\hlkwd{library}\hlstd{(datasets)}
\hlstd{fit1} \hlkwb{<-} \hlkwd{glm}\hlstd{(ncases}\hlopt{~}\hlkwd{factor}\hlstd{(agegp)}\hlopt{*}\hlstd{(}\hlkwd{factor}\hlstd{(alcgp)}\hlopt{+}\hlkwd{factor}\hlstd{(tobgp))}\hlopt{+}\hlkwd{factor}\hlstd{(alcgp)}\hlopt{:}\hlkwd{factor}\hlstd{(tobgp),}\hlkwc{family}\hlstd{=poisson,}\hlkwc{data}\hlstd{=esoph)}
\end{alltt}


{\ttfamily\noindent\color{warningcolor}{\#\# Warning: glm.fit: fitted rates numerically 0 occurred}}\begin{alltt}
\hlkwd{anova}\hlstd{(fit1)}
\end{alltt}


{\ttfamily\noindent\color{warningcolor}{\#\# Warning: glm.fit: fitted rates numerically 0 occurred}}\begin{verbatim}
## Analysis of Deviance Table
##
## Model: poisson, link: log
##
## Response: ncases
##
## Terms added sequentially (first to last)
##
##
##                             Df Deviance Resid. Df
## NULL                                           87
## factor(agegp)                5  138.256        82
## factor(alcgp)                3   24.106        79
## factor(tobgp)                3   22.169        76
## factor(agegp):factor(alcgp) 15   32.417        61
## factor(agegp):factor(tobgp) 15   18.109        46
## factor(alcgp):factor(tobgp)  9    7.658        37
##                             Resid. Dev
## NULL                           262.926
## factor(agegp)                  124.670
## factor(alcgp)                  100.564
## factor(tobgp)                   78.395
## factor(agegp):factor(alcgp)     45.979
## factor(agegp):factor(tobgp)     27.870
## factor(alcgp):factor(tobgp)     20.212
\end{verbatim}
\begin{alltt}
\hlkwd{qchisq}\hlstd{(}\hlnum{0.95}\hlstd{,}\hlnum{9}\hlstd{)} \hlcom{## rejette alcgp:tobgp}
\end{alltt}
\begin{verbatim}
## [1] 16.91898
\end{verbatim}
\begin{alltt}
\hlkwd{qchisq}\hlstd{(}\hlnum{0.95}\hlstd{,}\hlnum{15}\hlstd{)} \hlcom{## rejette agegp:tobgp mais conserve agegp:alcgp}
\end{alltt}
\begin{verbatim}
## [1] 24.99579
\end{verbatim}
\end{kframe}
\end{knitrout}
On trouve donc que l'interaction entre la consommation d'alcool et de tabac n'est pas significative parce que
\begin{align*}
\Delta Deviance = 7.658 < \chi^{2}_{(9;0.95)}=16.92.
\end{align*}
Cela signifie que le modèle \texttt{agegp*(alcgp+tobgp)} est une simplification adéquate du modèle
\texttt{agegp+alcgp+tobgp+agegp.alcgp+agegp.tobgp+alcgp.tobgp}. De plus, on peut enlever l'interaction entre l'âge et la consommation de tabac:
\begin{align*}
\Delta Deviance = 18.109 < \chi^{2}_{(15;0.95)}=25.
\end{align*}
Cela signifie que le modèle \texttt{agegp*alcgp+tobgp} est une simplification adéquate du modèle
\texttt{agegp*(alcgp+tobgp)}. Toutefois, on ne peut pas enlever l'autre terme d'interaction car
\begin{align*}
\Delta Deviance = 32.417 > \chi^{2}_{(15;0.95)}=25.
\end{align*}
Si on tente de remettre l'interaction entre la consommation d'alcool et de tabac dans le modèle, on trouve qu'elle n'est toujours pas significative:
\begin{knitrout}
\definecolor{shadecolor}{rgb}{0.969, 0.969, 0.969}\color{fgcolor}\begin{kframe}
\begin{alltt}
\hlstd{fit2} \hlkwb{<-} \hlkwd{glm}\hlstd{(ncases} \hlopt{~} \hlstd{agegp} \hlopt{*} \hlstd{alcgp} \hlopt{+} \hlstd{tobgp,}\hlkwc{family}\hlstd{=poisson,}\hlkwc{data}\hlstd{=esoph)}
\hlstd{fit3} \hlkwb{<-} \hlkwd{update}\hlstd{(fit1,}\hlopt{~}\hlstd{.}\hlopt{-}\hlkwd{factor}\hlstd{(agegp)}\hlopt{:}\hlkwd{factor}\hlstd{(tobgp))}
\hlkwd{anova}\hlstd{(fit2,fit3)}
\end{alltt}
\begin{verbatim}
## Analysis of Deviance Table
##
## Model 1: ncases ~ agegp * alcgp + tobgp
## Model 2: ncases ~ factor(agegp) + factor(alcgp) + factor(tobgp) + factor(agegp):factor(alcgp) +
##     factor(alcgp):factor(tobgp)
##   Resid. Df Resid. Dev Df Deviance
## 1        61     45.979
## 2        52     38.973  9    7.006
\end{verbatim}
\end{kframe}
\end{knitrout}
Par conséquent, le modèle final est \texttt{agegp*alcgp+tobgp}. Cela signifie que l'effet de consommer de l'alcool sur l'occurence du cancer de l'oesophage est différent pour chaque groupe d'âge.
\end{solution}
\begin{solution}{6.2}
Intégrer la densité conditionnelle Poisson sur $z$. Comme c'est plus agréable à faire à la main qu'à taper, je vous laisse le soin de réussir par vous-même.
\end{solution}
\begin{solution}{6.3}
On note $n=n_A+n_B$. La vraisemblance pour ce GLM est
\begin{align*}
\mathcal{L}(\beta_0,\beta_1)&=\prod_{i=1}^n \exp(y_i\log(\mu_i)-\mu_i+\mbox{cte})\\
&=\prod_{i=1}^n \exp(y_i\log(g^{-1}(\eta_i))-g^{-1}(\eta_i)+\mbox{cte}).
\end{align*}
La log-vraisemblance est donc:
\begin{align*}
\ell(\beta_0,\beta_1)&=\sum_{i=1}^n (y_i\log(g^{-1}(\eta_i))-g^{-1}(\eta_i)+\mbox{cte}).
\end{align*}
On dérive par rapport à $\beta_0$ et $\beta_1$:
\begin{align*}
\frac{\partial \ell(\beta_0,\beta_1)}{\partial\beta_0}&=\sum_{i=1}^n (y_i\frac{1}{g^{-1}(\eta_i)g'(g^{-1}(\eta_i))}-\frac{1}{g'(g^{-1}(\eta_i))})\\
&=\sum_{i=1}^n \frac{(y_i-g^{-1}(\eta_i))}{g^{-1}(\eta_i)g'(g^{-1}(\eta_i))}\\
\frac{\partial \ell(\beta_0,\beta_1)}{\partial\beta_1}&=\sum_{i=1}^n (y_i\frac{x_i}{g^{-1}(\eta_i)g'(g^{-1}(\eta_i))}-\frac{x_i}{g'(g^{-1}(\eta_i))})\\
&=\sum_{i=1}^n \frac{ x_i(y_i-g^{-1}(\eta_i))}{g^{-1}(\eta_i)g'(g^{-1}(\eta_i))}.
\end{align*}
On égalise à 0 pour obtenir le système d'équations à résoudre.
\begin{align*}
0&=\sum_{i=1}^n \frac{(y_i-g^{-1}(\hat{\eta}_i))}{g^{-1}(\hat{\eta}_i)g'(g^{-1}(\hat{\eta}_i))}\\
0&=\sum_{i=1}^n \frac{ x_i(y_i-g^{-1}(\hat{\eta}_i))}{g^{-1}(\hat{\eta}_i)g'(g^{-1}(\hat{\eta}_i))}
\end{align*}
On utilise que $x_i=0 \forall i \in (n_A+1,...,n_A+n_B)$:
\begin{align*}
0&=\sum_{i=1}^{n_A+n_B} \frac{(y_i-g^{-1}(\hat{\eta}_i))}{g^{-1}(\hat{\eta}_i)g'(g^{-1}(\hat{\eta}_i))}\\
0&=\sum_{i=1}^{n_A} \frac{ (y_i-g^{-1}(\hat{\eta}_i))}{g^{-1}(\hat{\eta}_i)g'(g^{-1}(\hat{\eta}_i))}\\
\Rightarrow 0&= \sum_{i=n_A+1}^{n_A+n_B} \frac{ (y_i-g^{-1}(\hat{\eta}_i))}{g^{-1}(\hat{\eta}_i)g'(g^{-1}(\hat{\eta}_i))}.
\end{align*}
Aussi, $\forall i \in (1,...,n_A), \hat{\eta}_i=\hat{\beta}_0+\hat{\beta}_1$, ce qui ne dépend pas de $i$. Le dénominateur ne dépend pas de $i$ et peut sortir de la somme et s'annuler. De même, $\forall i \in (n_A+1,...,n_A+n_B), \hat{\eta}_i=\hat{\beta}_0$, ce qui ne dépend pas de $i$. Le dénominateur ne dépend pas de $i$ et peut sortir de la somme et s'annuler. On obtient donc les équations:
\begin{align*}
0&=\sum_{i=1}^{n_A} (y_i-g^{-1}(\hat{\eta}_i))\\
0&= \sum_{i=n_A+1}^{n_A+n_B}  (y_i-g^{-1}(\hat{\eta}_i)).
\end{align*}
Finalement, $g^{-1}(\hat{\eta}_i)=\hat{\mu}_i$ par définition. Alors
\begin{align*}
0&=\sum_{i=1}^{n_A} (y_i-\hat{\mu}_A) \Rightarrow \sum_{i=1}^{n_A}y_i =n_A\hat{\mu}_A \Rightarrow \frac{\sum_{i=1}^{n_A}y_i}{n_A} =\hat{\mu}_A\\
0&= \sum_{i=n_A+1}^{n_A+n_B}  (y_i-\hat{\mu}_B) \Rightarrow \sum_{i=n_A+1}^{n_A+n_B}y_i =n_B\hat{\mu}_B \Rightarrow \frac{\sum_{i=n_A+1}^{n_A+n_B}y_i}{n_B} =\hat{\mu}_B.
\end{align*}
\end{solution}
\begin{solution}{6.4}
\begin{enumerate}
\item Avec ce modèle, on a que
\begin{align*}
\mu_A &= \exp(\beta_0)\\
\mu_B &= \exp(\beta_0+\beta_1)=\mu_A \exp(\beta_1),
\end{align*} ce qui implique que $\exp(\beta_1)=\mu_B/\mu_A$. On ajuste le modèle en \texttt{R}, et on vérifie que cela est bien vrai:
\begin{knitrout}
\definecolor{shadecolor}{rgb}{0.969, 0.969, 0.969}\color{fgcolor}\begin{kframe}
\begin{alltt}
\hlstd{y} \hlkwb{<-} \hlkwd{c}\hlstd{(} \hlnum{8}\hlstd{,}\hlnum{7}\hlstd{,}\hlnum{6}\hlstd{,}\hlnum{6}\hlstd{,}\hlnum{3}\hlstd{,}\hlnum{4}\hlstd{,}\hlnum{7}\hlstd{,}\hlnum{2}\hlstd{,}\hlnum{3}\hlstd{,}\hlnum{4}\hlstd{,}\hlnum{9}\hlstd{,}\hlnum{9}\hlstd{,}\hlnum{8}\hlstd{,}\hlnum{14}\hlstd{,}\hlnum{8}\hlstd{,}\hlnum{13}\hlstd{,}\hlnum{11}\hlstd{,}\hlnum{5}\hlstd{,}\hlnum{7}\hlstd{,}\hlnum{6}\hlstd{)}
\hlstd{x} \hlkwb{<-} \hlkwd{rep}\hlstd{(}\hlnum{0}\hlopt{:}\hlnum{1}\hlstd{,}\hlkwc{each}\hlstd{=}\hlnum{10}\hlstd{)}
\hlstd{fit1} \hlkwb{<-} \hlkwd{glm}\hlstd{(y}\hlopt{~}\hlstd{x,}\hlkwc{family}\hlstd{=poisson)}
\hlkwd{summary}\hlstd{(fit1)}
\end{alltt}
\begin{verbatim}
##
## Call:
## glm(formula = y ~ x, family = poisson)
##
## Deviance Residuals:
##     Min       1Q   Median       3Q      Max
## -1.5280  -0.7622  -0.1699   0.6938   1.5399
##
## Coefficients:
##             Estimate Std. Error z value Pr(>|z|)
## (Intercept)   1.6094     0.1414  11.380  < 2e-16 ***
## x             0.5878     0.1764   3.332 0.000861 ***
## ---
## Signif. codes:
## 0 '***' 0.001 '**' 0.01 '*' 0.05 '.' 0.1 ' ' 1
##
## (Dispersion parameter for poisson family taken to be 1)
##
##     Null deviance: 27.857  on 19  degrees of freedom
## Residual deviance: 16.268  on 18  degrees of freedom
## AIC: 94.349
##
## Number of Fisher Scoring iterations: 4
\end{verbatim}
\begin{alltt}
\hlkwd{log}\hlstd{(}\hlkwd{mean}\hlstd{(y[}\hlkwd{which}\hlstd{(x}\hlopt{==}\hlnum{1}\hlstd{)])}\hlopt{/}\hlkwd{mean}\hlstd{(y[}\hlkwd{which}\hlstd{(x}\hlopt{==}\hlnum{0}\hlstd{)]))}
\end{alltt}
\begin{verbatim}
## [1] 0.5877867
\end{verbatim}
\end{kframe}
\end{knitrout}

\item Puisque $\exp(\beta_1)=\mu_B/\mu_A$, alors si $H_0: \mu_A=\mu_B$ est vraie, $\beta_1=0$. On peut utiliser la statistique de Wald directement, on trouve que le seuil observé du test est 0.000861. On rejette donc l'hypothèse nulle à un niveau de confiance de 99\%, ce qui implique que les moyennes diffèrent de façon significative.

\item Un I.C. à 95\% pour $\beta_1$ est
\begin{knitrout}
\definecolor{shadecolor}{rgb}{0.969, 0.969, 0.969}\color{fgcolor}\begin{kframe}
\begin{alltt}
\hlstd{fit1}\hlopt{$}\hlstd{coef[}\hlnum{2}\hlstd{]}\hlopt{+}\hlkwd{c}\hlstd{(}\hlopt{-}\hlnum{1}\hlstd{,}\hlnum{1}\hlstd{)}\hlopt{*}\hlkwd{qnorm}\hlstd{(}\hlnum{0.975}\hlstd{)}\hlopt{*}\hlkwd{summary}\hlstd{(fit1)}\hlopt{$}\hlstd{coefficients[}\hlnum{2}\hlstd{,}\hlnum{2}\hlstd{]}
\end{alltt}
\begin{verbatim}
## [1] 0.2420820 0.9334913
\end{verbatim}
\end{kframe}
\end{knitrout}

Alors, un I.C. pour $\mu_B/\mu_A$ est $(\exp(0.2421),\exp(0.93349))=(1.273899 , 2.543373)$.

\item Il n'y a pas d'indications de surdispersion, puisque la déviance est 16.26 sur 18 degrés de liberté, et $16.26/18<1$.

\item Quand on ajuste une binomiale négative à ces données, on trouve que $\theta_z$ tend vers l'infini, donc le modèle Poisson est une simplification adéquate du modèle NB. En fait, les estimations des paramètres $\beta_0$ et $\beta_1$ sont exactement les mêmes que celles obtenues dans le modèle Poisson.
\begin{knitrout}
\definecolor{shadecolor}{rgb}{0.969, 0.969, 0.969}\color{fgcolor}\begin{kframe}
\begin{alltt}
\hlkwd{library}\hlstd{(MASS)}
\hlstd{fit2} \hlkwb{<-} \hlkwd{glm.nb}\hlstd{(y}\hlopt{~}\hlstd{x)}
\end{alltt}


{\ttfamily\noindent\color{warningcolor}{\#\# Warning in theta.ml(Y, mu, sum(w), w, limit = control\$maxit, trace = control\$trace > : iteration limit reached}}

{\ttfamily\noindent\color{warningcolor}{\#\# Warning in theta.ml(Y, mu, sum(w), w, limit = control\$maxit, trace = control\$trace > : iteration limit reached}}\begin{alltt}
\hlkwd{summary}\hlstd{(fit2)}
\end{alltt}
\begin{verbatim}
##
## Call:
## glm.nb(formula = y ~ x, init.theta = 113420.3107, link = log)
##
## Deviance Residuals:
##     Min       1Q   Median       3Q      Max
## -1.5280  -0.7622  -0.1699   0.6937   1.5398
##
## Coefficients:
##             Estimate Std. Error z value Pr(>|z|)
## (Intercept)   1.6094     0.1414  11.380  < 2e-16 ***
## x             0.5878     0.1764   3.332 0.000861 ***
## ---
## Signif. codes:
## 0 '***' 0.001 '**' 0.01 '*' 0.05 '.' 0.1 ' ' 1
##
## (Dispersion parameter for Negative Binomial(113420.3) family taken to be 1)
##
##     Null deviance: 27.855  on 19  degrees of freedom
## Residual deviance: 16.267  on 18  degrees of freedom
## AIC: 96.349
##
## Number of Fisher Scoring iterations: 1
##
##
##               Theta:  113420
##           Std. Err.:  4076965
## Warning while fitting theta: iteration limit reached
##
##  2 x log-likelihood:  -90.349
\end{verbatim}
\end{kframe}
\end{knitrout}

\item Dans ce cas, on remarque que, bien que l'estimation du paramètre est égale pour les deux modèles, l'écart-type diffère. Aussi, le modèle de Poisson ne semble plus adéquat, car $Deviance/dl=27.857/19>1$, alors que le modèle NB s'ajuste bien aux données. Cela montre que lorsqu'une variable explicative importante n'est pas observée, le modèle de Poisson peut perdre sa validité pour des données de comptage. La variable explicative manquante introduit de la sur-dispersion dans les données, ce qui est capturé efficacement avec la loi NB.
\begin{knitrout}
\definecolor{shadecolor}{rgb}{0.969, 0.969, 0.969}\color{fgcolor}\begin{kframe}
\begin{alltt}
\hlstd{fit3} \hlkwb{<-} \hlkwd{glm}\hlstd{(y}\hlopt{~}\hlnum{1}\hlstd{,}\hlkwc{family}\hlstd{=poisson)}
\hlstd{fit4} \hlkwb{<-} \hlkwd{glm.nb}\hlstd{(y}\hlopt{~}\hlnum{1}\hlstd{)}
\hlkwd{summary}\hlstd{(fit3)}
\end{alltt}
\begin{verbatim}
##
## Call:
## glm(formula = y ~ 1, family = poisson)
##
## Deviance Residuals:
##     Min       1Q   Median       3Q      Max
## -2.2336  -0.9063   0.0000   0.4580   2.3255
##
## Coefficients:
##             Estimate Std. Error z value Pr(>|z|)
## (Intercept)  1.94591    0.08451   23.02   <2e-16 ***
## ---
## Signif. codes:
## 0 '***' 0.001 '**' 0.01 '*' 0.05 '.' 0.1 ' ' 1
##
## (Dispersion parameter for poisson family taken to be 1)
##
##     Null deviance: 27.857  on 19  degrees of freedom
## Residual deviance: 27.857  on 19  degrees of freedom
## AIC: 103.94
##
## Number of Fisher Scoring iterations: 4
\end{verbatim}
\begin{alltt}
\hlkwd{summary}\hlstd{(fit4)}
\end{alltt}
\begin{verbatim}
##
## Call:
## glm.nb(formula = y ~ 1, init.theta = 18.2073559, link = log)
##
## Deviance Residuals:
##     Min       1Q   Median       3Q      Max
## -1.9810  -0.7836   0.0000   0.3859   1.9033
##
## Coefficients:
##             Estimate Std. Error z value Pr(>|z|)
## (Intercept)  1.94591    0.09944   19.57   <2e-16 ***
## ---
## Signif. codes:
## 0 '***' 0.001 '**' 0.01 '*' 0.05 '.' 0.1 ' ' 1
##
## (Dispersion parameter for Negative Binomial(18.2074) family taken to be 1)
##
##     Null deviance: 20.279  on 19  degrees of freedom
## Residual deviance: 20.279  on 19  degrees of freedom
## AIC: 104.77
##
## Number of Fisher Scoring iterations: 1
##
##
##               Theta:  18.2
##           Std. Err.:  21.0
##
##  2 x log-likelihood:  -100.767
\end{verbatim}
\begin{alltt}
\hlkwd{exp}\hlstd{(fit3}\hlopt{$}\hlstd{coef[}\hlnum{1}\hlstd{]}\hlopt{+}\hlkwd{c}\hlstd{(}\hlopt{-}\hlnum{1}\hlstd{,}\hlnum{1}\hlstd{)}\hlopt{*}\hlkwd{qnorm}\hlstd{(}\hlnum{0.975}\hlstd{)}\hlopt{*}\hlkwd{summary}\hlstd{(fit3)}\hlopt{$}\hlstd{coefficients[}\hlnum{1}\hlstd{,}\hlnum{2}\hlstd{])}
\end{alltt}
\begin{verbatim}
## [1] 5.931421 8.261090
\end{verbatim}
\begin{alltt}
\hlkwd{exp}\hlstd{(fit4}\hlopt{$}\hlstd{coef[}\hlnum{1}\hlstd{]}\hlopt{+}\hlkwd{c}\hlstd{(}\hlopt{-}\hlnum{1}\hlstd{,}\hlnum{1}\hlstd{)}\hlopt{*}\hlkwd{qnorm}\hlstd{(}\hlnum{0.975}\hlstd{)}\hlopt{*}\hlkwd{summary}\hlstd{(fit4)}\hlopt{$}\hlstd{coefficients[}\hlnum{1}\hlstd{,}\hlnum{2}\hlstd{])}
\end{alltt}
\begin{verbatim}
## [1] 5.760386 8.506374
\end{verbatim}
\end{kframe}
\end{knitrout}

\end{enumerate}
\end{solution}
\begin{solution}{6.5}
\begin{enumerate}
\item On y va
\begin{knitrout}
\definecolor{shadecolor}{rgb}{0.969, 0.969, 0.969}\color{fgcolor}\begin{kframe}
\begin{alltt}
\hlstd{sex} \hlkwb{<-} \hlkwd{rep}\hlstd{(}\hlnum{0}\hlopt{:}\hlnum{1}\hlstd{,}\hlkwc{each}\hlstd{=}\hlnum{6}\hlstd{)}
\hlstd{Dep} \hlkwb{<-} \hlkwd{rep}\hlstd{(}\hlnum{0}\hlopt{:}\hlnum{5}\hlstd{,}\hlnum{2}\hlstd{)}
\hlstd{y} \hlkwb{<-} \hlkwd{c}\hlstd{(}\hlnum{512}\hlstd{,}\hlnum{353}\hlstd{,}\hlnum{120}\hlstd{,}\hlnum{138}\hlstd{,}\hlnum{53}\hlstd{,}\hlnum{22}\hlstd{,}\hlnum{89}\hlstd{,}\hlnum{17}\hlstd{,}\hlnum{202}\hlstd{,}\hlnum{131}\hlstd{,}\hlnum{94}\hlstd{,}\hlnum{24}\hlstd{)}
\hlstd{no} \hlkwb{<-} \hlkwd{c}\hlstd{(}\hlnum{313}\hlstd{,}\hlnum{207}\hlstd{,}\hlnum{205}\hlstd{,}\hlnum{279}\hlstd{,}\hlnum{138}\hlstd{,}\hlnum{351}\hlstd{,}\hlnum{19}\hlstd{,}\hlnum{8}\hlstd{,}\hlnum{391}\hlstd{,}\hlnum{244}\hlstd{,}\hlnum{299}\hlstd{,}\hlnum{317}\hlstd{)}
\hlstd{nb} \hlkwb{<-} \hlstd{y}\hlopt{+}\hlstd{no}
\hlstd{fitpSex} \hlkwb{<-} \hlkwd{glm}\hlstd{(y}\hlopt{~}\hlkwd{factor}\hlstd{(sex)}\hlopt{+}\hlkwd{offset}\hlstd{(}\hlkwd{log}\hlstd{(nb)),}\hlkwc{family}\hlstd{=poisson)}
\hlkwd{summary}\hlstd{(fitpSex)}
\end{alltt}
\begin{verbatim}
##
## Call:
## glm(formula = y ~ factor(sex) + offset(log(nb)), family = poisson)
##
## Deviance Residuals:
##      Min        1Q    Median        3Q       Max
## -14.1129   -3.6826   -0.2719    3.7437    8.0834
##
## Coefficients:
##              Estimate Std. Error z value Pr(>|z|)
## (Intercept)  -0.80926    0.02889 -28.011  < 2e-16 ***
## factor(sex)1 -0.38298    0.05128  -7.468 8.15e-14 ***
## ---
## Signif. codes:
## 0 '***' 0.001 '**' 0.01 '*' 0.05 '.' 0.1 ' ' 1
##
## (Dispersion parameter for poisson family taken to be 1)
##
##     Null deviance: 551.69  on 11  degrees of freedom
## Residual deviance: 493.56  on 10  degrees of freedom
## AIC: 573.76
##
## Number of Fisher Scoring iterations: 4
\end{verbatim}
\end{kframe}
\end{knitrout}

On trouve donc que la valeur-$p$ du test de Wald $H_0: \beta^{SEX}=0$ est $8.15\times10^{-14}$ ce qui est hautement significatif. Puisque le coefficient est négatif et que le niveau de base utilisé est ``hommes'', cela signifie que les femmes ont moins de chance d'être acceptées aux études graduées que les hommes.

\item On ajoute le département:
\begin{knitrout}
\definecolor{shadecolor}{rgb}{0.969, 0.969, 0.969}\color{fgcolor}\begin{kframe}
\begin{alltt}
\hlstd{fitp2} \hlkwb{<-} \hlkwd{glm}\hlstd{(y}\hlopt{~}\hlkwd{factor}\hlstd{(sex)}\hlopt{+}\hlkwd{factor}\hlstd{(Dep)}\hlopt{+}\hlkwd{offset}\hlstd{(}\hlkwd{log}\hlstd{(nb)),}\hlkwc{family}\hlstd{=poisson)}
\hlkwd{summary}\hlstd{(fitp2)}
\end{alltt}
\begin{verbatim}
##
## Call:
## glm(formula = y ~ factor(sex) + factor(Dep) + offset(log(nb)),
##     family = poisson)
##
## Deviance Residuals:
##        1         2         3         4         5
## -0.68882  -0.01474   0.96655   0.02569   0.97713
##        6         7         8         9        10
## -0.28371   1.77895   0.06756  -0.71131  -0.02632
##       11        12
## -0.68503   0.28254
##
## Coefficients:
##              Estimate Std. Error z value Pr(>|z|)
## (Intercept)  -0.44677    0.04148 -10.771   <2e-16 ***
## factor(sex)1  0.05859    0.06166   0.950    0.342
## factor(Dep)1 -0.01391    0.06625  -0.210    0.834
## factor(Dep)2 -0.63911    0.07660  -8.344   <2e-16 ***
## factor(Dep)3 -0.66125    0.07675  -8.615   <2e-16 ***
## factor(Dep)4 -0.97250    0.09836  -9.887   <2e-16 ***
## factor(Dep)5 -2.32388    0.15468 -15.024   <2e-16 ***
## ---
## Signif. codes:
## 0 '***' 0.001 '**' 0.01 '*' 0.05 '.' 0.1 ' ' 1
##
## (Dispersion parameter for poisson family taken to be 1)
##
##     Null deviance: 551.6926  on 11  degrees of freedom
## Residual deviance:   6.6698  on  5  degrees of freedom
## AIC: 96.868
##
## Number of Fisher Scoring iterations: 3
\end{verbatim}
\end{kframe}
\end{knitrout}

Dans ce modèle, le résultat du test de Wald pour le coefficient de la variable Sexe est différent. Puisque le seuil observé du test est 34.5\%, on ne peut pas rejeter l'hypothèse nulle que $\beta^{SEX}=0$. Cela signifie que le sexe n'est pas un facteur qui influence le taux d'admission aux études graduées lorsqu'on prend en considération le département. Il en est ainsi car les femmes appliquent plus souvent que les hommes dans des départements où il est plus difficile d'être admis.

\item À l'aide de l'analyse de la déviance, on trouve que l'interaction n'est pas significative:
$$\Delta Deviance = 6.67 < \chi^2 (0.95,5) = 11.07.$$
\begin{knitrout}
\definecolor{shadecolor}{rgb}{0.969, 0.969, 0.969}\color{fgcolor}\begin{kframe}
\begin{alltt}
\hlstd{fitp} \hlkwb{<-} \hlkwd{glm}\hlstd{(y}\hlopt{~}\hlkwd{factor}\hlstd{(sex)}\hlopt{*}\hlkwd{factor}\hlstd{(Dep)}\hlopt{+}\hlkwd{offset}\hlstd{(}\hlkwd{log}\hlstd{(nb)),}\hlkwc{family}\hlstd{=poisson)}
\hlkwd{anova}\hlstd{(fitp)}
\end{alltt}
\begin{verbatim}
## Analysis of Deviance Table
##
## Model: poisson, link: log
##
## Response: y
##
## Terms added sequentially (first to last)
##
##
##                         Df Deviance Resid. Df
## NULL                                       11
## factor(sex)              1    58.13        10
## factor(Dep)              5   486.89         5
## factor(sex):factor(Dep)  5     6.67         0
##                         Resid. Dev
## NULL                        551.69
## factor(sex)                 493.56
## factor(Dep)                   6.67
## factor(sex):factor(Dep)       0.00
\end{verbatim}
\begin{alltt}
\hlkwd{qchisq}\hlstd{(}\hlnum{0.95}\hlstd{,}\hlnum{5}\hlstd{)} \hlcom{## reject interaction}
\end{alltt}
\begin{verbatim}
## [1] 11.0705
\end{verbatim}
\end{kframe}
\end{knitrout}

\item Le modèle final est celui avec une seule variable explicative dichotomique: le département. La déviance pour ce modèle est 7.5706, ce qui est légèrement supérieur à 6, le nombre de degrés de liberté. Toutefois, puisque $Deviance/dl\approx1.26$, cela n'est pas très alarmant, et il n'y a pas de raison de supposer que le modèle de Poisson est inadéquat. La statistique de Pearson est  8.03, ce qui est aussi une valeur attendue pour la loi chi-carrée avec 6 degrés de liberté.
\begin{knitrout}
\definecolor{shadecolor}{rgb}{0.969, 0.969, 0.969}\color{fgcolor}\begin{kframe}
\begin{alltt}
\hlstd{fitpDep} \hlkwb{<-} \hlkwd{glm}\hlstd{(y}\hlopt{~}\hlkwd{factor}\hlstd{(Dep)}\hlopt{+}\hlkwd{offset}\hlstd{(}\hlkwd{log}\hlstd{(nb)),}\hlkwc{family}\hlstd{=poisson)}
\hlkwd{summary}\hlstd{(fitpDep)}
\end{alltt}
\begin{verbatim}
##
## Call:
## glm(formula = y ~ factor(Dep) + offset(log(nb)), family = poisson)
##
## Deviance Residuals:
##     Min       1Q   Median       3Q      Max
## -0.8481  -0.4187   0.1160   0.4595   2.2321
##
## Coefficients:
##              Estimate Std. Error z value Pr(>|z|)
## (Intercept)  -0.43981    0.04079 -10.782   <2e-16 ***
## factor(Dep)1 -0.01830    0.06608  -0.277    0.782
## factor(Dep)2 -0.60784    0.06906  -8.801   <2e-16 ***
## factor(Dep)3 -0.64004    0.07336  -8.725   <2e-16 ***
## factor(Dep)4 -0.93966    0.09201 -10.212   <2e-16 ***
## factor(Dep)5 -2.30243    0.15298 -15.050   <2e-16 ***
## ---
## Signif. codes:
## 0 '***' 0.001 '**' 0.01 '*' 0.05 '.' 0.1 ' ' 1
##
## (Dispersion parameter for poisson family taken to be 1)
##
##     Null deviance: 551.6926  on 11  degrees of freedom
## Residual deviance:   7.5706  on  6  degrees of freedom
## AIC: 95.769
##
## Number of Fisher Scoring iterations: 4
\end{verbatim}
\begin{alltt}
\hlkwd{sum}\hlstd{((y}\hlopt{-}\hlkwd{fitted}\hlstd{(fitpDep))}\hlopt{^}\hlnum{2}\hlopt{/}\hlkwd{fitted}\hlstd{(fitpDep))}
\end{alltt}
\begin{verbatim}
## [1] 8.025236
\end{verbatim}
\begin{alltt}
\hlkwd{pchisq}\hlstd{(}\hlnum{8.025236}\hlstd{,}\hlnum{6}\hlstd{)}
\end{alltt}
\begin{verbatim}
## [1] 0.7637397
\end{verbatim}
\end{kframe}
\end{knitrout}

\item On recommence et on obtient exactement les mêmes conclusions:
\begin{knitrout}
\definecolor{shadecolor}{rgb}{0.969, 0.969, 0.969}\color{fgcolor}\begin{kframe}
\begin{alltt}
\hlstd{fitbSex} \hlkwb{<-} \hlkwd{glm}\hlstd{(}\hlkwd{cbind}\hlstd{(y,nb}\hlopt{-}\hlstd{y)}\hlopt{~}\hlkwd{factor}\hlstd{(sex),}\hlkwc{family}\hlstd{=binomial)}
\hlkwd{summary}\hlstd{(fitbSex)}
\end{alltt}
\begin{verbatim}
##
## Call:
## glm(formula = cbind(y, nb - y) ~ factor(sex), family = binomial)
##
## Deviance Residuals:
##      Min        1Q    Median        3Q       Max
## -16.7915   -4.7613   -0.4365    5.1025   11.2022
##
## Coefficients:
##              Estimate Std. Error z value Pr(>|z|)
## (Intercept)  -0.22013    0.03879  -5.675 1.38e-08 ***
## factor(sex)1 -0.61035    0.06389  -9.553  < 2e-16 ***
## ---
## Signif. codes:
## 0 '***' 0.001 '**' 0.01 '*' 0.05 '.' 0.1 ' ' 1
##
## (Dispersion parameter for binomial family taken to be 1)
##
##     Null deviance: 877.06  on 11  degrees of freedom
## Residual deviance: 783.61  on 10  degrees of freedom
## AIC: 856.55
##
## Number of Fisher Scoring iterations: 4
\end{verbatim}
\begin{alltt}
\hlstd{fitb2} \hlkwb{<-} \hlkwd{glm}\hlstd{(}\hlkwd{cbind}\hlstd{(y,nb}\hlopt{-}\hlstd{y)}\hlopt{~}\hlkwd{factor}\hlstd{(sex)}\hlopt{+}\hlkwd{factor}\hlstd{(Dep),}\hlkwc{family}\hlstd{=binomial)}
\hlkwd{summary}\hlstd{(fitb2)}
\end{alltt}
\begin{verbatim}
##
## Call:
## glm(formula = cbind(y, nb - y) ~ factor(sex) + factor(Dep), family = binomial)
##
## Deviance Residuals:
##       1        2        3        4        5        6
## -1.2487  -0.0560   1.2533   0.0826   1.2205  -0.2076
##       7        8        9       10       11       12
##  3.7189   0.2706  -0.9243  -0.0858  -0.8509   0.2052
##
## Coefficients:
##              Estimate Std. Error z value Pr(>|z|)
## (Intercept)   0.58205    0.06899   8.436   <2e-16 ***
## factor(sex)1  0.09987    0.08085   1.235    0.217
## factor(Dep)1 -0.04340    0.10984  -0.395    0.693
## factor(Dep)2 -1.26260    0.10663 -11.841   <2e-16 ***
## factor(Dep)3 -1.29461    0.10582 -12.234   <2e-16 ***
## factor(Dep)4 -1.73931    0.12611 -13.792   <2e-16 ***
## factor(Dep)5 -3.30648    0.16998 -19.452   <2e-16 ***
## ---
## Signif. codes:
## 0 '***' 0.001 '**' 0.01 '*' 0.05 '.' 0.1 ' ' 1
##
## (Dispersion parameter for binomial family taken to be 1)
##
##     Null deviance: 877.056  on 11  degrees of freedom
## Residual deviance:  20.204  on  5  degrees of freedom
## AIC: 103.14
##
## Number of Fisher Scoring iterations: 4
\end{verbatim}
\begin{alltt}
\hlstd{fitb} \hlkwb{<-} \hlkwd{glm}\hlstd{(}\hlkwd{cbind}\hlstd{(y,nb}\hlopt{-}\hlstd{y)}\hlopt{~}\hlkwd{factor}\hlstd{(sex)}\hlopt{*}\hlkwd{factor}\hlstd{(Dep),}\hlkwc{family}\hlstd{=binomial)}
\hlkwd{anova}\hlstd{(fitb)}
\end{alltt}
\begin{verbatim}
## Analysis of Deviance Table
##
## Model: binomial, link: logit
##
## Response: cbind(y, nb - y)
##
## Terms added sequentially (first to last)
##
##
##                         Df Deviance Resid. Df
## NULL                                       11
## factor(sex)              1    93.45        10
## factor(Dep)              5   763.40         5
## factor(sex):factor(Dep)  5    20.20         0
##                         Resid. Dev
## NULL                        877.06
## factor(sex)                 783.61
## factor(Dep)                  20.20
## factor(sex):factor(Dep)       0.00
\end{verbatim}
\begin{alltt}
\hlkwd{qchisq}\hlstd{(}\hlnum{0.95}\hlstd{,}\hlnum{5}\hlstd{)}
\end{alltt}
\begin{verbatim}
## [1] 11.0705
\end{verbatim}
\end{kframe}
\end{knitrout}
\end{enumerate}
\end{solution}
\begin{solution}{6.6}

\begin{enumerate}
If $Y_{i}\sim Poisson(E_{i}\lambda_{i})$, then, using the canonical link, $$\log(\mu_{i})=log(E_{i})+log(\lambda_{i}),$$ where $\lambda_{i}$ is the mean $SMR$ for observation $i$. $\log(E_{i})$, the natural logarithm of the expected count of lung cancer based on the demographics of the county, is passed to the \texttt{glm} function as an offset factor.

The data for males and females are concatenated to create a model with one covariate, Radon exposure, and one factor predictor, Sex, which takes 2 levels (0 for males and 1 for females).
\begin{knitrout}
\definecolor{shadecolor}{rgb}{0.969, 0.969, 0.969}\color{fgcolor}\begin{kframe}
\begin{alltt}
\hlstd{Ytot} \hlkwb{<-} \hlkwd{c}\hlstd{(YM,YF)}
\hlstd{Etot} \hlkwb{<-} \hlkwd{c}\hlstd{(EM,EF)}
\hlstd{Sex} \hlkwb{<-} \hlkwd{c}\hlstd{(}\hlkwd{rep}\hlstd{(}\hlnum{0}\hlstd{,}\hlkwd{length}\hlstd{(YM)),}\hlkwd{rep}\hlstd{(}\hlnum{1}\hlstd{,}\hlkwd{length}\hlstd{(YF)))} \hlcom{## 1 if female}
\hlstd{Radontot} \hlkwb{<-} \hlkwd{rep}\hlstd{(Radon,}\hlnum{2}\hlstd{)}

\hlstd{modsex} \hlkwb{<-} \hlkwd{glm}\hlstd{(Ytot}\hlopt{~}\hlstd{Sex}\hlopt{+}\hlkwd{offset}\hlstd{(}\hlkwd{log}\hlstd{(Etot)),}\hlkwc{family}\hlstd{=poisson)}
\hlstd{modsexrad} \hlkwb{<-} \hlkwd{glm}\hlstd{(Ytot}\hlopt{~}\hlstd{Radontot}\hlopt{+}\hlstd{Sex}\hlopt{+}\hlkwd{offset}\hlstd{(}\hlkwd{log}\hlstd{(Etot)),}\hlkwc{family}\hlstd{=poisson)}

\hlkwd{anova}\hlstd{(modsex,modsexrad)}
\end{alltt}
\begin{verbatim}
## Analysis of Deviance Table
##
## Model 1: Ytot ~ Sex + offset(log(Etot))
## Model 2: Ytot ~ Radontot + Sex + offset(log(Etot))
##   Resid. Df Resid. Dev Df Deviance
## 1       172     410.27
## 2       171     364.05  1    46.22
\end{verbatim}
\begin{alltt}
\hlkwd{qchisq}\hlstd{(}\hlnum{0.99}\hlstd{,}\hlnum{1}\hlstd{)}
\end{alltt}
\begin{verbatim}
## [1] 6.634897
\end{verbatim}
\end{kframe}
\end{knitrout}

As shown above, the analysis of deviance shows strong evidence that the radon exposure influences the number of lung cancer in a particular county:
\begin{align*}
\Delta Deviance = 46.22 > \chi^{2}_{(1;0.99)}=6.6349.
\end{align*}

\item The null model and the model including sex only are fitted.
\begin{knitrout}
\definecolor{shadecolor}{rgb}{0.969, 0.969, 0.969}\color{fgcolor}\begin{kframe}
\begin{alltt}
\hlstd{modtot} \hlkwb{<-} \hlkwd{glm}\hlstd{(Ytot}\hlopt{~}\hlnum{1}\hlopt{+}\hlkwd{offset}\hlstd{(}\hlkwd{log}\hlstd{(Etot)),}\hlkwc{family}\hlstd{=poisson)}
\hlstd{modsex} \hlkwb{<-} \hlkwd{glm}\hlstd{(Ytot}\hlopt{~}\hlstd{Sex}\hlopt{+}\hlkwd{offset}\hlstd{(}\hlkwd{log}\hlstd{(Etot)),}\hlkwc{family}\hlstd{=poisson)}
\hlkwd{anova}\hlstd{(modtot,modsex)}
\end{alltt}
\begin{verbatim}
## Analysis of Deviance Table
##
## Model 1: Ytot ~ 1 + offset(log(Etot))
## Model 2: Ytot ~ Sex + offset(log(Etot))
##   Resid. Df Resid. Dev Df  Deviance
## 1       173     410.28
## 2       172     410.27  1 0.0093398
\end{verbatim}
\begin{alltt}
\hlkwd{qchisq}\hlstd{(}\hlnum{0.95}\hlstd{,}\hlnum{1}\hlstd{)}
\end{alltt}
\begin{verbatim}
## [1] 3.841459
\end{verbatim}
\end{kframe}
\end{knitrout}
The analysis of deviance shows that $\Delta Deviance= 0.0093398<\chi^{2}_{(1;0.95)}=3.8415$. Hence, the null model is an appropriate simplification of the model including the factor Sex, so the factor is not significant. However, below is the \texttt{R} output for the analysis of deviance when the covariate Radon (known to be significant from a) is included in the model. If we first consider the model with main effects and interactions, we see that $\Delta Deviance= 8.823>\chi^{2}_{(1;0.99)}$, meaning that the model with main effects only is not an adequate simplification of the model with main effects and interactions. Thus, the factor predictor Sex is significant in the model through its interaction with the covariate Radon. Note that even if the main effect of the Sex does not appear to be significant, it is kept in the model by convention.

\begin{knitrout}
\definecolor{shadecolor}{rgb}{0.969, 0.969, 0.969}\color{fgcolor}\begin{kframe}
\begin{alltt}
\hlstd{modtotrad} \hlkwb{<-} \hlkwd{glm}\hlstd{(Ytot}\hlopt{~}\hlstd{Radontot}\hlopt{+}\hlkwd{offset}\hlstd{(}\hlkwd{log}\hlstd{(Etot)),}\hlkwc{family}\hlstd{=poisson)}
\hlstd{modsexrad} \hlkwb{<-} \hlkwd{glm}\hlstd{(Ytot}\hlopt{~}\hlstd{Radontot}\hlopt{+}\hlstd{Sex}\hlopt{+}\hlkwd{offset}\hlstd{(}\hlkwd{log}\hlstd{(Etot)),}\hlkwc{family}\hlstd{=poisson)}
\hlstd{modsexradINT} \hlkwb{<-} \hlkwd{glm}\hlstd{(Ytot}\hlopt{~}\hlstd{Radontot}\hlopt{*}\hlstd{Sex}\hlopt{+}\hlkwd{offset}\hlstd{(}\hlkwd{log}\hlstd{(Etot)),}\hlkwc{family}\hlstd{=poisson)}
\hlkwd{anova}\hlstd{(modtot,modtotrad,modsexrad,modsexradINT)}
\end{alltt}
\begin{verbatim}
## Analysis of Deviance Table
##
## Model 1: Ytot ~ 1 + offset(log(Etot))
## Model 2: Ytot ~ Radontot + offset(log(Etot))
## Model 3: Ytot ~ Radontot + Sex + offset(log(Etot))
## Model 4: Ytot ~ Radontot * Sex + offset(log(Etot))
##   Resid. Df Resid. Dev Df Deviance
## 1       173     410.28
## 2       172     364.06  1   46.219
## 3       171     364.05  1    0.011
## 4       170     355.23  1    8.823
\end{verbatim}
\end{kframe}
\end{knitrout}

\item The predictions are obtained using the command
\begin{knitrout}
\definecolor{shadecolor}{rgb}{0.969, 0.969, 0.969}\color{fgcolor}\begin{kframe}
\begin{alltt}
\hlkwd{predict}\hlstd{(modsexradINT,}\hlkwd{data.frame}\hlstd{(}\hlkwc{Radontot}\hlstd{=}\hlnum{6}\hlstd{,}\hlkwc{Sex}\hlstd{=}\hlnum{0}\hlstd{,}\hlkwc{Etot}\hlstd{=}\hlnum{1}\hlstd{),}\hlkwc{type}\hlstd{=}\hlstr{"response"}\hlstd{,}\hlkwc{se.fit}\hlstd{=}\hlnum{TRUE}\hlstd{)}
\end{alltt}
\end{kframe}
\end{knitrout}
 If the model Sex*Radon is used, we find $$\hat{SMR}_{Sex=0,Radon=6}=0.9708183,$$ with a standard error of 0.01415307.

\item The model Sex*Radon has a deviance of 355.23 on 170 degrees of freedom. A heuristic check for the validity of the model is to calculate the estimated dispersion parameter $$\hat{\phi}=\frac{355.23}{170}=2.089$$ and to compare it with 1, the dispersion parameter implied in the Poisson model. This check suggests the presence of overdispersion in the data as $\hat{\phi}$ is greater than 1. Fitting the quasipoisson model also leads to the same conclusion: the estimated dispersion parameter is 1.98311, closer to 2. Thus, we can conclude that the Poisson model is not adequate, we might consider fitting a Negative Binomial model to capture the overdispersion.
\end{enumerate}
\end{solution}

\newpage
\input{solutions-binom}

%%% Local Variables:
%%% mode: latex
%%% TeX-master: "exercices_methodes_statistiques"
%%% End:


%\nocite{Miller:stat:1977}

%\bibliography{stat,vg,r} %%% à arranger

\cleardoublepage
%\printindex %%% à vérifier

\cleardoublepage
\cleartoverso

\pagestyle{empty}
\renewcommand{\ttdefault}{hlst}

\bandeverso
\begin{textblock*}{71mm}(135mm, -50mm)
  \textblockcolor{}
%  \includegraphics{codebarre}
\end{textblock*}

\end{document}

%%% Local Variables:
%%% mode: latex
%%% TeX-master: t
%%% End:
